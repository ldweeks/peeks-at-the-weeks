\PassOptionsToPackage{unicode=true}{hyperref} % options for packages loaded elsewhere
\PassOptionsToPackage{hyphens}{url}
%
\documentclass[]{book}
\usepackage{lmodern}
\usepackage{amssymb,amsmath}
\usepackage{ifxetex,ifluatex}
\usepackage{fixltx2e} % provides \textsubscript
\ifnum 0\ifxetex 1\fi\ifluatex 1\fi=0 % if pdftex
  \usepackage[T1]{fontenc}
  \usepackage[utf8]{inputenc}
  \usepackage{textcomp} % provides euro and other symbols
\else % if luatex or xelatex
  \usepackage{unicode-math}
  \defaultfontfeatures{Ligatures=TeX,Scale=MatchLowercase}
\fi
% use upquote if available, for straight quotes in verbatim environments
\IfFileExists{upquote.sty}{\usepackage{upquote}}{}
% use microtype if available
\IfFileExists{microtype.sty}{%
\usepackage[]{microtype}
\UseMicrotypeSet[protrusion]{basicmath} % disable protrusion for tt fonts
}{}
\IfFileExists{parskip.sty}{%
\usepackage{parskip}
}{% else
\setlength{\parindent}{0pt}
\setlength{\parskip}{6pt plus 2pt minus 1pt}
}
\usepackage{hyperref}
\hypersetup{
            pdftitle={Peeks at the Weeks},
            pdfauthor={Helen Weeks},
            pdfborder={0 0 0},
            breaklinks=true}
\urlstyle{same}  % don't use monospace font for urls
\usepackage{longtable,booktabs}
% Fix footnotes in tables (requires footnote package)
\IfFileExists{footnote.sty}{\usepackage{footnote}\makesavenoteenv{longtable}}{}
\usepackage{graphicx,grffile}
\makeatletter
\def\maxwidth{\ifdim\Gin@nat@width>\linewidth\linewidth\else\Gin@nat@width\fi}
\def\maxheight{\ifdim\Gin@nat@height>\textheight\textheight\else\Gin@nat@height\fi}
\makeatother
% Scale images if necessary, so that they will not overflow the page
% margins by default, and it is still possible to overwrite the defaults
% using explicit options in \includegraphics[width, height, ...]{}
\setkeys{Gin}{width=\maxwidth,height=\maxheight,keepaspectratio}
\setlength{\emergencystretch}{3em}  % prevent overfull lines
\providecommand{\tightlist}{%
  \setlength{\itemsep}{0pt}\setlength{\parskip}{0pt}}
\setcounter{secnumdepth}{5}
% Redefines (sub)paragraphs to behave more like sections
\ifx\paragraph\undefined\else
\let\oldparagraph\paragraph
\renewcommand{\paragraph}[1]{\oldparagraph{#1}\mbox{}}
\fi
\ifx\subparagraph\undefined\else
\let\oldsubparagraph\subparagraph
\renewcommand{\subparagraph}[1]{\oldsubparagraph{#1}\mbox{}}
\fi

% set default figure placement to htbp
\makeatletter
\def\fps@figure{htbp}
\makeatother

\usepackage{etoolbox}
\makeatletter
\providecommand{\subtitle}[1]{% add subtitle to \maketitle
  \apptocmd{\@title}{\par {\large #1 \par}}{}{}
}
\makeatother
\usepackage{booktabs}
\usepackage{amsthm}
\makeatletter
\def\thm@space@setup{%
  \thm@preskip=8pt plus 2pt minus 4pt
  \thm@postskip=\thm@preskip
}
\makeatother
% https://github.com/rstudio/rmarkdown/issues/337
\let\rmarkdownfootnote\footnote%
\def\footnote{\protect\rmarkdownfootnote}

% https://github.com/rstudio/rmarkdown/pull/252
\usepackage{titling}
\setlength{\droptitle}{-2em}

\pretitle{\vspace{\droptitle}\centering\huge}
\posttitle{\par}

\preauthor{\centering\large\emph}
\postauthor{\par}

\predate{\centering\large\emph}
\postdate{\par}
\usepackage[]{natbib}
\bibliographystyle{apalike}

\title{Peeks at the Weeks}
\author{Helen Weeks}
\date{}

\begin{document}
\maketitle

{
\setcounter{tocdepth}{1}
\tableofcontents
}
\hypertarget{a-weeks-family-book}{%
\chapter*{\texorpdfstring{\href{https://beggarsborn.com/peeks-at-the-weeks/}{A Weeks Family Book}}{A Weeks Family Book}}\label{a-weeks-family-book}}
\addcontentsline{toc}{chapter}{\href{https://beggarsborn.com/peeks-at-the-weeks/}{A Weeks Family Book}}

This is where I can put some text.

This is where I can put some more text.

\hypertarget{cover}{%
\subsubsection*{Cover}\label{cover}}
\addcontentsline{toc}{subsubsection}{Cover}

\hypertarget{introduction}{%
\chapter*{Introduction}\label{introduction}}
\addcontentsline{toc}{chapter}{Introduction}

\emph{Christmas, 1988}

Every now and then someone mentions that I should write a book based on our old letters. But this is a project that I have always just pushed aside for some future date---when all the other projects are finished.

Then by chance Steven and Julie saw a few of these old letters in which some ``cute'' things some of you had said or done were mentioned. When Julie reported you all enjoyed them when you were together in Kinshasa, I decided to hunt up a few more that I knew were somewhere in the letters.

So one thing led to another and without intending to compile a book, a book of quotes from old letters seems to have emerged. The first is from a general letter which went to our complete mailing list. Other than that, all of these quotes are from letters to my parents unless a note indicates differently.

I have tried to include in this the important happenings in our life during those years as they pertained to the Weeks. You will note that very little is included about the work, the lives of the local people, etc. All this is another story. The question was, where to stop. The letters go on and on, if not to your grandparents, then to you once you were in the U.S. in college or else in the hostel. So mostly I've tried to get you through the hostel years and until you were more or less on your own.

You were all Such Cute Kids! And I did so love to write all about your doings and sayings and I tried to share your growing up with your grandparents. I hope you can now enjoy reliving the past and sharing it with your spouses and your children.

Love,

Mom

\hypertarget{peaks-at-the-weeks-nineteen-forty-nine}{%
\chapter{Peaks at the Weeks: Nineteen Forty-nine}\label{peaks-at-the-weeks-nineteen-forty-nine}}

\textbf{January 26, 1949}

Dear friends,

The Christmas season has come and gone. Was it as joyous for you as it was for us? Our most wonderful gift was the arrival on Christmas morning of baby Linda Lou! She is a present that grows sweeter as the days go by. She weighed 8 lbs. 6 oz. was twenty inches long and is a perfect beauty. We had all been worrying if her arrival was going to interfere with other Christmas plans but if she times all her arrivals as well as she did this one, she nor her friends will have to worry. The Christmas pageant by the village school had been on Friday afternoon, then the Christmas story in pictures was shown in an outside service on Christmas eve after which we missionaries had our gift exchange and a little party. Than Linda arrived at 10:00 A.M. and that night all enjoyed the annual Christmas dinner.

The Congolese were almost as tickled as we were over it all, especially the fact that she came on noel. They all wanted to see her. Some of our people had never seen a white baby. One of the women remarked--why she leeks just like one of our babies. Their babies are white for the first few days. It is the custom to name each white person for some Christian native who has recently died. Just after one of our young couples had made all preparations to go to the Congo Christian Institute at Bolenge for their further training, the wife died in child birth. Linda Lou is now Botoka Malia in her honor.

\textbf{May 28, 1949}

Linda is growing like a weed. She has two teeth, sits alone and crawls as big as life. She absolutely refuses to stay on her blanket, I have a boy watch her to keep her on but he was gone for a minute and Claylon said he thought she would be O.K. as she was lying quietly but by the time he had put on a tie she
had gotten clear off the blanket and was scooting across the fleer. she cries so seldom that Mrs.~Boyer has been declaring that she couldn't but this afternoon Linda showed her. We went down to visit the new state people for tea and everyone was saying their ``hellos'' and making over Linda and she really let loose and kept on until I was almost ashamed of her after having bragged about her conduct. She isn't used to crowds and even the place looked strange as we have scarcely had a chance to have her away from home.

\textbf{Sept.~13, 1949}

(From the Penn Sheraton Hotel in Philadelphia)

Hi from good old U.S.A.! We docked about noon here at Philadelphia but by the time we finished customs and got to the hotel it was nearly 5 P.M. so Ells (Lewis) and Claylon left Lil and me with all the kids and dashed off to New York for the evening. They were to see Emory Ross, pick up any mail, get our clergy permits, etc. and spend the evening with the Rosses. So Lil and I had the fun of taking all the young ones right from the heart of Africa to a restaurant. Mary Jo, at 6 1/2 should be sort of self reliant but she's all eyes and ears and can't pay much attention. Casey at 2 1/2 is impossible. Knows no fear and pays no attention to anyone. Then Mark at 13 mo. and Linda at 9 are our babies. Our first obstacle was the revolving door that Casey had to stop and examine when half way through. Red lights meant nothing to them but the red and white stripes on the barber pole across the street was a great fascination. We finally managed to get supper ordered and eaten but not without some struggles but half-way through Mary Jo says, ``My, but isn't this fun and aren't we glad our husbands aren't here?'' Lil has harnesses for both little boys so she harnessed them up and we `drove' out of the restaurant. We have been wondering ever since we arrived at the hotel if people stare at us because we look as queer as we feel, if it's the children, or if we're just not used to people.

Linda Lou is a little dear. She has acted all day as if she knew something exciting was on foot and she didn't want to miss out. She hasn't been fussy but just jumps up and down and won't let us out of her sight for fear she will be left behind. She is a wonderful traveler, seems to love every minute of it and every person she sees.

We had an uneventful crossing. Just 15 days from Matadi and very wonderful weather. I'm no sailor so was sick a lot but that gave Claylon lots of experience at his housewifely duties and he came through fine. It was just the opposite with the Lewises. Ells was sick so Lil had it all to do. Still can't realize we've arrived. Can you imagine - 3 months ago yesterday we left Wema. Nothing speedy about that, is there?

(During the rest of 1949 and 1950 we were in the U.S. so there are no long letters about any of us. Since Ronnie was born in a hospital at Rochester, Indiana on Feb.~27, 1950 there is not much recorded about his earliest days or antics. It is only when we start traveling again that he comes into his own.)

\hypertarget{peaks-at-the-weeks-nineteen-fifty}{%
\chapter{Peaks at the Weeks: Nineteen Fifty}\label{peaks-at-the-weeks-nineteen-fifty}}

December 20, 1950

(From a publicity letter written just before I left the U.S.)

Linda Lou, Ronnie Dee and I are sailing January 5, 1951 on the Queen Mary and hope to be met at Cherbourg, France by Claylon who left the U.S. in August far a year's study in Brussels, Belgium. I am anticipating my trip more now that I an sure my sister, Hilda Mitchell, is going along. She has a leave of absence from the Methodist Hospital in Indianapolis and will have the experience of the trip and of seeing Europe for about seven weeks.

\hypertarget{section}{%
\chapter{1951}\label{section}}

Feb.~3, 1951

(From a letter of Hilda's to the folks)

Ronnie is walking with holding to someone with one hand but won't walk alone yet--the lazy bum! But he is cute as a bug's ear. Played in the play pen for about an hour this morning before I put him to bed. He makes all kinds of noises, says Ma Ma, Da Da and grins all over the place-is fat as a little butterball and eats everything he can lay hands on. Linda came strolling out last night at about 9:45 after I got here (Hilda had just returned from a trip to Rome) - we were talking too loud as usual. She surely hadn't forgotten me in a week for she hugged me good and hard and crawled all over me. When she finally fell asleep, she had her Book of Prayers open, dropped down on her chest just as if she had fallen asleep reading her prayers.

Feb.~9. 1951

(Again fron Hilda)

Helen and Claylon have gone te town to sign Helen and the children in at the American Consulate. I am baby sitting. Oh! what kids your grandchildren are. Linda has learned to crawl in and out of the play pen by herself and does it often. She pulls a chair up to the side and climbs over. She is very independent and has to be handled with kid gloves part ef the time to get her to do what she must. She is still as sweet and loving as ever most of the time. Ronnie is growing like a weed, and in a very short time he is going to be defending himself adequately when his sister starts pushing him around. Incidentally, she does that quite often. He has taken a step or two a few times but still is not walking by himself. Their ``Auntie Bea'' brought them each a helium filled balloon last night and you should have seen the excitement. Ronnie's eyes got as big as silver dollars, and he was so excited he shook all over.

Feb.~21, 1951

We have been having baby troubles! Briefly it's that Linda doesn't want to go to bed and no matter whether we put her in at 7:30 or 10:30 she wants to wake Ronnie and play. She can climb into his bed and I know of no way to keep her out except to solidly screen it. We have moved everything movable away from it but still she gets in. We've pleaded, entertained her for long sessions, spanked, threatened, and have done everything we know to do except lay her out but still we have problems. It is better than when Hilda was here. The day that Hilda left I lay down with Linda and she was asleep in 10 minutes so thought that was the solution but last night and tonight that didn't work, though it does for afternoon naps.

Ronnie still doesn't walk alone much. Hilda may tell you about his first steps. Claylon bought some popcorn and fed Ronnie 2 or 3 grains. Then he put some on a chair which just happened to be about 2 steps away and Ronnie simply flew over to them and then repeated the performance twice more but he hasn't done it since. Of course he hasn't had any popcorn since!

Later: Today (the 24th) Ronnie took several steps and does he think he's somebody! It sticks out all over him. He has been fussy again lately but today one of those bottom teeth is finally through and he was as chipper as a squirrel. The other gum is swollen about 3 times normal size so surely only a few more days for it. If he thinks he is not getting his share of attention he throws up his hand and says, ``Hi, there.'' and then looks like the cat that swallowed the canary. Quite often I wish you had him at least for a while. He absolutely refuses to be fed. Everything is to be put before him and he helps himself - sometimes even with a spoon. He'll eat eggs for breakfast and leave the cereal because I won't give it to him to feed to himself. Two little independent ones we have and you should see them when they cross one another. Ronnie doesn't take much now but I can already see we are going to have to watch the discipline. If we scold Linda for mistreating him, then he cries at any little thing so he'll be petted. If we have Linda love him, then she hits him so he'll cry and she can put her arms around him and say, ``Don't cry, Ronnie. Linda loves you.''

March 28, 1951

(This from Claylon)

Wish you could see the kids. Ronnie is entirely a walker now. He has a fight at the table because he wants to hold his own spoon and he usually wins. He can do it well enough for his age but you should see him after having chocolate pudding, which incidentally is the favorite dessert for both the kids. He loves books as well as Linda ever did and has a few of his own though he is prone to be very hard on them.

Linda is doing so many things. She talks like a parrot. She repeats any kind of word or phrase which she hears in addition to making her own. Her favorite now is telling everyone to be careful. ``Be careful, mommy. Be careful Ronnie.'' And she always wants to be kissed when she hurts herself. Her mommy is always kissing her head or fingers. The best one was when she slid down and sat down hard hurting her backside and then came out saying, ``Mommy, kiss it.''

April 22, 1951

Linda pulls some fast ones now and then. Yesterday she wiggled so I was quite sure she needed to go to the bathroom but once we go there I was having quite a struggle to get her to sit. Was in fact holder he with main force when she says very emphatically, ``Now Mommy, don't make me mad!'' And then when I still held on, ``Mommy, I mean it, you're going to make me mad!'' Wonder where she ever heard that. She talks constantly simply gasping for breath sometimes until I think I will surely go crazy.

Big old Ronnie has taken to pushing chairs around, climbing up onto them and then going wherever he likes. A few days ago I found him sitting on the desk surrounded with fountain pens, pencils, etc. Then just this evening Bea found him in the kitchen sink. We from shoes on up. He is so pesky quick that you think you know where he is and he's up and into something more interesting.

May 13, 1951

Ronnie gets cuter all the time. He climbs into and onto everything. They both get on the table in our room; up into the window or anywhere else available. When he gets spanked for going places he is not allowed he pouts and puckers up just like Linda used to even if he isn't hurt a bit. He has quite a vocabulary, we think, for 15 months for he says bow-wow whenever he sees a dog either in a picture or on the street. He says book, points to one and wants to be read to. He also says spoon, high chair, and wanna-go-out so that they are quite easily understood. He has 13 teeth with the three others showing a bit white as if it wouldn't be long but I can't see that he has any more hair than when we left and what he has isn't long enough to hardly stand up by itself. I am beginning to think he is going to just be a baldy. He is good at pulling Linda's hair and today I saw her trying to get hold of his without success. He is a little old dickens about some things. He still wants a bottle at night but we never know when. It may be midnight or 6am. He wakes up, grunts a little, then talks a little and we hope he'll talk himself to sleep, then soon he begins to get angry and so gets louder and louder. Ignoring him does no good unless we could put up with it all night. I think I could but Claylon is afraid the neighbors might not like it so after so long a time he gives in. Ronnie doesn't bother to stand up or get out from under his covers, he just lies on his tummy and yells. This A.M. he had his head turned to the wall so didn't even bother to look at me. He could hear me coming so just stuck his arm out behind his back, took to the bottle, grunted and went to work.

May 22, 1951

The package with the children's books came the other day and have by now been read through and through. Ronnie is as bad as Linda ever was about a ``book, a book'' and he hangs on to my tail until he gets it and has it read to him. He is different in that instead of sitting quietly and looking at it himself he walks around with it opened in front of him and read aloud. His reading aloud and talking is equal to what his crying used to be in volume! He jabbers lots more than Linda ever did but he is beginning to get quite a vocabulary. He climbs like a monkey, gets up on and then off Linda's tricycle all alone. Gets all seated and then wants to be pushed. They are some pair. Linda says ``Where's my Ronnie? Ronnie boy, where are you?''

July 8, 1951

Linda loves jig-saw puzzles better than most anything. We asked what she wanted from London and it was a puzzle and a book so that's what she got. After help in putting a 15 piece puzzle togetehr once she can figure it out herself and will do it over and over for an hour at a time. They are big wooden pieces, very lovely and durable.

They didn't miss us in the least, it seems. (We had gone to London from Monday till Friday.) Bea said Linda asked for her daddy once and when she first saw me Saturday morning she languidly says, ``Oh, I see Mommy's back.'' I said, ``How are you, Linda?'' and she very formally replied, ``i'm just fine, thank you.'' and that was that.

There are both three day measles and mumps in the neighborhood. This afternoon Linda has been extra cross and has tried that her ear aches so don't know if its cold or mumps but soon shall I suppose.

July 15, 1951

Linda really has the mumps but they don't bother her in the least. One side is real fat, the other just enough that I can tell it mostly by feeling. She eats anything, sweet or sour and is not the least bit sick. However, for three days before she swelled up she had quite a bit of a fever. It has all been most peculiar for Ronnie had one, too. For almost one whole day his temp was 104.2 without any other symptom of illness. By evening it was nearly normal, the next day he was O.K. and has been fit as a fiddle ever since. I have expected him to be swollen up, break out with measles or something, but they have not even been cross as they sometimes are after such a flare-up. Guess there are just a lot of mysteries in the rearing of children.

We are not going to the country this week what with the measles and one thing and another but hope to go next week. It is so cold here today folks are out in overcoats again.

August 9, 1951

I wonder if you can imagine how glad we are to be able to be writing dates in August? It means September is getting closer but not a minute too soon. This is the sort of a time one is glad not to have to live through too often. The Colonial Course exams are on. Claylon leaves in a few minutes and will have 5 this afternoon. He has already had 3 and will have 3 more Monday and then it's over -- be it good or bad. So far he feels he has passed but the hardest come this afternoon so tonight we'll see how he feels. If writing State Boards or taking Bar Exams are any worse I don't see how either the contestants or their families survive. If we don't get away from here soon Claylon is going to have ulcers unless he already does for he complains constantly. If the profs are all really fair there is no real danger of any of our folks not passing but one never knows.

They have to study and be able to give details on such things as Belgian History and Congo History back to the Stone Age (or whatever was first); education history and the present set up in Congo (and it's really complicated); hygiene; laws and government of Congo, etc., etc. Belgian state officials going out for the first time have to take the same course and exams and even they consider it tough. We have been living in a sort of vacuum fo so long I don't remember what I wrote last. Anyway I think I told you about going to the Lewises every day for 10 days while Lil lived here as her youngsters were exposed to three day measles. She was 2 months pregnant and hadn't had them so Dr.~Watson said she must take no chances.

It was during that time that Linda had mumps. I had just been home 2 or 3 days when Ronnie's fat little old jaws got still fatter. He is finished now, too, but it sort of kept us inside during the few days that were pretty enough for us to be out. Now it's cold and rainy again. We had about 2 weeks of fairly good summer but acts like fall is with us now. I wouldn't live here by choice if they gave us the place for that and other reasons.

The lady I had for a baby sitter went to Holland for 3 weeks at least 5 weeks ago and I haven't heard of her since so about all I do is act as baby sitter. They are both naturally bored with everything they have inside to play with and so long as they had mumps there was no place to take them without exposing others and now it rains. Oh, me. How do millions of mothers do it in the city? I just know one thing -- they don't all live with someone else where nothing is private, not even your own thoughts!

September 14, 1951

(Aboard the ``Elizabethville'' on the way to Matadi)

Today is wonderful -- smooth, warm and sunny and everybody is happy after having been seasick most of the time up till now. Sunday we are to dock at Tenerife (Canary Islands) for nearly half day and there are excursions planned for us. This is tropical so from then on we'll get rid of all our warm clothes.

I think our accommodations are better than on the Queen Mary probably because there is only one class so we are all treated alike. We have a big cabin - 3 beds and a folding baby bed for Ronnie with our own private bath equipped with shower.

Our meals have been good with lots of everything and my children eat as if they had been starved for years. They could eat cookies and other goodies continuously and still put away full sized meals. Today I had a special birthday cake with the compliments of the Commanding Officer and his crew. There is a coffee bar which serves ice cream sodas and sundies. There is a hair dresser with whom I have an appointment for tomorrow as part of my birthday from my hubby.

The nursery is on the open deck as well as a room inside so it is very nice for the children except that as usual ours don't like to stay alone. Since lots of the parents stay there and deck chairs are available it is easy to watch them and just being in sight is all they ask.

Wema, October 20, 1951

I started school with Bill and Norma (Horner) this week so from 8:30-12 we do that and Marjorie has been keeping my infants. Everything is going smoothly and I'm sure all Bill needs is lots of patience, praise and confidence that he can do O.K. I was really surprised at how well he does and he is so much better physically than before. Norma is a little whiz bang at her work but spoiled enough that she sometimes likes to get off too easy. Anyway they are well brought up children and very pleasant to work with.

We are going to build a yard fence and then I hope to keep ours corralled at least part of the morning so Marjorie will be free. I have a woman, Malia, who follows them around and keeps them out of trouble. It seems like it will work out nicely. Norma and Linda haven't hit it off too well yet as Norma thinks Linda is too little. Bill plays better with them but I notice the girls are doing better each day. It is wonderful for all of us to have some freedom. Horners think the kids are wonderful and the smartest things they every saw!

Marjorie says Ronnie is just like you, Dad. His build, his walk and mostly his expression. He has a way of wrinkling his forehead and puckering his mouth when he says
no" or ``I do'' very determinedly that she thinks is like you would do it. They think Linda is more like Claylon. She took her first penicillin needle without wimpering and they were both agast. Both Bill and Norma watched and they were too and said, ``She's better than we are for we cry.'' Then the next day she took the next one just as calmly and I think Marjorie and Howard were both about in tears. Said they had never before seen a child take shots without crying and especially when she knew what was coming so think that she has some friends for life.

November 3, 1951

We are sort of getting settled into a routine now so that it is hard to imagine that we were gone so long and that have so recently returned. At the moment the Horners and we are alone on the station as Heimers left a few days after we arrived to get their things down from Mondombe. They should be in on the boat in two or three days.

Everybody thinks Ronnie is about the last word in cuteness, it seems. Of course, he is at the age when everything he does looks that way but he knows it too! He rolls those big brown eyes around and knows he'll get a laugh. They are both so much easier to manage here and seem so much happier all the time. Instead of them always being under foot and all of us wondering what on earth to do they are up and out as quickly as possible. I have a woman follow them around and she keeps them in sight of the house and helps to keep them interested in their play. However, she doesn't have to entertain them all the time like we used to in the house. She just sort of acts as sentry. Ronnie loves to follow the goats and I think he would follow them all the way into the forest without any fear whatsoever if she didn't stop him. They play together now much better than they used to.

They love bananas better than anything. I asked Eale yesterday how many they had eaten and by the middle of the afternoon it was 4 big ones apiece. Then they each had some for supper. Marjorie is afraid we'll go broke this term, and she never saw children eat like these do.

I was glad to hear from Doris all about Randy and his development. Especially how he gets too busy to get to the bathroom on time. I'm going to tell her I am more and more in agreement with those who say boys are more difficult than girls. I never had any trouble with Linda, but Ronnie -- we haven't even started to make any progress and it even makes him firey mad to have his pants changed. Just would rather wear them as is than be bothered.

December 31, 1951

Guess this is the last time I'll use that date. Just think it will be seven years tomorrow since I agreed to thinks sort of life! Hardly seems ike it could be that long and yet a tremendous amount of living has been done during those years.

Right now I am very much aware of it. Just out the window is our play yard where about seven or eight children are playing. For the most part the little ones do very well together but now and then I hear a yell indicating that at least two of them are wanting the same thing at the same time. The Cardwell children (the family has spent Christmas with us) have marvelled at our play yard and say it is just like a park. Mary Ann (4) told her daddy she wants a ``skidding'' board when she gets home. The swing, teeter-totter, slide and sand box furnish entertainment for them all.

We had a good Christmas. Linda was tickled with her dolly and doll buggy but some pets and pans and sand toys from Santa tickled her just as much. Even before she had hardly looked into the buggy Ronnie blew a whistle from Santa and she says, ``Don't blow that thing, Ronnie. My baby is asleep and I don't want her awake yet!'' At the breakfast table she was paying no attention to any of us and saying in a sort of singsong voice, ``I'm just as happy as I can be!'' So we decided Christmas and Happy Birthday had been a success.

\hypertarget{section-1}{%
\chapter{1952}\label{section-1}}

\textbf{February 11, 1952}

Guess I'd just as well start off telling you some more cute things your grandchildren do. They are the ``book'' craziest children I ever saw I think. Don't know what that is a sign of, if anything, but it almost gets to be a nuisance sometimes. If I so much as sit down on a chair, no matter what the time of day, and one of them sees me in they come for a story. We have read nursery rhymes until I could say them in my sleep and by now even Ronnie can recite fairly long portions of several of them. Linda does them with no coaching, scores of them. But the thing that really made his daddy swell up with pride was the other day when Ronnie exhibited his counting ability. Claylon had been helping Linda count to 5, then immediately Ronnie spoke up 6, 7. We thought it a coincidence and tried him out several times but always he had the right answer. Then the next morning Claylon said 1, 2 and Ronnie follows 3, Claylon 4, Ronnie 5, and so on up to 7 so naturally his daddy is convinced he is just smart. Some of the time lately I have been letting them come upstairs to school if they will be fairly quiet. Ronnie doesn't stay long but Linda would keep at her ``studies'' all morning if I would let her. One day I wanted her to go down and play and she says no, I just have to study my spelling.

Within a few months I plan to order the Calvert kindergarten course so that if she is really interested next year she can do some work. Think it may be easier to manage her that way than to insist she stay down and play outside.

Guess yon had as well know all our troubles at once. We are expecting another little Weeks about September 1! That isn't trouble except that since I haven't quite reached the third month yet I am sick as a dog most of the time so am no help either in the work or the morale of things around here. It is
encouraging to know that I am no ways nearly as bad off as when either of the first two were on the way. About like with Ronnie, I guess. I have been able to keep at school each morning but when our hot afternoons hit then I'm done for the day. Another 2 to 4 weeks should find me feeling fine, I hope, then maybe everything will look brighter. We're happy about the baby, of course, for Ronnie will be 2 1/2 by then and then maybe we won't be traveling with a tiny baby again.

\textbf{April 12, 1952}

It is so hot and dry that one can hardly get through the afternoons and by night when the children are in bed are so exhausted from putting up with the heat and it's usually still hot, that all one can do is go to bed. The heat is bothering blacks and whites alike and causing more sickness than the doctors have most ever seen. There has been an epidemic of dysentery that really lays them out. It is tropical disease but up until last year they say we had never been bothered around here. Last year while the Horners were gone and the nurses ran out of medicine several patients died. There have been no deaths this year but at the last count there had been about 75 cases treated at the hospital. There were 25 from the boys' house in one day. It's from the low water, for the folks drink fron the streams and when the streams are badly contaminated and the water isn't high enough to keep moving on, they get sick. Claylon has been boiling the water for the school boys lately and that has helped.

Then there have been a lot of cases of polio. There have been three of the nurses' children at the hospital had it just recently and several others from near here---even an adult or two. We are keeping all the children away from crowds as much as we can.

I told you earlier about the measles. Marjorie gave all the children shots and though ours both had the measles, neither was very sick so guess they really helped. They weren't even very cross but were broken out like genuine measles.

We acquired a new member of the household this A.M. A Belgian couple are leaving and wanted a good home for their dog. When they had visited with us we noticed that Ronnie was making up to the dog especially well. (He has always been a little afraid of dogs.) I had always been opposed to such young children having dogs but what could I say when he was loving it and when I was asked for my opinion in front of all the visitors and family. So-- it is a lovely one, about three years old, out of its baby ways and well house broken, though is definitely a house dog. Since it understands only French, both Ronnie and Linda are beginning to talk to it in French. Ronnie suggested this afternoon that it take its nap in bed with him but we quickly vetoed that. Since it slept under one of their beds, don't know how long it will be till we find them all curled up together.

Linda has been picking up a lot of her elders' expressions. She rattles off ``for goodness sakes'' and ``my gracious'' just like her mother. Then she has Marjorie's ``it's simply wonderful'' down pat. She says, ``My gracious, can't you see that I can't do that right now. I am simply too busy.'' But Ronnie is the one with a mind of his own. we have been trying, though not so very hard, to impress correct bathroom procedures on him. He goes willingly and alone IF it is all his idea but just let someone else suggest it! One day while Marjorie was here he grabbed himself so I rushed him to the bathroom and Very Much against his will he did everything right. Then he threw himself and had a regular tantrum. ``I didn't want to go on the potty!'' He must have yelled for 5 min. Marjorie repeated that to Howard who, of course, thought it about the funniest thing he ever heard of (since it wasn't his child, I expect.) They used to think Billy was abnormal with lots of tantrums but it looks perfectly OK to them in Ronnie. Another one he has, Dad, is sometimes at the table when he doesn't want to say ``Thank you''. He's not like you in not being thankful because he doesn't want the food, he just wants it right now and not after time for a blessing. Sometimes halfway through the meal he is ready for time out.

\textbf{May 13, 1952}

My spirits as well as my physical condition seem to have revived with the coming of cooler weather. Hope its a lasting condition. My school will be out in less than a month and I am hoping to get a lot of housework done as well as other things. I have an assignment to write another Lonkundo text, I need to take lots of pictures, l want to work in the yard, paint some furniture, write a form letter, and would like to take three or four days off to go with Claylon into the back country for a look into schools there and a bit of a vacation. Then we should start school by the middle of July since we'll have to take a few days off during the fall.

The children are as cute and noisy as ever. Linda knows we are going to have a babv one of these days and is convinced that it is to be a brother. We were discussing names today and I was saying that if it is a girl Hilda wants her called Susan and Linda pipes up ``And if it's a boy it will be called Tommy.'' She says it's all right for Ronnie to have a baby sister but she wants a baby brother.

I am being besieged with questions now. Linda picks up a book -- ``Now, Mommie, who makes that book? Who makes this picture?'' Tonight at supper, ``Where did we get this pineapple?'' ``Well, it grew in the dirt and sand.'' ``What is sand?'' ``What makes flowers grow?'' I say, ``God'' knowing I'm letting myself in for something. ``Where is God? How does He make the flowers grow?'' Then I wish she had a grandmother or two here to help me out. The books I have read don't seem to give the answers, they just say all questions should have sensible answers. Oh, me.

Ronnie has taken up Randy's old trick of calling his daddy by his name. In the morning it's ``Where's Claylon? Claylon, breakfast is ready. CLaylon, read me a story.'' As Claylon says, if everyone could say his name as plainly as Ronnie surely we wouldn't get so many letters addressed incorrectly. I heard Linda tell him one day, ``That's Daddy.'' But Ronnie get her with ``But his name is Claylon.'' Linda's favorite expression this week is, ``Oh Mother, I'm sooooooooooooo--excited!''

\textbf{May 27, 1952}

I thought I'd get this done before the noisiest half of the family got up fron his nap but no such luck. Linda has been quietly sitting on the floor drawing but now she tells me, ``My little Ronnie boy is getting up!'' Sometimes she is very motherly and condescending to him and takes great delight in pleasing hin but the spells are usually short lived. She says now that he is such a happy little boy!

They are messily having iced drinks in ``breakable'' glasses. A few days ago they grew up and any more are highly insulted to be served in their tin plates and plastic glasses. As Linda says, I'm growing bigger
every day, and now I'm big enough for a ``breakable'' plate. Of course, whatever she does is good enough for little brother.

\textbf{June 3, 1952}

Ronnie and Hal, Jr.~play outside all morning long until they are both so tired they can hardly eat lunch. Ronnie is going to miss Hal, Jr.~for they leave July 3 and they are real good playmates. Hal, Jr.~is just 4 months younger than Ronnie though a lot smaller. He holds his own so sometimes we hear some real squalls but not too often. Linda is more prone to play by herself.

She is quite interested with the new arrival. When asked what she wants, she says, ``A baby brother.'' Well, what if it's a sister? ``Oh, Ronnie may have a sister but I want a brother!'' Of course, their daddy thinks they are real cute and especially when Linda begins acting subtle to get her way. The other day I was rushing around from one job to another when she says, ``Mommy, aren't you awfully tired?'' I admitted I was getting that way. Her comeback was, ``I just thought so. Now you come right over here and sit down
on this nice soft couch and read me a story!'' So what could I do?

\textbf{June 30, 1952}

My `vacation' is nearly over---2 more weeks to go, and I am going to have to keep moving or I won't get everything done that I had planned. I haven't done too badly though for took out some days whan I just was plain lazy and didn't do a thing. I am so anemic that I take iron pills all the time and still am on the low borderline. Then my blood pressure is so low that Marjorie says its a wonder I cam move at all. Most of the time, though, I feel real good and when I do have sometimes overdone the work, such as housecleaning, then have to take it easy for awhile.

Horners had a doctors' convention here over last week end and that meant entertaining for us, too. Since the Keane Watsons stayed here we were delighted to have them. Then Bakers from Mondombe were here, too, for several meals. The others were Belgians. Anyway I had wanted to clean house so thought this was a good goal to set to get it done before they came. We just about made it but it took some doing. It was amazing the amount of dirt we got out. I guess it was sort of like spring housecleaning at home but one would think that with hired help all the time we could at least keep the house clean but they don't even see cobwebs and dust unless they have their noses stuck in them. After spending several days in the kitchen, Eale told me one day, ``I just don't like this.'' I asked him why not, if he didn't like to work in a clean kitchen or if he just didn't like to clean it. But he just laughed so I still don't know. Of course, compared to the way they live we must seem too particular for words. He is clean about his work and keeps the kitchen in good enough shape generally but his cupboards get as messy as they do for any of us and he just can't be bothered with cleaning them until we make an issue of it.

I am enclosing a picture that Linda cut out of her color book. I thought it was pretty good for a 3 1/2 year old without any help whatsoever. She gets terribly provoked when she doesn't do a good job of following the lines. She still sucks her thumb but I think that is about her worst habit. I wish she would quit but we have tried all sorts of talk and persuasion and nothing helps though she does often promise that when the new baby comes she won't do it any more so it won't learn how!

This is the last week of Claylon's school until the first of August. He is so glad to have it out and so am I. He can't keep up this pace just on and on. Now with Sat. classes there is no break whatsoever for him. He is up at 5:30 every morning. Sunday maybe it's 6, and stays right on the job until 6:30 at night. Then goes to church 3 nights a week, usually works int the office afterwards or has native palavers until lately he hasn't been going to bed until 10:30 or 11:00. I believe the pressure is worse than it has ever been and we can see no hope for its getting better. This next term of school he'll have that as usual and any evangelistic work and palavers that come up. Heimers leave this Thursday and Davises, who will be new to the work can't possibly get here until in January.

We are planning to make one short trip into the back country together and then he is going to make a long one by himself where he will be riding his bike a lot and won't have all the teachers and pupils, too, on his neck every day.

\bibliography{book.bib,packages.bib}

\end{document}
