% Options for packages loaded elsewhere
\PassOptionsToPackage{unicode}{hyperref}
\PassOptionsToPackage{hyphens}{url}
%
\documentclass[
]{book}
\usepackage{amsmath,amssymb}
\usepackage{lmodern}
\usepackage{iftex}
\ifPDFTeX
  \usepackage[T1]{fontenc}
  \usepackage[utf8]{inputenc}
  \usepackage{textcomp} % provide euro and other symbols
\else % if luatex or xetex
  \usepackage{unicode-math}
  \defaultfontfeatures{Scale=MatchLowercase}
  \defaultfontfeatures[\rmfamily]{Ligatures=TeX,Scale=1}
\fi
% Use upquote if available, for straight quotes in verbatim environments
\IfFileExists{upquote.sty}{\usepackage{upquote}}{}
\IfFileExists{microtype.sty}{% use microtype if available
  \usepackage[]{microtype}
  \UseMicrotypeSet[protrusion]{basicmath} % disable protrusion for tt fonts
}{}
\makeatletter
\@ifundefined{KOMAClassName}{% if non-KOMA class
  \IfFileExists{parskip.sty}{%
    \usepackage{parskip}
  }{% else
    \setlength{\parindent}{0pt}
    \setlength{\parskip}{6pt plus 2pt minus 1pt}}
}{% if KOMA class
  \KOMAoptions{parskip=half}}
\makeatother
\usepackage{xcolor}
\IfFileExists{xurl.sty}{\usepackage{xurl}}{} % add URL line breaks if available
\IfFileExists{bookmark.sty}{\usepackage{bookmark}}{\usepackage{hyperref}}
\hypersetup{
  pdftitle={Peeks at the Weeks},
  pdfauthor={Helen Weeks},
  hidelinks,
  pdfcreator={LaTeX via pandoc}}
\urlstyle{same} % disable monospaced font for URLs
\usepackage{longtable,booktabs,array}
\usepackage{calc} % for calculating minipage widths
% Correct order of tables after \paragraph or \subparagraph
\usepackage{etoolbox}
\makeatletter
\patchcmd\longtable{\par}{\if@noskipsec\mbox{}\fi\par}{}{}
\makeatother
% Allow footnotes in longtable head/foot
\IfFileExists{footnotehyper.sty}{\usepackage{footnotehyper}}{\usepackage{footnote}}
\makesavenoteenv{longtable}
\usepackage{graphicx}
\makeatletter
\def\maxwidth{\ifdim\Gin@nat@width>\linewidth\linewidth\else\Gin@nat@width\fi}
\def\maxheight{\ifdim\Gin@nat@height>\textheight\textheight\else\Gin@nat@height\fi}
\makeatother
% Scale images if necessary, so that they will not overflow the page
% margins by default, and it is still possible to overwrite the defaults
% using explicit options in \includegraphics[width, height, ...]{}
\setkeys{Gin}{width=\maxwidth,height=\maxheight,keepaspectratio}
% Set default figure placement to htbp
\makeatletter
\def\fps@figure{htbp}
\makeatother
\setlength{\emergencystretch}{3em} % prevent overfull lines
\providecommand{\tightlist}{%
  \setlength{\itemsep}{0pt}\setlength{\parskip}{0pt}}
\setcounter{secnumdepth}{5}
\usepackage{booktabs}
\usepackage{amsthm}
\makeatletter
\def\thm@space@setup{%
  \thm@preskip=8pt plus 2pt minus 4pt
  \thm@postskip=\thm@preskip
}
\makeatother
\ifLuaTeX
  \usepackage{selnolig}  % disable illegal ligatures
\fi
\usepackage[]{natbib}
\bibliographystyle{apalike}

\title{Peeks at the Weeks}
\author{Helen Weeks}
\date{}

\begin{document}
\maketitle

{
\setcounter{tocdepth}{1}
\tableofcontents
}
\hypertarget{a-weeks-family-book}{%
\chapter*{\texorpdfstring{\href{https://beggarsborn.com/peeks-at-the-weeks/}{A Weeks Family Book}}{A Weeks Family Book}}\label{a-weeks-family-book}}
\addcontentsline{toc}{chapter}{\href{https://beggarsborn.com/peeks-at-the-weeks/}{A Weeks Family Book}}

This is where I can put some text.

This is where I can put some more text.

\hypertarget{cover}{%
\subsubsection*{Cover}\label{cover}}
\addcontentsline{toc}{subsubsection}{Cover}

\hypertarget{introduction}{%
\chapter*{Introduction}\label{introduction}}
\addcontentsline{toc}{chapter}{Introduction}

\emph{Christmas, 1988}

Every now and then someone mentions that I should write a book based on our old letters. But this is a project that I have always just pushed aside for some future date---when all the other projects are finished.

Then by chance Steven and Julie saw a few of these old letters in which some ``cute'' things some of you had said or done were mentioned. When Julie reported you all enjoyed them when you were together in Kinshasa, I decided to hunt up a few more that I knew were somewhere in the letters.

So one thing led to another and without intending to compile a book, a book of quotes from old letters seems to have emerged. The first is from a general letter which went to our complete mailing list. Other than that, all of these quotes are from letters to my parents unless a note indicates differently.

I have tried to include in this the important happenings in our life during those years as they pertained to the Weeks. You will note that very little is included about the work, the lives of the local people, etc. All this is another story. The question was, where to stop. The letters go on and on, if not to your grandparents, then to you once you were in the U.S. in college or else in the hostel. So mostly I've tried to get you through the hostel years and until you were more or less on your own.

You were all Such Cute Kids! And I did so love to write all about your doings and sayings and I tried to share your growing up with your grandparents. I hope you can now enjoy reliving the past and sharing it with your spouses and your children.

Love,

Mom

\hypertarget{section}{%
\chapter{1946}\label{section}}

\hypertarget{may}{%
\section{May}\label{may}}

\hypertarget{may-3-1946}{%
\subsection{May 3, 1946}\label{may-3-1946}}

\emph{DCCM, Coquilhatville, Congo Belge}

Dear folks,

In spite of that heading, we are still in Leo waiting for passage on to Coquilhatville. It looks as if we would be here until May 14 unless there is a cancellation on the plane before then. We have definitely decided on the plane trip as everyone here advises it for me. The boat takes about 5 days and they say that it is very crowded. The food is poor and we would have to carry our own drinking water. Since it doesn't leave until the 8th we would arrive about the same day as if we fly. It is just 2 A hours by plane.

Talk about hard-working missionaries! We have really been having a life of leisure. For instance, we have both spent the morning reading. There is nothing else to do and I guess we could be glad for the free time for it won't last much longer. Also we are told that it is just as well to take things easy when we first come. I was completely worn out for a few days but am real perky again.

This is such a restful, beautiful place that it is like a real vacation to be here. It reminds me of Florida more than anything else. Don't believe it is as hot here as Fla was in the summers. It is almost always very pleasant until noon, then about 1 o'clock when the sun is beating down the hardest it is very hot. However by 4 or 4:30 it is usually pleasant and the evenings and nights are fine. My bed is back in a hot corner of our room so I have been comfortable without cover most of the time but Claylon, who sleeps by the window, always uses some.

Right now I am sitting on the back porch. There is such a wind my paper will hardly lie flat and it is thundering. Probably by tonight we'll have a nice shower that will cool things off.

Have just taken time out to read the newspaper---French---so now you know we are well informed. Just saw that Hoover is in China.

Again about our setting---there are 2 orange trees here in the back yard and although this is not orange season there are some on the trees. Also there are some grapefruit trees, cocoa-nut palms as well as other palm trees. Right here by the porch is a poinsettia bush higher than our heads and just covered with the red blooms like the ones you get at Christmas. All around are bougainvillea, hibiscus, and many other tropical flowers, the names of which I still have to learn. Also I have seen cannas, marigolds, and beautiful big dahlias. Everything is very colorful and as we walk down the street panorama of color and beauty practically takes our breath away.

Claylon has taken a good many color pictures which we hope will be good. In fact he has almost finished the second roll of film since we have been here. The first one he sent off to NY about our second day here. We will enclose the list of pictures. However, can't be positive you will get them for some time for he made a mistake in addressing the film. Instead of putting the return as your address he put ours at Coq so on the next mail he sent an airmail letter to Eastman explaining what he had done and asking them to send the pictures to you. So if they understand you should get them eventually. It may take some time for them to get to NY although he sent them airmail. If they do come, we will be much interested to know just which ones are good and which are not. He thinks that some will not be any good.

The native boys here have interested us very much. The boys do all the work, in the kitchen, laundry, rooms, and elsewhere. They wait table, make our beds, clean the room and do everything so we don't have to lift a hand. We were pleasantly surprised at how well the laundry is done. At first I hesitated to let them do my dresses and slips but I needn't have worried. They

were beautifully done---better than I would have done and Claylon's shirts were perfect. By the way they use the charcoal iron so we know it will work. The boys are quite intelligent and interesting as can be. Of course they all speak their own native language and French as well. Then some of them are learning English. There are two very cute little fellows who speak it some and are tickled to pieces for us to talk to them. Claylon helps them with English words and they help him with French. Yesterday one boy was reading here on the porch and asked me for some help. Then he read the story of the three bears to me---much better than I could have done for him in French.

There is a native man, Luteti, who runs this house and he seems to be a brilliant man and very well educated. He lived with a missionary family as a boy and is a devout Christian. He speaks 6 or 7 languages and has travelled abroad in Europe. His ambition is to go to America. Wouldn't he be terribly disillusioned if he were to go where he would be treated as lower than dirt because he is black?

We have seen many interesting African curios, especially ivory, for sale. So far we have resisted the temptation to buy because they say we can get better things cheaper at Coq. Will be glad when we can send you some though I am not sure how easy that will be. There is quite a heavy export duty on the ivory.

We have become acquainted with the other families who are here right now trying to get home. Compared with them we have had a mighty easy time in getting out. This one family in particular we have been so sorry for! They have been here for over 8 years. Have a daughter 9, at home that they father has never seen and the mother has not seen since it was 7 months old. Have been trying to get home for 3 years. About 6 weeks ago they had a telegram that they should come to Leo and could get back to Belgium. Have been waiting for a Belgian boat ever since and had finally decided they would not get one before Christmas. I think that just today they have had assurance that they can fly. They have 2 small children here and the wife has been quite ill.~It makes one's heart ache to think of how severely the war has affected all these people. They are native Hollanders but their folks have gone to Belgium as there is nothing left for them in Holland. Many, many others from here are in the same boat. They say there are 20,000 Belgians here who want to go home now---haven't been there since before the war. There is one boat leaves here once a month carrying 500 people. Imagine the waiting ahead. All their boats were lost.

We are most anxious for news from home but, of course, don't expect any here. Do hope we will have some mail waiting when we get to Coq.

Claylon sent a letter today airmail and we have decided that we can send 3 of these tissue thin sheets for the minimum price of 16 francs or about 35 cents.

Guess I have said all I know for now so will leave some space for Claylon. He has a nice little tale for you. Am putting the picture list on the bottom of this sheet so you can cut it off or copy it on more firm paper.

Dear folk, when we first got here Helen was put downstairs and I was sent up because of crowded conditions. On Wednesday of this week the other lady moved out of her room and so I moved down. The first night it happened and Helen has been laughing at me ever since so have all the others. You see my jaw and neck began to swell from what I though to be an insect bite. But it is still swollen and everyone thinks I have the mumps. I have been exposed to mumps many times as a kid but never had them., so guess it is possible that I have them. Everyone has

fun at my expense over it. Personally, I think it was an insect of some kind and it will be all right by the time you are reading this.

Tell the Rauchs to laugh about my two little bananas if they want to but I have the laugh now. We have them here every day from 8 to 10 inches in length. I just ate two for breakfast.

Also oranges cost less than 1 cent each. As yet we have seen no apples and you know how well Helen likes them. We went to town the other day and ordered 3 more cases of Klim and a case of apple sauce. You see Mr.~Keating told us that the export of Klim from the states had been curtailed greatly so we found some here and decided to get extra. Also we looked for American coffee chocolate, rice etc., and couldn 't find any at all. Guess we will do without those things.

Excuse the torn spot, but I am sitting up in bed. We don 7 have the Double Dip here but we have our ice cream and pretty good too. We ate some about every day until I got laid up. It costs about 20 cents but we eat anyhow. Once we leave Leopoldville we won 7 get anymore unless we make it ourselves. As yet we haven 7 been able to find a refrigerator or a radio. We haven 7 bought bicycles yet-The only ones we saw for sale cost \$75.00 when they should cost about half. So we will wait and hope.

Suppose you think it silly to refer to all the pictures as probably not good. Yet it is true for they were taken under all kinds of conditions. Only wish I could see them so I could profit by past mistakes. Space is gone so bye for now. Our love to everyone and do write.

Helen \& Claylon

\begin{center}\rule{0.5\linewidth}{0.5pt}\end{center}

\hypertarget{may-9-1946}{%
\subsection{May 9, 1946}\label{may-9-1946}}

\emph{DCCM, Coquilhatville, Congo Belge}

Dear folks,

Again we are taking advantage of some people's going to the States to send you a letter.

It will get to you just as quickly as for us to send it on the Pam American and will be cheaper! (Already we are pinching our pennies, you see!) You will have enough with Belgian Congo stamps, I am sure, to satisfy you as time goes on.

We are still in Leopoldville and there isn't so much new to write about but just thought it would be a chance to tell you we're still liking it fine! However, we are very anxious to get on to Coq to see what is in store for us there for the next few months and to see again some of the people we know. Claylon was in Canada with the Lewis' family who are stationed in Coq. and the are very anxious to meet again.

I guess it's good for us not to have much to do right now but it does seem odd for supposedly hard-working missionaries to be spending most of their time reading, playing Rook (imagine) and just loafing. We really have been studying French, though. Claylon is my teacher.

It is just about hopeless to go to town unless you can speak some. Claylon does very well making himself understood in the stores. Really I am very proud of him!

Today a carload of us went for quite a drive along the Congo River. My how I wished you could have been along. The beauty of it all is simply indescribable. On one side of the road (a good wide dirt road part way and cement the rest) were magnificent homes of government officials and other big shots, I suppose. They are all built of bricks or stucco much like that used in Florida and are painted cream color or a light yellow. The bougainvillea and other vines sprawl all over the walls and the yards were a riot of color with many different flowers. Their homes have big, wide porches and outside hallway-like things that connect one part of the house with the other.

Then on the other side of the road the Congo. Big, beautiful Congo. It must be at least a mile wide here and with the sun on it was gorgeous. We saw the rapids where boats cannot pass because there are too many big rocks and the water is so swift. There it made waves and splashed about like the ocean but from the most part the river is perfectly smooth.

On our trip we bought two of the most delicious pineapples I have ever eaten. Really I'm ashamed to tell you about them for all I can do is to make your mouths water. We just sliced them up and ate them with our fingers and they were so soft and juicy that juice just ran in every direction. It would be an insult to the fruit to have used any sugar on it. Yum! Yum! But they were good. Is that enough said?

I dreamed one of my crazy dreams the other night---that we were all at home eating fried chicken but that you all ate so fast and so much that I didn't get my share! But I've got it on you when it comes to the bananas and pineapple.

By the way, how are those chicks doing, Dad? Suppose it won't be long until you can begin to eat up all the profits. Are you still raising too many of them?

Shopping here is just the same old story. Ask for something and they smile sweetly and say they don't have that yet. It must be the same all over the world. However, after long searching Claylon found a bicycle for himself for about \$55 or \$60. (2450 francs and there are 43.75 francs in \$1.00 so you figure it.) Since that is high priced we are going to wait a while to

get mine. There just are no refrigerators to be found so maybe whenever they start making them again back there instead of striking all the time we can have one.

It's bedtime (10:00 o'clock) so goodnight for now.

\begin{center}\rule{0.5\linewidth}{0.5pt}\end{center}

\hypertarget{friday-may-10-1946}{%
\subsection{Friday, May 10, 1946}\label{friday-may-10-1946}}

Since this will be sent from America and I don't have to worry about pages, will chatter on for a while longer

We think we have sort of a funny story to tell you if you promise not to worry about us. Of course, there is nothing to worry about for the worst is over and it wasn't very bad.

You know that insect bite that Claylon told you about---well, just one week later the same insect bit him on the other side. This time there has been no doubt as to what that insect was for very plainly he has the mumps! Even on this side the swelling has not been bad, not nearly so much as they usually are on a man but there is swelling enough to be sure of what is wrong. He is staying in bed as all the Drs and nurses here advise and so far has been a very good patient.

One thing, here we have lots of medical attention. Now the amusing thing to me is that we turned down the boat trip to Coq for my sake and on the very day we would have sailed, he came down with the mumps. We have been thanking our lucky stars ever since that we did not go and are again convinced that God works in mysterious ways. I have laughed at him, or course. Just to think of all those shots and vaccinations we took to come to Congo and of all the interesting tropical diseases he might have had but instead no sooner than he lands but that he comes down with plain old American mumps. Seriously, of course, I'm thankful it is no more serious than it is but it has been quite a joke. He is being careful, too, realizing the seriousness of it if he isn't.

The medics here say he should be able to fly by Tues. if he is careful until then. The swelling has gone down a lot today so he should look about normal by tomorrow.

Claylon says to tell Bud ``hello'' for him since he always seemed so interested in him. I have wondered what you hear from Dale---if he is still enjoying traveling all over Europe. How goes the farming---suppose the crops are about all out by now. Did it ever frost hard enough to ruin all the fruit trees? Probably when we get your letters all our questions will be answered.

The crowd here surely changes fast and often. There were 5 people came in today from their mission station and tomorrow AM there are 7 others flying to the US. Keeps us interested just seeing who is who (and wondering how soon we will look like some of them coming from the bush.)

We have finished one color film since we have been here and are sending it along to NY with the people going tomorrow so will send you the list of shots in this. You should get them before too long. We are very anxious to know which ones are good for everyone has warned us that it is most difficult to get good ones here. Some were taken on the trip and the rest here around the UMH (Union Mission House)

Love to all,

Helen \& Claylon

(Explanation about film at end of listing of pictures)

\begin{center}\rule{0.5\linewidth}{0.5pt}\end{center}

\hypertarget{may-17-1946}{%
\subsection{May 17, 1946}\label{may-17-1946}}

\emph{DCCM, Coquilhatville, Congo Belge}

Dear folks,

Just wonder how everyone is in and around Winamac. We have had no news as of today since we left New York several weeks ago. Since arriving at Coq last Monday we have been constantly looking for news from you. We still feel sure that a letter will come in the next day or two. They seem to come real fast by airmail now and so long as we stay in Bolenge we can get them by airmail all the way from New York. It really is nice that it comes so quickly. Hope ours goes as fast to you.

In the last letter Helen told you about my mumps. It was really funny or everyone seemed to think so. They thought it quite a joke to come to Africa to have the mumps. But it was really a very mild case and I was up in three or four days ready to fly to Coq when the time came.

Suppose I ought to tell you that while in Leopoldville we lived like the king and queen.

Our beds were made, our floors swept, and every want taken care of. While I had the mumps I stayed in my room since there were so many kids around. They brought me food and water galore. Sometimes we would sit on the porch and read. For three francs or about 7 cents we could buy a quart of lemonade (cold). Or favorite boy Joseph brought the drink to us, filled our glasses, cleaned up all the mess we made and made us feel very lazy indeed. We thought to ourselves, ``Oh if the folks could see us now they would not feel sorry for us at all. '\,' We really fell in love almost with two of the boys (native) who were very bright and who spoke a little English as well as their own native language and French. They had a nice personality and such pleasant smiles that we had to like them. One was Joseph, a 16-year-old and the other Simon who is 19. We have their pictures in black and white and at least one of them in color. Certainly hope the color ones are good. We have the black and whites and they are rather good. Will send them to you before too long. As for us you can understand why we felt so spoiled.

But it hasn 't changed at all since we reached Bolenge as I will show later. First I'll describe our trip from Leopoldville and our arrival in Coq. We flew as you know but oh what a plane. It was a small one built for only a dozen passengers and was a German Junker. I don't know why the Belgians would buy German planes but they did. It didn't ride nearly so good as the big Pan American but was not bad at all. We flew up the river for a long ways and then cut over land and after 3 hours offlying landed in coq. Our flight was moved up a day right at the last so we almost didn't get a telegram sent to Coq. As it was the telegram reached Coq 10 minutes before we did, so the folk came driving to the airport just as the plane landed. I knew the Lewis ' and we were very glad to see them again. We got our bags and were soon at the mission in Coq. There we found quite a few papers and magazines of mine. Also the box of books which we sent from Logan back in February had been here about a month. There were several letters from people upriver also some from Keating. Our first shipment of goods from Nashville which Keating send in December have been waiting at the seaport for 2 months. We filled out some papers today and should have the things in a few weeks. Also our last shipment should be here in another month or two. You notice I don't keep on the subject but just write what I think of next.

We had lunch in Coq with the Lewis ' and Rowes ' and visited part of the afternoon. We met some of the mission boys and had a hand shaking good time for a few minutes. We sure were sorry not to know Lonkundo and to talk with them.

We had a pleasant afternoon in town then came out to Bolenge and were very impressively received. Will tell you of this later for have to go out to dinner now.

\begin{center}\rule{0.5\linewidth}{0.5pt}\end{center}

\hypertarget{may-21-1946}{%
\subsection{May 21, 1946}\label{may-21-1946}}

Several days have passed since we wrote the first two pages, but we missed getting the letter on the plane last week so just kept it until now. It should leave Africa on Saturday of this week. Hope it does and you get it first of June.

To continue the story of our arrive---we came to Bolenge last Monday afternoon. As we drove into the campus all the native and whites came out to meet us. There were about 350 natives all singing a hymn of welcome. We were so happy, but almost overcome to see the joy on their faces. And we didn't even know how to say any sort of greeting to them. The singing over they went their ways and we chatted with the other missionaries. They had arranged a special place for us. A new apartment was all new and fresh painted just for us. Dishes, cooking utensils, and linens all supplied. However, we were invited out every meal for the first whole week.

Our welcome was not yet finished. As we sat in the house resting just before bedtime a knock came at the door. Native leaders were coming to greet us. A group of boys from Wema came to say hello. Also a note inviting us to visit with the entire Wema group the next day. You see Bolenge is the graduate school and the students who had finished their work at Wema are here for more advanced work. They invited us down and of course we accepted. One of the missionaries acted as interpreter. We met them all then sat down to visit. They asked questions about our trip over, about what we did and most everything. The airplane was very interesting to them. They think a teacher is an angel from heaven so Helen rated with them. Also a woman about to have a baby is highly revered. The natives desire only to do her bidding. So you see how much they think of Helen already.

They have called her Mama from the beginning. They have already decided our names (even before we came) and have named the baby, which is more than we have done. But these were a group of Wema people, with whom we will be working later. We felt it an honor to be graciously received by them, and when they brought their gift of welcome---a bunch of bananas, two eggs, a cup of lima beans, some oranges, tomatoes and other fruits we didn 7 know what to say. Our eyes were wet and with reason. The gift represented the best they had to offer and they gave it gladly. We tried to thank them as best we could. Their welcome was so sincere. They had prayed constantly for our well being and safe journey and their prayers were answered. I only wish American people could have seen it, or could see any of the work for that matter.

What a difference it would make in their own religious living.

I must go on---The entire mission gave us a reception. Had a grand time. Have been going to town, and do about as we please so far. Will start studying language tomorrow. Went to see Doctor today. He said Helen was doing fine. We are going to stay here where she can go to the hospital with the baby. It is a beautiful hospital and well kept. The doctor is very good-he studied in America at Mayo's and John Hopkins. I though this would be best and you folk would have no need to worry about her I will send pictures of the hospital later.

Must say that we aren 7 living in luxury but almost feel as if we are. We get the housework done, cooking and laundry for 115 francs or \$1.00 per week. A boy is making our garden for 35 cents a week. I feel like we are cheating them yet 1.00 is high wages. So guess I will just have to like it. Have lots more to tell you about going to church and hearing the

wonderful harmony in the native singing and so on. Will tell you of it next week. Love to all and bye for now - Helen \& Claylon

(Helen writing now)

Say we just had mail from the US which was sent airmail one week ago today, May 13. We have been gone from New York for one month. What's the matter---doesn't mail service work in Indiana anymore? One letter was from Claylon's living link and the other from Eastman Kodak Co.~Not a word from any families. Feel there must be some explanation but do write us. Eastman is sending the roll of film we addressed wrong (the first one) to us here so we will see it and then send it to you. Will be sometime but you'll have to develop patience just as everyone in Congo has to. Writing on the back of this tissue paper isn't a good idea and we have used our four sheets so will send you more of our story next week.

PS. Claylon planted a garden a week ago and we have onions, radishes, beans, com, and greens way up. The com was up good on the 3rd day and has been growing fast ever since. Also has tomato plants and sweet potatoes set out.

\begin{center}\rule{0.5\linewidth}{0.5pt}\end{center}

\hypertarget{june}{%
\section{June}\label{june}}

\hypertarget{june-4-1946}{%
\subsection{June 4, 1946}\label{june-4-1946}}

\emph{Bolenge}

Dear folks,

We feel better now that we have had the second airmail letter from you. It was not nearly so long in coming-if I remember it came on May 27 and you wrote it May 17. The first one took about 3 weeks but it had gone all around Robin Hood's bam to get here. Someone said to have you mark on the front of your letters via Pan-American NY and they would be sure to come direct. If they do we should get them quickly for air service is getting better all the time. There are now 2 planes every week from NY to Leo and always 1 and sometimes 2 from Leo to Coq.

When Claylon was in Leo he broke his glasses, of all things, but had managed to tape them up so he could see. When we arrived here we wrote to Kline to send him 3 new lenses--- one whole pair of glasses plus the extra lens. The letter left Leo on the May 19 plane, and we had the glasses back here airmail just 12 days later---May 31. No one can hardly believe it possible for he always takes several days to fill a prescription, you know.

Did we write that Gray Russell has arrived? His was some more speedy travel. In fact he has set a record that won't be beaten for some time. He left NY on Wednesday PM about 8:00 and was in Leo Saturday AM at 7:00. In less than an hour he was on a plane for Coq and was there by 10:00 AM. His telegram from Yocum saying that he was on the way just beat him to Coq by about an hour. He came on out to Bolenge that afternoon and was everyone surprised! From NY to his own mission station in 2 Vi days! It was a week ago Sat that he arrived. Since then the Homers and two Bateman sisters have come so the missionaries are happy that folks are now getting back.

I am a widow again for a week. That seems to happen to me quite often. Claylon has gone on an itineration trip into the back country with Mrs.~Snipes so he can find out what goes on.. They left yesterday morning (Mon) and had the station truck take them about 60 miles from here. Then they were to leave the road and go by bicycle for about 2 hours to the village.

Sunday afternoon the truck will again meet them at the road to bring them home. Of course, Claylon knows so little of the language that he won't be able to know what goes on except as Esther translates for him but he will get some idea of what a native village is like and of what an evangelist does when he goes out. You see we have learned that the Boyers will be gone when we get to Wema so the whole evangelistic and educational program will be left in our laps. The Homers say they know nothing about that phase of the work.

I am not alone for Georgia Bateman, the Bolenge nurse, is staying with me at night. She came over and offered so Claylon could go and we thought it very nice of her. Yesterday I was invited out for both dinner and supper and am invited for noon today, so you see I'll not suffer from lonesomeness. Anyway I have lots to do.

There is so much to tell that I hardly know where to begin. We should have written last week but didn't---decided airmail was so high we would have to skip now and then. Still you won't have to wait as long as if we wrote regular mail. Guess I'll begin with the Horners being here. They arrived on Tues and were here until yesterday when they with several others went to their stations via the Oregon. They are certainly nice people. Both young---in their thirties. Had the two children, Bill 4 and Norma 11 mo., with them.. They were anxious for us to get to Wema

but after we talked all the pros and cons of our going now or waiting, they decided we should stay here for several reasons---mainly housing conditions there. They were not in favor of our moving into an unscreened house and at present that's what it would be. Also since none of our freight has arrived nor theirs either it would be hard for us all to get settled. So since July 1 is so close they figured we had better stay put. Although Marjorie said she was almost selfish enough to say for us to come anyway for Boyers are leaving Aug 10 and we will arrive there about Sept 10, when the Oregon makes its next trip, so Homers will be entirely alone for about a month.

The housing doesn't sound too bad for us. We felt very encouraged after their coming. They are moving back into a tiny three-room house where they had lived for a few months before their furlough which is screened. That leaves for us the house which we thought they would have. It is a so-called new house which has never been finished even though started 10 years ago. Now all it lacks is a little inside work and screening which Howard H plans to have finished for us when we get there. He took up enough copper screen from here to do the whole house. The house is brick with cement floors. It has a big living and dining room combined, office, storeroom, kitchen, two big bedrooms and bathroom. He thinks there are bathroom fixtures there but the plumbing has not been fixed up. However, it can easily be and until then the houseboy carries water to the bathroom for us. There is a \$1000 building fund for this house left from which we can draw to have plumbing, painting, and any necessary repair work done. Sounds better than a mud hut, doesn't it? We'll be able to tell you more when we have actually seen it but I feel quite happy about the prospects. Horners are going to build a house as soon as they can. Have plans all ready.

Howard says Wema has the best equipped carpenter shop on the mission but they don't have many good carpenters. Claylon will probably get the chance to train some. In order to have a beginning of furniture we are having some made here but that's hard too as carpenters are few and busy. We are having a wicker breakfast table and chairs made and two little wicker serving tables. The entire set will cost less that \$15.00 and they are very sturdy and pretty. We can paint them to suit our tastes. Also are having a bassinet and stand made for about \$3 which looks much like the ones in Montgomery Ward. We used their measurements and showed the man the pictures and he does the job. I intend to line it and fix it up with mosquito netting for the fancy drapery work when he brings it this week. Also we were able to buy the dining room furniture of a retired missionary for \$34.50. Have table (similar to ours in kitchen with drop leaves but no extension bds.) 9 chairs, a buffet, food chest, and serving table. It is not so pretty but is good and solid so will be a beginning. Has been ivory but needs a new paint job. Also we are having a chest of drawers made but have not seen it yet so will await the results. Anyway, you see we are getting a start.

Now while we are here, we are supposed to concentrate on the language. Gray Russell is teaching us along with a native so are in good hands. We are picking up a few phrases but it is a slow process. I am still surprised when I tell my boy ``sola bito'' and he washes the clothes or ``yela basi'' and he brings in water. They just laugh at us when we make mistakes and correct our pronunciation. Am afraid they are more patient than we would be. Their motions help, too. Sunday as I was walking home from church a man, our sentry, stopped me and I finally made out that he was asking if we were going to Wema on the Oregon this trip. I said ``nyonyo'' and he talked on with motions about the hospital at Coq. When he rubbed his stomach and pointed to Coq I knew he was asking if I was going to go there to have the baby! (No embarrassment here!) When I told him yes he said ``boloci mongo''---very good. They were all afraid I would start on the trip and something would happen. One can count on the natives being concerned.

Marjorie H. said I did not have enough diapers and booties and little kimonos so I have doing what she did. Bought lots of flannel and am making them. I borrowed a hand sewing machine and right now should be sewing the things up. Have made 1 coatie but need 5 more, also 6 pr booties. They are cut out and will not be hard to do on the machine. Then if I have time am going to make another gown and bed jacket. Clay Ion got some seersucker in Coq that will be the very thing. I am still feeling fine and have every idea 1 will be going strong till July, but that's only a month. Gee! I am so big and so tired of wearing my 3 or 4 dresses that will sure be glad when 1 can return to others. Marjorie and others have remarked on how well 1 look and how well I seem to stand the heat. Am glad they think so but some afternoons I get pretty hot.

This is the last page so must conserve every bit. Hope reading these things do not put your eyes out. Know it must not be easy. You needn't worry about your letters not being interesting for we enjoy every word of them, Claylon just as much as I.

One thing more the Horners told us I want to repeat. It might be a suggestion if you would be interested this winter but I realize it would be a lot of work as well as expense. He said their most appreciated gift from home was a barrel of canned meat. His folks had canned it in mason jars and packed it in a barrel just like yours, Dad, in sawdust and they got it all in good condition. I think they would just ship it to Keating and have him take care of importing it etc. but am not positive about that. Forgot to ask. Not only was the meat good---Marjorie said only 3 cans were spoiled and they were fit to feed to the dog---but they could use the cans too. If you used ball lids and sent extra rubbers, the cans would be very usable. Marjorie has a huge pressure cooker and does some canning when things are right in season and going to waste. Also we need to keep all our flour, sugar, soda, and everything of that kind in sealed cans. So far we have none but have been able to borrow a few. What I had in mind about this was that if the church should want to do anything for Xmas or any other time this winter maybe they could have a canning bee or something. I wouldn't want you to undertake such a thing by yourselves. It's both too expensive and too much work. Anyway, was just an idea. We'll probably have lots of them about things for you to send.

Claylon has already figured that candy would keep till it arrived. He brought a chocolate bar from NY that is still good and if sent in a tin box would be OK. However, he was happy the other day when I baked a cake and put on fudge icing and had some left. It was real good so he warned me that that he didn't intend to want for candy. One trouble is we have no chocolate or cocoa and can buy none so I used Thompson's malted milk as a substitute. Wasn't bad, either. For a potluck supper the other day I baked two big cakes and used over a pound of sugar. Sure was fun not to have to count the cupful's.

Since I won't send this for two or three days will leave a little room for any future news. Bye for now.

\begin{center}\rule{0.5\linewidth}{0.5pt}\end{center}

\hypertarget{thursday-june-6-1946}{%
\subsection{Thursday, June 6, 1946}\label{thursday-june-6-1946}}

Today has been wonderful---two letters from you and one from Doris. They are rather old---hers dated Apr 28 and one of yours Apr 30. The other May 8---when Dad and Hilda both wrote. Why they take longer I am not sure unless it makes a difference about the postage. However, they may just get on the wrong plane. These had all come up from S. Africa so were not at all direct. However they were very welcome anyway. Had a letter from Grandma yesterday. Came in about a week.

We are glad to know about the pictures. Sorry so many are not good but what can you expect from beginners? Maybe with time and lots of practice we'll improve. Claylon has the

camera this week so should get some natives and native villages. We hope to have someone here who takes pictures help us for guess it is quite a technique in getting them here. Hope you will not be disappointed with the projector. Know they are good if the pictures that go in them are.

Right now I am taking time out from Lonkundo study. Am supposed to spend 3 or 4 hours a day on it and we surely need to. Have done a lot of my sewing. The booties are finished and the coaties and blankets will underway. I borrowed some pinking sheers and will just cut out the diapers so that won't take long. Everything that is ready has been washed and right now the boy is ironing them. I hope to be ready for any emergency! Dreamed the other night that the baby came and weighed 20 pounds. Guess I had been feeling unusually big that day. Don't think I'll work any harder when the time comes than I did that night. You see what's happening to my space and my writing. All write again soon. Letters are the high light of the day and glad they come often and not all bunched together.

Love,

Helen

\begin{center}\rule{0.5\linewidth}{0.5pt}\end{center}

\hypertarget{oct-26-1946}{%
\subsection{Oct 26, 1946}\label{oct-26-1946}}

(Christmas card)

Dear Mom \& Dad,

Christmas at Wema promises to be a busy and a joyous time. We will celebrate much as you do, with special services, plays, and singing of Christmas hymns. The boys will have a special feast for they know the ``bendele'' will give them chickens and rabbits.

Our itineration will be finished by then and a new term of school will be starting. Between Christmas and New Year's Day all the village teachers, about 50 of them, will be in for supplies and re-assignments to their schools.

All we will lack will be the possibility of snow and being with home folks. We are already saving evergreens for decorations; we will imagine our millions of tiny white butterflies are snow and in our thought and prayers we will be with you.

May God bless you throughout the coming year.

Helen \& Claylon

(inside)

You know us, we never thought of Christmas greetings when we were leaving way back in April so are making our own. We certainly can't get all the way around but are managing 25 or 30 of these. (Better than many years at home.) Hope they bring some letters in answer. Also they will probably get around nearer Easter than Christmas but that can't be helped either. We would be interested to know when they arrive. I am sending you an airmail letter on this same boat so you can tell the difference in the time they take.

The boat which was due last Fri from Coq has not yet arrived and this is Tues so we still patiently wait for mail and whatever it might bring us.

I had thought of sending one of these to Dale and Doris but am afraid I might be a bit previous so will wait and send them an airmail letter when I know they are Mr.~\& Mrs.~So if they're there, tell them we meant Merry Christmas to them, too. Sorry we can't send you some of our good things but guess we can't do that! Have a good Christmas and don't miss us for really we are there with you.

Love to all,

Helen

(front)

Bosalo wa Eotswelo ea Yesu Masiya (Joy of Christmas)

Boloci el'inyo nda Mpel'Eoya (Goodness to you in the Coming Year)

(Claylon is sending his part of this in a separate envelope. We hear postal clerks are still taking stamps so don't want them to get away with too much at one time.)

\begin{center}\rule{0.5\linewidth}{0.5pt}\end{center}

\hypertarget{dec-29-1946}{%
\subsection{Dec 29, 1946}\label{dec-29-1946}}

\emph{Wema, DCCM}

Dear folks,

I didn't realize it had been so long since we had written you until I looked at the date on the last letter, Nov.~10. However, we sent it on the last boat so did the best we could, but time just passes so fast that we can hardly keep up. Anyway, I guess it was your letter that was written the 10th and ours was in answer to it, sometime later.

Christmas is over and maybe you are in Florida by now. If so, you will probably be too busy to even read letters but will write anyway and if you are not home, Dale and Doris can read it and might even write an answer! We still are looking for mail from them. Our mail service seems to get worse instead of better. They sent word from Coq on this last boat that not one package or regular letter had come through from America on the last boat---whether there are too many strikes at home, or whether the boats just missed connections no one knows but at least everyone is in the same fix---not a missionary got a Christmas package or letter, except that was sent airmail. Still have something to look forward to maybe. We are really not expecting anything except the package you said you were sending, and my living link church wrote in Sept that they were then mailing us a package.

Christmas was quite an experience for us---the first Christmas dinner in our own home.

So far as food was concerned, bet you didn't beat us one bit. Instead of turkey we had rabbit but I'm telling you it was good. We had three big ones and they filled that great big Pyrex roaster that Mary Lange had given me. Tame rabbits are way ahead of wild---every bit as good as chicken. I even cooked enough in the pressure cooker to make a big pan of oyster dressing in it. We had cranberry sauce---have dehydrated cranberries that when cooked you wouldn't know from fresh and when ground make as good sauce as we ever had at home. Had jello salad and cake and ice cream for dessert. Oh, yes, mashed potatoes and gravy. The potatoes are also dehydrated but venture you would never know the difference. Also, we have a Christmas tree made from branches of evergreens from the station, and it is decorated with the same kind of colored balls and tinsel that you use---thanks to someone who left some of that sort of thing for station use. We had fresh flowers everywhere---one large bouquet of gardenias, another of red cannas, one of bougainvillea, and even two small forest orchids. We had nut cups for everyone, made from the paper nut cups, lace paper doilies, pipe cleaners, and Christmas cellophane tape. Also had place cards!

Oh, yes we had guests, all the station people plus Mrs.~Mombeek and Nellie, and Mr.~and Mrs.~Meert. They are the Belgian people who have a plantation just adjoining the station. Mrs.~Mombeek and children have just recently come, with Mrs.~Meert, from Belgium. Of course, they all speak French except Mr.~Meert speaks good English and Mrs.~speaks some so we get along. (However I am mad every time they come to think that the Society made me finish my degree instead of learning French!) Mr.~Mombeek and their 9-year-old boy were to come but the boy was sick so could not make it. Mr.~Mombeek was to have been Santa Claus, we have the suit and everything, even gifts for everyone but since he couldn't come Nellie, the 18yrold daughter substituted.

Billy Homer had a wonderful time. Claylon had made him a very nice desk in the shop so he got that, plus books, color books, and pencils but most of all he like some little plastic

airplanes and boats. He sailed the boat in the bathtub until Norma went in to help him by washing all my guest towels and then just jumping in herself to rinse them out. She is just beginning to walk and naturally is into everything. She is chubby and good-natured and as sweet as she can be. Billy is too and the way he takes care of his little sister is quite comical.

Perhaps that is enough about our Christmas, now we want to know all about yours. We couldn't help but think about you and your day a lot. In fact, at Marjorie's mother's suggestion we all took time out about 6:00 pm to quietly remember our families and friends at home. Noon in Washington DC would be 6:00 pm here. Claylon says that the only difference between Christmas here and at home is that there it is cold all the time and here you can warm up the next morning. Even with a wool blanket he was cold most of the night and made that remark the next morning that at least he could warm up out here. Sunday School time here now!

While we wait for dinner!-----

We had a big attendance at church and SS this AM. Had about 350 for SS and probably more for church. This is a very special occasion for the teachers from all the villages come in for a week's visit, meetings, and business about where they will teach next year, who from their villages will stay there in school, etc. It will also be a very busy week for all of us connected with the school.

Claylon and I are all excited again about taking pictures. We have been taking some the black and whites and developing them ourselves. Homers have the outfit that you use. Then last night we did the last step and printed the pictures ourselves so now we have just as nice pictures as you would get from any photographic shop with the exception that ours are small, just the size of your slides. We do not have any enlarger and are not sure we can fix one up, but Claylon is going to work on it as we have our batteries now so that he can fix up an electric light, which is necessary. However, what we think would be more satisfactory now would be to have more 120 film for my camera and take pictures on that to develop. We could print these and send home to all our friends. Since the whole process is very simple and not too expensive we are going to give you a list of the things we would like to have and would appreciate your sending what you can find. Everything is listed in M Ward but if we wait until they would send them, we wouldn't get any pictures this term. I don't know if you can get 120 film now or not but if so we could sure use it. Think it would be all right if well wrapped, maybe put in a tin can for all our film is still good that was not tropically wrapped. I am going to ask some other folks to send us film, too, if they would like some pictures.

We have sad news about some of our color film. The las three rolls that we sent home with Boyers never arrived. The post office at Coq. would not let them send the film with your address as the return as we had done all of the others. So Mr.~Boyer put our address as the return and wrote a letter to Eastman asking that the film be sent to you. On the last boat we had a letter from Eastman in regard to that saying that they had been looking for the film for three months and they had never arrived so were telling us that we might trace them from this end. We have no record of them and no way to trace. The PO at Coq has had a scandal lately about some of the employees taking stamps off airmail and reselling the stamps and discarding the mail. We are afraid that is what happened as it would have been sizable amounts on three different packages. They were pictures taken at Bolenge, on our trip upriver to Wema, of Wema station, and of the baby's grave at Bolenge. The Wema pictures we can replace as we now have color film again. Got 9 rolls in our Ward order the other day. Oh, yes, it came in Dec.~so was just exactly one year arriving.

Jan 7, 1947

Over a week has passed since I started this and every day I thought I would see it finished. You just can't imagine the headaches we have been having. Saturday we were to give tests to all the boys and men the teachers had brought in to see if they were good enough to consider for Bekwolo or for Workmen. The Bekwolo are the boys in the boys' house. Claylon figured he could take 22 more and could have 15 me as we have just that many houses and no more for new fellows. We went down at 8:00 Sat AM as he, I, and Bonjale, our head teacher, were to give their tests and decide. We were simply swamped. The schoolhouse was packed to overflowing, as well as others in the yard. Fortunately, Bonjale took charge and right away weeded out some but left about 100 to test for Bekwolo and 50 for Workmen. That afternoon we had all their slates to grade and pick out those who might stay. It still breaks our hearts to have to send so many home who want to go to school, but there is no other way. Our building is popping its sides now.

Then that didn't end my problem. About 50 others wanted to stay in to live with relatives here and come to the morning school. We already have two classes that meet out under the trees and 15 more in a one room building that is about the size of the old Mooresburg Church so we can't very well take on 50 more. Bonjale and I have been testing them and weeding them out every day since. This morning I thought I would be hard boiled and just not take anymore. Then some fellows wanted to stay so badly that I let them read. They read very well---way above many we now have. I asked if they could work problems and they humbly answered, ``we'll try''. They got them all right. Well, as you can guess they are in school. Others are going to get to try out tomorrow. As you see, we have one difference here. If they don't measure up after a trial in school, we can send them home. Also, that is a threat that hangs over their heads that if they don't work well and if they get into too much mischief, out they go and we will take someone else in their place. One of the worst punishments is a refusal to let someone come to school for some misdemeanor for a few days. Often the Bekwolo work in the morning when they are actually not able because if they don't work they can't come to school. Otherwise, they would stay home from work when sick and would all come to school anyway for they really hate to miss. Haven't seen many kids in America bothered with that disease.

Anyway, that has been a part of our problem since I started this. Having the teachers here is a mess in many ways. We had to pay all of the six months salary, take their church offerings, have to give them school supplies, talk over their troubles (they have many) until no one has had much peace. There were over 500 in SS Sunday when usually we have 200.

One nice surprise happened Sun AM. We had a whole packet of airmail letters. Two from you, including the Christmas greeting and letter of Dec.~5. Also had letters from Claylon's living link written the 17th so took just about 3 weeks. Not bad. Much of our airmail comes directly through now so speeds it up but I am not sure that ours will go out much quicker. We are going to try sending this and some by carrier to a place where it will be picked up in the airmail and hope it will go sooner that to wait for the boat to Coq.

We are so far behind on the news at home we don't know what to think about it. Sometimes it sounds as if things would loosen up and we have hopes of getting a few things we need rather soon. Then we hear that John L. is on the warpath again and it looks as if we won't ever get anything. Know that you must feel the same way. Just interrupted again by someone wanting to tell me his palaver---Claylon and Myrle didn't admit him into school and he feels sooooo badly! Right now Claylon has about 10 on the back porch taking test that say they didn't

know we gave them Sat. and they are SURE they should have a chance for they can just outshine everyone else!

This is very interesting work and we like it more all the time but it sure is no place for anyone who does not like work. Even though we do have folks to do our housework we are busy every minute. We get up at 5:30 and go right to work. For the past two weeks I have been getting home in time to eat breakfast at 10:00 AM. We haven't had a language lesson for over two weeks because neither we nor Myrle has time to manage. We are all hoping that in a few days thing will sort of settle down to normal and we can get back to work.

Dale, even so I am glad I have my job instead of yours in building a new church. At least when we get ready to build we will only have to try to please 5 or 6 white people instead of so many all of whom have different ideas. Wish you luck on the job and hope you can keep peace in the church family.

We had word the other day that the Foreign Division is putting on a Crusade for money for all the foreign work and has in its list quite a number of things for Congo. Among them, electric light plants and a truck for Wema. They hope to have the money to purchase the things by 1950 so can start looking around by then deciding what they want to send us. Maybe by our second or third term we will have them. We had thought from the way Sly and Yocum talked back before we left that we practically had electricity, and so did everyone else. Now we wonder. However, the lack of a truck bothers us more. The only hope for one sooner is if some of us from Wema are able to raise the money for that purpose. The Boyers are home now and we hope they will give everyone good sales talks.

Say, your chicken business sounds prosperous. I figured how much your income would be for the year from them and then deducted about \$5000 as overestimation and still had a figure bigger than I could almost comprehend. Claylon laughed at me for taking off so much. I said that I knew that was the way Dad would do it! Think you had better save your nickels and plan a trip to Congo. Shoot, you could come and never miss the price after a year or two of such prosperity. We are so happy that things are looking brighter and just hope that they will not crash until you get your houses built and everything fixed for a comfortable future. Maybe you should be careful about telling us everything, you know how the Mitchell kids are---if they know there is money, they want things! It seems like I ask for something on about every letter but not because I think you are hunting for way to spend money. Really we would be happier if you would charge the things to us but however you do them we will be happy! (Claylon sure made fun of that little speech but heartily agrees that we enjoy every kit and parcel from home!)

Jan 21, 1947

You are sure ahead of us---just read at the supper table your Christmas letter that was sent the 29th---the day I started this! I'm getting to be as bad as Hilda and never thought I would! Don't aim for it to happen often and when you hear Claylon's story you will understand something of why. The letter was swell---next to a visit home for Xmas. We could just see all of you playing rook, eating candy, taking and have a big time. Must use my space to tell about the picture equipment. And, oh yes, we had a big surprise when yesterday the boat brought the three rolls of color film slides that we though had been lost. Eastman had developed them and sent them to us instead of to you, but we are glad for we enjoyed seeing them even if we don't have a projector. We will be sending them on this boat so you will get them before too long. They are all very good. We think will have all our pictures sent back to us even though you have

to wait about three or four months for them for when we see them can improve and can know what takes good pictures.

On pp.~770 in Wards Fall Catalogue are all the things we want ad if their quick service would extend to here would order from them.

1 standard print roller 1 Ferrotype Plate (for drying pictures)

Eastman Microdol Developer (DK20)

1 gal solution Eastman Acid Fix (Hypo)

Fedco Dev. Tank only (for developing film from 35 mm to 2 Vi in wide)

Duckbill Print Tongs

Velox Glossy contact Print paper 67BP4118---3 gross---No.~2 Emulsion paper, size 2 14x3 Vi 67BP4119---1 gross---No 3 emulsion size 2 !4 x 3 !4 120 film---Panatomic X or Verichrome or anything you can get.

Also could use the little viewer if you don't need it. Claylon admits I was right and we should have brought one. Claylon says he wasn't wrong---Sly was wrong. He said we could buy them here.

Eastman Universal MQ Developer 36 pkgs (very important and I almost forgot it)

\hypertarget{june-6-1946}{%
\subsection{June 6, 1946}\label{june-6-1946}}

\emph{Coquilhatville Congo Beige, Africa}

Dear Grandma, Otto, Myrtle, and anyone else around,

Just must start a letter to you even though I ought to be doing something else. Yours was so nice and so very welcome. It came through in good time---was dated May 28 and we got it on June 4. It came much faster than some of the folks have although once in a while we get one in a week. It makes America seem very near when mail comes in a week and people fly in about 2 days from NY to Leopoldville.

You remember Mrs.~Russell? Well, you know Mr.~Russell has been hoping to get out for ages. Wanted to come with us but didn't have permit. When it finally came he really made the dust fly. Left NY Wednesday night at 8:00 and was in Coq on Saturday AM. Made the trip in 2 A days from home to his own mission station. He flew to Leo even faster than we did and then instead of having to wait there almost 3 weeks for a plane, he got on one in less than an hour. He was just plain lucky. Mrs.~Russell plans to fly out in October. You met their girls Jeanette and Eoto, didn't you? Jeanette, the youngest, biggest, and most talkative was married last Christmas. I may have told you that before. Eoto, the nurse, is planning to be married in September so Mrs.~Russell can leave knowing they are well taken care of.

Already I am a widow for a week. Claylon had a chance to go into one of the back country villages with Mrs.~Snipes, the evangelist here at Bolenge, so realizing how much he is going to need some knowledge of the work, he went. They will be back Sunday and I'm dying to hear how he liked it, what happened, etc.

Don't worry about me, now, for I'm not alone. He had decided not to go on the trip because we neither wanted me to be alone nights. But Sunday AM the station nurse came over and offered to sleep with me so he could go. During the day I don't mind for our closest neighbors house is not as far away as from your house to the road. Just like neighbors in town. Then there are others around all the time. The boy who does the housework is here as well as our garden boy. Oh, yes, we have a garden! I have been invited out to several meals and spend an hour with Mr.~Russell every morning for a Lonkundo lesson. Then I have been sewing, making more baby things. I hardly had enough so bought yds of flannel (yes, it's plentiful here) and am making more kimonos, diapers, blankets, and booties. Am taking Mrs.~Homer's advice on what and how to do them. So you see, I have enough to keep me from being lonesome.

Remember how everyone worried about what we would eat? Just so you will really feel sorry for us---one day last week we had fried chicken, roasting ears, sliced tomato and avocado pear salad, and banana pudding for dessert. That was just very common everyday grub, too, not a company meal at all. In town one day we bought 8 cans of red cherries so I came to Africa to have my cherry pie. For a pot-luck supper I baked two cakes and piled the icing on thick. Used about 1 Vi lbs. of sugar and did I have fun. We have no cocoa, worse luck, so I used Thompson's malted milk as a substitute and it did very well. Why don't you all come to Congo and live off the fat of the land?

I am taking for granted you have read our letters to the folks describing Bolenge, etc. It is such a job to write so much to everyone would prefer the news be passed around. I have written today till I have writers cramps. Instead of mailing this tomorrow I am going to wait till next week so Claylon can tell you a bit about his trip. I wrote to Doris today and after writing 3 */2 pages decided I should leave some space for him. As 4 pages is the load limit I am sure I'll be

ragged about leaving him a whole half page. I'll do differently this time and quit with two. I have part of tomorrow's Lonkundo yet to study anyway and the nurse will soon be here.

Just remember I am feeling fine and we like it better here all the time. Our only sorrow is that we can't share it more with everyone at home. Our letters are bound to be inadequate in giving all the details that one wonders about. More later---

Love, Helen

Well, I got back and found Helen still going strong. I was a little worried about being gone for a week, but I know she was in good hands. Georgia Bateman, the nurse who stayed with her, is as good as any doctor. She delivers a baby here for the native women every day or two. The natives think she is almost a magician. They say that when she is here even the old women have babies. Helen is now writing to the folks. Be sure and read it when you are over there.

I had a wonderful trip about 50 miles into the back country. We went part way on the truck then rode our bicycles for two hours along a little road about two feet wide on each side was dense jungle and over head for the most part we were completely sheltered from the sun by giant trees. The trip was so interesting and so full of new experiences that it would take a book to tell it all. The bicycle ride, the native villages, the churches, the house we lived in, the hen that was setting in the dining room and came off with her chicks, the fruit and gifts, the monkey which the chief gave us for dinner, the goats, the two weddings in which the groom made the bride's dress, the church services, the requirements for becoming church members, the baptism of 27 new members which I helped do, and the trip back again through the forest to the highway, all makes a story too long to tell now. But I plan to write it to the folks next week so you be sure and read it. Wish you could visit us and see for yourself just how wonderful it is here. There is so much to tell but we are so busy studying and working that letter writing pushes us. Even now it is nearly bedtime and I have to write my folks. Do take care of yourself, tell Errett's and anyone else we know hello and to write to us. You write again too.

All our love,

Helen \& Claylon

\begin{center}\rule{0.5\linewidth}{0.5pt}\end{center}

\hypertarget{june-13-1946}{%
\subsection{June 13, 1946}\label{june-13-1946}}

Dear folks,

Air mail is certainly great stuff. We had a letter from you on June 11 that you had written June 3 so 8 days is almost as good service as when we were in Nashville. Altogether we must have all or at least most of your letters for we have 7 or 8 from you, 1 from Grandma, and one from Doris. Hers and the one in which Dad \& Hilda wrote took much longer than usual in coming.

We are living quite normal lives here now with nothing unusual happening to me. Claylon's being gone last week made things quite lonesome but he's back now so I'm appreciating him all over again! I spend most of my time just puttering around waiting for something to happen! Still have sewing to do but have enough done that we could get by in a pinch.

Today we saw something interesting---our first iyonza (eonza) with the whole mission present. This is the official welcome to the returning missionaries given by everyone. This was for Gray Russell and Georgia Bateman, the Bolenge nurse. We all gathered on the church lawn and all the school children paraded in and gave a program. The same principle as a program back home-first the first grade sang a song, then the second grade recited in unison scripture and a prayer and sang, then their next school---our 3rd, 4th, and 5th grades sang. Last the higher school or about 6th and 7th grades (I think) presented a pantomime. Since they were welcoming their nurse it was in her honor. There are three other white women working in the school now so they had three boys dressed in their dresses to represent each one. They brought in a patient on a stretcher and took him to each one to be given medicine. Although they administered remedies, the patient did not recover. Then their nurse walked in and when she doctored him he rose right up from his bed and walked. Quite a tribute to Georgia and all very cleverly done. (The women said it was too true to be funny.) These folks are natural bom actors. In fact, Georgia says they are a real problem because they act so well that it is very hard to tell when they are really sick. The families had brought in gifts---eggs, bananas, pineapple, pai-pai, a duck, and some genuine native foods. They were divided up so we got among other things a pineapple which are not just too plentiful right now.

We have been quite busy today. Had a native make us a bassinet---a lovely wicker basket on a little stand of its own. We made the mattress for it. Got some native cotton blankets and cut them our just the shape of the basket, then tacked them all together like you do comforts. Georgia gave us enough rubber sheeting to cover the whole thing just like a pillowcase so it should be well protected. We have some sheeting and tomorrow will hem them to fit. The next big job is to line it and cover with mosquito netting. Claylon did most of the work today. In fact he just cut and sewed like a veteran. Even wore a blister on his finger. I asked him how many times he planned to do this but I got no response so guess he doesn't have his mind made up as to that!

It has been delightfully cool here the last few days. So cool that those who have been here awhile go around in the mornings with the windows closed and sweaters on. Personally I enjoy every minute of it and feel like a million dollars so long as it lasts. It has been raining about every day and stays cloudy. Is raining right now but we have needed it. Unless it rains every few days things get terribly dry. Our garden is doing fine. Especially the com and beans and tomatoes. The lettuce didn't come and our radishes were so thick we wonder if they will ever radish.

We were so happy to hear that the last roll of pictures were so good. Gee, if we could only see them. We could hardly believe they would be, especially those taken in the air. Would like for you to tell us more about each one in detail on that roll. In Leo for instance, was the scenery really clear? One was a poinsettia bush and a whole group of natives ran out to be in it. Claylon was mainly interested in the scenery so we wondered how the boys looked. He knows why the river was no good. While we were there someone tinkered with the camera and turned the whole film back up into it. When he wound it out was not sure which number he had taken last so evidently he shot no. 10 twice. Because of this we wondered if the whole film had been ruined. Claylon finished a roll last week in the back country which he plans to send on the plane tomorrow. We hope it will be good. Am afraid our main trouble from now on will be that the film doesn't keep too well and some of ours isn't too well packed. Also may have been in the camera too long. Will appreciate your comments. They are all he has to go by. Now he has in a roll of black and white which Mr.~Rowe will develop for him so we can see it. Then if it's any good will send it on. We can have the black and white done here; in fact, Claylon may buy Mr.~Rowe's outfit and do his own. If so we can at least get some idea of how they look.

You tell Lila that she will see a real picture show when your grandchild arrives! Then when she has here second one you can compare notes, gossip about them and just have lots of fun. Am glad everyone is getting fun out of the outfit.

Claylon has a wonderful story to tell you about his trip last week but that will be enough for a letter in itself so will send it next week. He is planning to write up his experiences as a story but wants to tell you a few juicy bits that couldn't be published. We have to study Lonkundo yet tonight and write his folks. He is working at that now. Besides learning a vocabulary and translating sentences we are to write a story in Lonkundo for tomorrow. It is lots of fun but this is sure no easy language to learn. Besides saying the words one must use the right tone. For example, the word ``jipoku'' all spoken in a low-level tone means ``a hole in the ground'' and ``jipoku'' with the ``ku'' raised to a different tone means ``a girl.'' Just think---such a little difference between a girl and a hole in the ground! There are many words like this and although they understand us if we get it wrong, they know we don't speak like natives. They just laugh and say ``that's the white man's Lonkundo'' so Gray is trying to teach us so we'll get it correctly. Am afraid he has himself a big job.

Say, we wondered if we could bother you to see a bout something else for us. Still our one biggest desire is for a refrigerator and, in spite of what we were told, it has to come from America. Someone had a letter from Servel Co.~saying they would begin taking orders for foreign shipment by June 1. I don't know if there is a dealer for Servel in either Winamac or Logan or not but if so wondered what information you could get for us. We would even be happy to have our name on a list (ha!) It would probably have to be ordered shipped direct from the company to Keating but the dealer might do it. We are not choosy either. If we could get any kind of kerosene one would take it, Electrolux or anything else. We are sorry now we didn't try harder to get a used one. However, the problem of packing is greater and if the new ones are coming out soon, won't mind waiting. Even if we could get our name on a list it will probably be well on to a year before we could actually have it. You know we have the money and still plan to sue what Winamac gave us for one, even though we have to wait.

We have a new gasoline lantern! We were surely given wrong impressions about a lot of things at home. We can get gasoline and can use it in lamps, even stoves---but must just be careful the natives don't tinker with it. They are not allowed to have it because of the danger and

their lack of knowledge about the mechanics involved. And just to think we could have brought lamps and lanterns from home! Well, we're learning.

This is quite a prosaic letter but just you wait till next week when Claylon takes over! He'll give you a little idea of what his job in Africa will be like. I can say right now that he was really thrilled and unless I pack up the baby and trot along, I'm going to be a widow quite often. We look forward to your letters every week and they are coming through fine now. We counted up your postage bill the other day if you keep up this airmail business and decided we weren't worth it but hope you think we are---at least for a while. Must study now.

Love,

Helen \& Claylon

\begin{center}\rule{0.5\linewidth}{0.5pt}\end{center}

\hypertarget{june-20-1946}{%
\subsection{June 20, 1946}\label{june-20-1946}}

\emph{DCCM}

Dear folks,

Guess it is my time to write again especially since I promised in our last letter to tell you of my trip into the back country.

We are both doing fine. As yet you are not old grandparents but you can just count on being one any time now. Jr should make his appearance during the first week in July according to the doctor here, but they tell me that one can never be sure just when.

We had a native carpenter make a lovely bassinet and we almost have it ready. Have only to drape the mosquito net over the outside. We will send a picture of it some of these days. As to pictures we are sending a roll on the same plane as this letter. It stayed in the camera nearly a month so stands a good chance of being bad all the way. I did not make a complete list of this roll either. No 1 was taken at the UMH in Leo and is of the cook and Simon, the one we liked so much. Simon is the taller one and perhaps with the smile. Iforget now how they were dressed, but you will know them as two of the mission boys. No 2 was taken of the old Auk plane on which we flew to Coquilhatville. No 3 is of one of the single ladies houses at Bolenge---Mrs.~P.D. Snipes formerly of Medaryville. No 4 is of the truck and native boys just before we left for the back county. If it is good and you look close you can see Helen just back of the truck. I am not sure but No 5 may also be of the truck. All the others were made in the native village which I visited. It is very hard to make pictures of the natives. You see on the light meter their faces register only 25 while the white shirt registers 300. There is such a difference that it is hard to reach a medium. Even so I do hope some of them are good. There are shots of the native houses, the small church, of large trees and of the natives themselves. I'll explain them all to you when we get home or when you come to visit us.

Dr.~Yocum is coming to Bolenge in a few days and we are all excited. I have been assigned to barbecue a pig for a 4th ofJuly celebration. What I do will depend upon Junior. An important conference is being held in Leo in which Dr.~Yocum is chairman of something. So he is coming to Bolenge for a week before the conference begins. Several of the older people from the mission are also going to the conference.

We have so many things to do. We spend about 7 hours a day in language study. The natives are always coming around for something. Right now one is coming up the path. He will

take 15 minutes at least. So you see everything included makes our day a busy one\ldots\ldots{} Back to my writing---the boy brought two eggs to sell. Both were good so I paid him the five cents for them. Eggs are one of our most expensive items. The other day Helen was baking a cake. In the eggs she found one chick just ready to be hatched. Such is cake baking in Africa.

Back to my trip before I run out ofpaper. We packed food, bicycles, Bibles, paper, pencils, etc., on the truck and set out. We went about 50 miles along a narrow gravel road to a village. There we stopped and were met by a mission teacher and his schoolboys. They unloaded our equipment and lining up caravan-style set off into the jungle along a narrow foot trail carrying things on their backs or heads. A few minutes later we started on our bicycles.

The path was very narrow, built up above the water level by the State. The bridges were made of round poles and in many places the road was still under a few inches of water.

Just a moment---another salesman with beans. Guess I'll get a few for dinner. That took a long time. I was out of change and had to go to the treasurer for some.

Riding through the water was fun if it had a hard bottom. A soft bottom was not so much fun.

Also in many places the path was worn down about 3 inches so we just sailed along in this path 3

Riding through the water was fun if it had a hard bottom. A soft bottom was not so much fun. Also in many places the path was worn down about 3 inches so we just sailed along in this path 3 inches deep and 8 inches wide and I never once had a fall. Surprised myself in that respect. This ride lasted for two hours and I was also surprised to find that I could keep it up for so long. I had never done so before. In all it was very interesting. Every so often we would have to push or lift the bike over a new fallen tree. There were several of these and they were never moved, for the natives would eventually start a new path around them.

We often came to plots of open ground which had been burned out for gardens. Fire is their easiest way of clearing a new plot. In fact it is the only way. Of course much good lumber goes to waste.

When we finally reached the village we were greeted by a large crowd singing a hymn. This was followed by prayer after which came a lot of hand shaking and jabbering or that is what it was to me. Then we were shown to our house which the headman in the village had turned over to us. It was his house but he had moved out everything he owned. It was made of mud and poles with a thatched roof and had four rooms. Mrs.~Snipes had a room, I had one, and the native evangelist and his wife had one. Mrs.~Snipes and I used the middle room for a dining room. The front porch served as a kitchen with a fire on the floor, the floor being made of clay. Every morning our cook would sprinkle water and then sweep the floor. It amused me to see them sweeping the dirt off a dirt floor.

We were really very comfortable, sleeping on our camp cots and sitting in native chairs. We had plenty to eat, fruit, chicken, etc. The natives kept us well supplied.

The first night came and we had church with a full house. The pews were made of small split bamboo poles tied to poles driven in the ground. They had a back to them and were really very comfortable. The service lasted for an hour with some of the most beautiful singing also a good sermon. After the service we hung the lantern (a Coleman) in the yard, and watched the native children play. Their games were very interesting, most of them being singing games for large groups.

The church night services went on through the week much the same except for more people coming. Being the missionary I had a chair on the rostrum. While sitting there I observed benches made for 5 persons with 4 grown men and 5 teenage boys sitting on them at once. The amazing thing is that they sat that way for an hour. The kids are more attentive in church than the average kid at home. Not only were the seats overcrowded but each window was filled with heads and people sat out front on the ground or else they brought chairs for themselves. It was a common sight also to see some of the natives coming to church with their beds on their heads---I mean the bedstead. They would sit on them outside the church door.

On Saturday night we were shown a play or a program in which all the kids took part.

The unison and rhythm with which large groups of kids could recite parts of the gospels was amazing. They never made a flaw. The songs given by small boys in quartets and any other way with each one doing a part perfectly sort of does something to me inside. Actually it does my soul good.

During the week I spent the days studying Lonkundo, giving exams, grading papers and helping Mrs.~Snipes in any way I could. The meeting was like a church convention and a young people's conference combined. So it was lots offun---I spent a lot of time trying to talk to the kids and I always enjoyed that. Of course I understood very little of what was said.

Many of the kids and adults alike were wanting to become church members. I was impressed by the rigid test which they had to pass. For six months they had to attend classes

similar to the Pastor's class at home. During this week of meetings their names were called at every service and they were expected to be present. Also every day for 2 hours they had a special class in which the elders lectured to them on what is expected of good church members. At the end of the week if they were considered worthy they were baptized as new members. I wondered if it might not be a good system to use in America. Some of the Winamac members might even be more Christian.

When Sunday came we got up early for prayer service at 6:30 then with the candidates for baptism marching in a body, 27 of them, and about 200 others following we went to the stream. They sang as they walked and to me it was most impressive to see these dark-skinned boys and girls marching to their baptismal grave there to be born anew in the name of the same Jesus that I came to know in my childhood. They had asked me to assist in the baptizing and I gladly said yes. They felt honored to have me and I certainly felt it an honor to be asked. Four of us did the baptizing in a stream in a beautiful setting of trees early on Sunday AM. The trees round about were filled with gazing eyes as was the ground about the water's edge.

The service over we returned to the church. There we had a service, communion etc., two weddings and then an interesting offering. In the afternoon we again traveled the little jungle trail to the highway where the truck was waiting for us. A short time later we were in Bolenge. The trip was finished within itself, but I don't think I will ever forget certain impressions which were stamped upon my mind. It did much to acquaint me with the fundamentals of my future work here.

(for you only) This is what I did essentially. I meant to add here the humor but don't know if there is room. Anyway we 'll see. When we first got to the village one of the natives showed me the toilet. Wish you could see it. First they dig a hole like the one we dug. Then they board it up, cover it over except for a small hole ofperhaps 8 inches in diameter. The dirt is piled into a mound and the hole is in the top. Over the hole was a cooking pot lid. Around the mound were palm fronds cut off about waist high and stuck into the ground. I sure felt funny when I had to use the trick, felt as if I was in public. It was located behind the house in the midst of a banana field. I heard a story about one of the ladies here, Miss Goldie Wells, who was using one of these toilets one day when all of a sudden her foot slipped into the hole. There she was with one foot in and one out and in a rather messy situation. She escaped safely, however. Another, Mr.~W. H. Edwards, took a box and cut a hole in it then placed it over the mound so he could sit in comfort. It worked all right until the ants ate the box so that one day it collapsed and sent him toppling down the hill. The toilets are crude but they serve the purpose and for the natives are not bad.

During the week in our dining room we found a hen setting in a dark corner. On Saturday she came off with several chicks and wandered around over the house.

On Sunday we had two weddings. One of the girls had a new dress which Mrs.~Snipes cut out and of all people the groom sewed up. He worked hard on it and got it done about 10:30 Saturday night. There she was Sunday all dressed. Both girls were wearing shoes for the first time so you can imagine how they walked. The service went off all right so far as I could tell.

You might be interested in what they say when the minister asks if they will love, honor, and obey. They say, ``I'll try''. They don't have a word for ``I will''. In other sections when the minister asks the question, their answer when translated to English means---yes I'll trample on her or I'll trample on him. These things make it hardfor the missionary to keep from laughing. One of the missionaries went into the back country and found 20 couples wanting to get married. He talked to them about it then said that unless they were willing to go through with it they

should leave. Three of the women left. Gray Russell told of one time at a wedding of his.

``Tona'' means to despise or dislike or treat wrong. So in the wedding ceremony when he asked if the girl would love, etc. and not ``tona'' her husband, she replied ``I'll try, but if he `tonas' me I'm going to `tona' him.'' Such is life for the missionary. Never a dull moment. This is tommyrot but thought you might be interested. Say we haven't started keeping a diary yet so maybe if you keep these letters we can get them together some day. Our love to everyone. Be careful who you read this to. Will write again soon.

Helen \& Claylon

(from the top of the first page)

You can see who does the talking at our house this week. He has been worrying all evening for fear he would run out of paper before he was ready to quit. We are now having Lonkundo class for 4 hours every day and are expected to put 3 more in study which we do. Sure does fill up our day. Russell seems to want us to be his prize language students! Between times I sew. Have made plenty of baby things now plus 1 gown and bedjacket for myself. Of course, Claylon does most of the machine work which helps.

\begin{center}\rule{0.5\linewidth}{0.5pt}\end{center}

\hypertarget{june-24-1946}{%
\subsection{June 24, 1946}\label{june-24-1946}}

Dear folks,

We were so glad to have your letter Saturday. It doesn't seem possible but it was postdated June 18 and arrived in Coq Sat, June 22---4 days from Ind to Africa! So sorry you haven't been getting our mail. We have written 1 airmail every week since we have been here with the exception of 1 week. We are afraid that what has happened is that our letters are too heavy and they may have ween put in regular mail. Sure hope not and they you are getting them. In Leo we weighed some letters and they said we could send 4 sheets. However, everyone says that is a lot for we are allowed only 5 grams for 16 francs. Anyway we sill send less this time and hope you get it real soon.

We are still both fine. The daddy may get a little nervous at times but the mama is fine as anything yet. Every day we wonder if this will be it but now are hoping I'll still be going strong for July 4! Yocum is coming for a visit from July 1 to 9 and the station is planning to have a barbequed pig for the 4th. As Claylon is the only Southerner here now he has been selected to do the job providing he isn't occupied elsewhere. Then I don't want to miss the fun! We are supposed to entertain him for a meal July 2---so we'll see how it all works out.

Claylon wrote you last week and told all about a trip into the back country. Simply spent hours writing the letter and we are so disappointed you don't have it. We have tried to keep you informed on all the happenings, big and little, so eventually you will get them. He also sent some colored film last week airmail and we know it had enough postage so you may get it before the letter describing the pictures. (The postage was over \$2.00)

Do you remember the roll of black and white film which he started at home---had a picture of us playing cards in the room (he thought he might have gotten Errett's feet) and some pictures of our leaving in Don's car? He has taken that film out and put it back in several times so there are pictures from Leo and a lot around here. Mr.~Rowe developed the negative for that so we saw them through a little viewer and they look like they would be fairly good. We are not sending them home yet for we can send them to Leo and have snapshots made and then send the roll home. It will be just like negatives from a Kodak but you can take it to any photo shop and have it developed into positive. Leave it in the film strip and you will be all set. In case we forget to remind you later when we do send the film, would like for you to have snapshots made from the best pictures and send to Claylon's folks (We'll probably forget and repeat all of this later.)

We were much interested in your news about the church. Do wonder what will happen but have a feeling things may work out for the best. Ginther should have been told off long ago. If he's any sort of a Christian he will surely reform and if he is not all will be better off with him outside the church. If others really care about the church and doing good they won't be led off by him and his queer ideas. Somehow when we see people here begging to hear the Christian message it seems mighty small for educated, civilized Americans to fight over such trivialities. Here people have many troubles and many things happen to keep them from always doing the most Christian thing but we have never heard tell of them in any way slighting their leaders, either missionaries or native preachers. They are highly respected and their words are looked to as being the best and are to be followed. Enough sermon---just do keep us informed as we are interested.

We are studying Lonkundo for 4 hours every day. 1 hour with Russell in the AM and 3 with a native teacher in the afternoon. Just started the afternoon class last week and Dad, guess what we read the first day. The Little Red Hen for 3 hours. I never thought I would come to that

again. By 5 o'clock I could read it without the Lonkundo book about as well as you can with the book. Our boys all laugh at us a lot but yet it tickles them when we learn something new and use it correctly. Maybe in a year we will be able to talk like the old timers but not yet.

We spent the weekend at Coq with Ellsworth and Lillian Lewis. They are the couple who are taking the Rowe's place. Rowe's are going home in July. Claylon knew Lewis' in Montreal. They are a lovely young couple and we had lots of fun. Rowe's sold some of their kitchen wares which we bought. Got several nice Wearever aluminum pieces, kettle, pitcher, tray, etc. Also a waffle iron, a tiny double boiler---just the size for baby foods, and lots of little gadgets we didn't have---cork screws, bottle opener, grapefruit knife and corer, etc. It seemed good to get hold of a lot of little things we couldn't fine at home.

We have just been interrupted by an ivory salesman. He is just a little boy but he speaks good French and is sure a salesman. We are still a couple of ninnies when it comes to buying ivory and ebony pieces but we keep saying we will quit getting them soon. They don't sell much of it upriver and we do want some of the native art. It seems rather high and yet is much cheaper than it would be at home. There is quite a technique to buying the stuff. They name their price---much higher than they expect to get. You name yours far too low. Then you dicker back and forth until you reach a happy medium. It is usually about a fair price. They are insulted if you agree to their first price.

My husband is over here studying Lonkundo out loud and every other word asking me something so consequently we are neither one making much progress. Besides our class work we really need to spend about 2 hours a day in preparation in order to get the vocabulary and grammar down pat. It's lots of fun.

Am glad Dale is having such a good time. At least maybe the time doesn't drag quite so much as if he were doing regular army stuff. Say, I suppose they don't have canned (tinned) chocolates in the stores to send to the army, do they? We ate some this weekend that were sure good!

It has been quite cool here for several days but is hot today. However, I haven't suffered from the heat like I expected to. My constitution seems to be very strong but I must admit when it is too hot my disposition is not so good and my better half thinks he has to suffer. He is thriving, though, and his hair gets curlier all the time so I am not too worried about him!!

Must study no. Haven't said much but hope you get it this week anyway. Don't worry about me. One day soon I'll present you with a big grandson or granddaughter.

Love,

Helen \& Claylon

\begin{center}\rule{0.5\linewidth}{0.5pt}\end{center}

\hypertarget{june-30-1946}{%
\subsection{June 30, 1946}\label{june-30-1946}}

Dear folks,

We have so much to tell you that I hardly know where to begin. First off, no big event has happened yet but we think it may any time so are all set. In fact we are at the hospital now--- both Claylon and I. It is a little hard to tell just who is having this baby! Friday I felt bad, had lots of pains and my back hurt like nobody's business so everyone at Bolenge was concerned. Georgia (the nurse) had already said the baby had dropped two weeks before so she was expecting me to go to hospital anytime. Well, Fri PM I felt better so we stayed the night at Bolenge. Sat. she and others advised us to come on to the hospital to save a midnight trip after I really felt miserable. Claylon made reservations with the Dr.~and Sat. afternoon we came to Coq. I have felt it was sort of silly for have never felt better than I did yesterday and am still ok today. However, we're here---Claylon has a bed, too, in my room, we have a lovely big porch off our room with the whole corner to ourselves, the view out front is as gorgeous as any park, the Dr and nurses are on call anytime so what more could we ask? One would never guess this is a hospital. I heard one baby cry so think there is a new baby not far from us but have seen no evidences of serious illness. Everything is very quiet; the nurse stops by several times during the day to chat (in French) and our meals are served on the porch. We pay for our meals but the state believes in treating people so our understanding is that the room and all Drs services are free. Oh, the hardships of having a baby in Africa! Claylon is taking color pictures of the place which we hope will be good for it is just like a park. Besides native flowers and shrubs, there are marigolds, cannas, bachelor's buttons, touch-me-nots, \& pink roses all in bloom. Front and back is all a riot of color.

We had another surprise this morning-Sunday. The Oregon has been upriver taking Homers and others up and got back here yesterday. Marjorie H and children are on it and were in a while ago. While the Oregon was up at Mondombe, Billy H was scratched and Marjorie bitten by a cat with rabies so they came here to take the series of shots. They will be at the hospital every day and can not go back to Wema until the next boat leaves in a month. We thought the Dr.~might let her stay with me and Claylon go home but he said no---probably thought her two children might be a bit noisy. Anyway she'll be around and may even stay with us at Bolenge when I get back. They are not sick, of course, just must have the shots. How everyone wishes for a truck at Wema. There are good roads there and if had been one they could have driven in to Stanleyville for these shots and have saved this trip. Horners have been given \$250 on a fund for a station truck and whenever the fund is increased to the right amount and truck is available we will have one. By the way, anyone or any church can give to such a fund by sending the money to the Society and designating it for the Wema Station Truck fund. Of course, this would have to be over and above the church's regular missionary giving. But so designated it would be used for that and the Society would be satisfied and Wema -well, we want and need a truck. Marjorie says they have good roads in all directions---we could go about 50 miles to Boende, a state post and buy anything we can get in Coq. Also there are other trading posts around. More good news is that we now can have airmail service. It comes to Stanleyville and is brought by truck (plantation owner's ) to within several miles of Wema. They have a messenger boy meet the truck and she had a letter from her mother in 11 days. When it comes to Coq we get it about a month after it arrives there. She also says our house is being screened and is in quite good condition. We need a cistern and spouting to catch rainwater and when that is put in we will have running water. The bathroom fixtures are already there---tub, stool and whole works.

Don't you envy us rather than feel sorry for us? I am so excited I can hardly wait to have my family and get moved into our own house for a change (Some paragraph, eh?)

We heard yesterday that there is a shipment of refrigerators at Matadi so we have our fingers crossed about getting one. Clay Ion intends to try tomorrow to order one. Mrs.~Rowe's cook has consented to work for us as they are leaving and he is a good cook so there again we are all set. Probably things are working out too good but right now things seem too good to be true.

Yesterday we had lots of boat mail, some of our magazines, Democrats, Clay Ion's papers, a letter from his folks and one from Dale. Dale had written Apr 29. It must have gone by air to NY and by boat to us. It was good to hear from him even if it was two months old.

Our dinner is here so must quit. ---We just had delicious think steak, Irish potatoes, creamed cucumbers, red cabbage and egg salad, rolls, native raw plums and cream puffs for dinner. Ask Hilda how that is for hospital fare!

7:45 pm---Will add some more now. Just finished supper---think slices of ham, potato and tomato salad, red cabbage salad, bread and butter, and your favorite prunes, Dad, for dessert. At 3:00 pm we had coffee and coffee cake served to us. Oh the hardships we do undergo!

Also to make things more pleasant, we just had another batch of mail. Rowe's were over and brought it. Imagine we read this Sunday (PM) the letter you wrote last Sunday and Monday. Also had 2 from Claylon's folks, 2 from his L(iving) L(ink) church, 1 from mine, and 1 from Neal and Sarah. This morning Marjorie had letters from every station upriver for us and with all the papers, etc. yesterday we feel we have really rated. It does make us feel so good to have news from home---puts added pep and zest into us.

The best thing, or one of them, from you was that you have heard from us. We, too, were discouraged to think you weren't hearing when we have been writing so faithfully and by air.

We figured we have sent you 4 letters since the June 6 one (this is the fifth) so sure hope you are getting them.

We plan not to send this for a few days hoping that Claylon can add some really big news but if nothing happens soon, will send it on and he'll send a note later. We figured he would be too busy for much writing for a few days so would tell as much ahead of time as possible.

Dr.~Yocum arrived in Coq this AM. We have not seen him yet but will soon no doubt.

He is to be at Bolenge until July 9., He left NY last Wed so still did not come as fast as Russell.

So Juanita beat me to the draw---I was sure I would be first. Now don't tell me Bemiece is beating me too. I must have a strong constitution when all this gadding about over the world riding to town in a truck, etc., does not speed things up any. At least guess I'll show Dr.~Kelsey, Hilda and some others I'm not too dumb. They were so sure I didn't know my onions but bet I'll be closer to July 2 than they either one expected! If it does not happen by then or very soon after, think I'll use Lois Russell's trick. Gray said she waited 2 weeks overtime for Jeannette and was so tired of it. She saw the kids trying to jump rope one evening without much success so she said, ``Here, let me show you how!'' That was at 5:00 pm and Jeannette arrived at 11:00 pm. I'm about to order a jumping rope!

Our weather here now is perfect for my comfort. A little chilly and damp for some but I thrive. It is the rainy season and rains almost every day---simply pours. I mean, sometimes twice a day. In between it is usually cool. Today has been very cool and pleasant. Has been too cloudy for picture taking so Claylon is going to have to watch for good days. Many folks have colds but not I. While others shiver and want the windows closed, I spread my wings and glory in the breezes. Know we will have to be very careful with the baby for the changes are very sudden.

Guess this is enough for today. Wish we could ride in that new car! Tell everyone---all the Rouch's etc., ``hello''. We just can't write airmail to everyone ant it seems foolish to send them regular mail when you could have the news two months earlier. However, we sure do eat up letters!

Love,

Helen

\begin{center}\rule{0.5\linewidth}{0.5pt}\end{center}

\hypertarget{july}{%
\section{July}\label{july}}

\hypertarget{july-5-1946}{%
\subsection{July 5, 1946}\label{july-5-1946}}

Well, here I still am, up and frisky as ever. I am afraid this delay is causing you a lot of unnecessary worry but I can't help it. We just can't report a baby until there is one. This morning I feel no closer to it than I ever have for have nary and ache nor a pain. I hope that long before you get this letter you will have had a cablegram to the effect that you are grandparents, but if you haven't received one just remember we'll send one as soon as possible. If I turn out to be two weeks or so late you should have the letter first. Maybe Hilda will be honored after all!

Say, could we make a request of you? This is bad to start asking for things already but I have the urge this AM. In the first place we would appreciate a box of cocoa---it isn't hard to get there, is it? Also I would like some paper patterns. Mine are all in the trunks, you know, and it may be months before they show up. We can buy lovely materials here but you know how much good I'd do without a pattern! Since they have voiles and many nice thin materials I'd like some with rather full skirts. According to the new magazines full skirts are style again and you know how I like them. There doesn't need to be much sleeve or neck trimming although anything you find will be OK. They have pretty ginghams, too., and I want to make some simple housedresses. Front neck openings, you know! Also Claylon needs pajamas so if you could send a pattern for them and one for a slip and panties for myself. The boys are a trifle rough on the rayon ones, panties, especially.

Someone here had a box from home the other day that had been sent in a regular overseas box and all came in splendid condition. The main things to remember are to pack securely as you would in sending one to Dale. The cocoa should be in a tin (Hershey's is, isn't it?) Wrap the box in plain paper and put our address on it. Make a list of the items included and the value (remember-do not over value---we may have customs to pay) and take with you to PO. There you will have two or three cards to fill out with all the information. If they can't do all this at Winamac, they can at Logan---just go when the PO is open! Remember?? Claylon says if there is any more room in the box, fill a tin with soft candies---choc, etc., and maybe some gum. We are convinced they can arrive OK if packed in tight jars. Coffee tins for instance with tape around the crack.

Claylon has applied for the license to buy a refrigerator which is step 1. If this is granted, which the salesman thinks will be, we will be buying a new Electrolux as soon as they reach here. You know us---we ordered the biggest model, a mere \$425.

Here's the truck with folks from Bolenge.

Oh, the baby's names---if a girl you may call here Barbara Jo. If a boy, Ronald Dee. We don't want to argue after the baby comes---just do it all before! Hope this will meet with approval Must quit for now.

Love,

Helen

Dear folks

It is now 10:00 am on July 14 and still you are not grandparents. However, the prospects for today are good. Helen as been up and around every day, feeling good all the time, but today is not up to par. We are each gladfor we are tired of sitting around with nothing to do. Also at Bolenge we have radishes, greens, and string beans about to go to waste because we are not there. Our string beans (Kentucky Wonders) are higher than I can reach andfull of beans. Limas will be a few days hence. My corn is also tasseling so we expect to be eating it ere too long.

Mom, you remember Helen bawling me out for not bringing more good pajamas along. Well one pair is completely worn out, leaving me only one pair. Being at the hospital I couldn 7 get them washed and ready to wear again by night. So I got busy, cut myself a pattern and just last night finished a pair which do very well indeed. It is the first sewing job that I have ever completed on my own. I do lots of sewing but Helen is always here to advise. It costs half as much to make a pair as to buy one, so think I'll make others if my trunks don 7 soon come.

Helen told you of two rolls offilm. Hope by now you have received a roll which we sent in the mails sometime ago. The two present rolls should be fairly good unless the film has molded or something. One roll is of the hospital mainly which the other is at Bolenge \& Coq. First the hospital roll---I am not sure the numbers are exactly correct but any rate you can figure them out. (No.l) Picture of a native boy carrying a large basket on his head. Nearly all men, women and children carry things this way. I see them run, walk down steps, and do other things, all the while carrying a bucket of water, never spilling a drop. Think of carrying a round 5-gallon jug lying on its round side on your head with a bucket of water in each hand. (No.~2) A corridor in the hospital, taken inside. (No.~3) A corner scene in front yard. (No.~4) The front entrance to hospital. (No.~5) Another front scene (Nos. 6, 7, \& 8) taken in succession to show the entire left wing of hospital---maybe you can tell where the pictures overlap. We have a front corner room at the end of the corridor overlooking the yard. It's really a beautiful scene. (No.

\begin{enumerate}
\def\labelenumi{\arabic{enumi})}
\setcounter{enumi}{9}
\tightlist
\item
  Close-up of Helen (No.~11) A distant view of hospital \& the right wing. (No.~12) taken on the golf course. Several anthills in the distance. (No.~13) An ant hill---on top you can see a bench for visitors to relax. Steps have been cut up one side. These hills are very hard. (No.~14) Still another anthill. There are many of these here, some covered with grass and some with large trees growing out of the top. By large trees I mean 2 and 3 feet in diameter. (No.~15) Main street in Coq. Palms are pretty. (Nos. 16 \& 17) Shots taken from hospital porch across front yard.
\end{enumerate}

(No.~18) Shot of hospital kitchen and backyard.

The next roll is not listed properly at all since I had no paper along. As I remember then they are No.~1 Picture of Mr.~\& Mrs.~David Byerlee's house at Bolenge. No.~2 \&3 are shots of the Bolenge church. The boys school may show in the left background. No.~4- shot of side view of Edward's house at I. C. C. No 5 shot offront doorway of same house---It is very pretty. No.~6-1 think this is the right number-Gray Russell and Dr.~Yocum just before picnic began. No.~7-Shot of Goldie Wells house at I. C. C. Part of this is used for classrooms. No.~8-Shot of school at I. C. C. Boys are preparing the stage for a play to begin later. I think I made only one shot of this building but am not sure. No.~9-Rows of students ' houses at I.C.C. Nos. 10-16 shots taken at picnic-taken near night with people moving about a lot. Hope they are not blurred too badly. When we come home I'll tell you who they all are. Nos. 17 \& 18 Helen made two shots of me in front yard of hospital. I hope you like my Congo style suit. Most everyone here wears this type. Hope my face is not too badly shaded by the helmet. Nearly everyone wears the helmet. And we find them comfortable. Don 7 know anything else about pictures so will await the arrival of your grandchild. Before finishing. I don 7 think I'll have to wait long.

\begin{center}\rule{0.5\linewidth}{0.5pt}\end{center}

\hypertarget{july-16-1946}{%
\subsection{July 16, 1946}\label{july-16-1946}}

\emph{DCCM, Coquilhatville, Congo Beige}

Dear folks,

This is not a letter, just a note to let you know the news about us and our baby. I suppose by now that you have already received a telegram from Indianapolis telling you the sad news. It is true. Our baby came on Sunday, July 14, but alas it was gone when it came.

Our hearts are broken for all our hopes and dreams were suddenly dashed against that inevitable thing called death.

Though we are deeply saddened by what has happened we feel that God has done best for us. The doctor assured us that had the babe been alive it could not have lived very long. It was the fault of no one, and though we had stayed in America it would have been the same. I so much want you to understand this fact that it was not our travel or anything that we or anyone else have done that caused the trouble. The doctor has cared for Helen very well, he has spent many years in America studying medicine, so I know that he is a good doctor and knows what he is talking about.

Helen is doing extra well, in fact so much stronger than we expected her to be. I am not telling you this to just to try to make you feel better, it is really true and I am so happy about her. She will be ready to go home from the hospital in a few more days, certainly not more than a week. Please don't worry about her for she will be writing the next letter to you be the last of this week. The baby came on Sunday night and she sat up and ate her dinner the next day.

We buried the baby in the cemetery at Bolenge, and it was a beautiful sight to me. The ladies had the most beautiful flowers formed into a cross with the grave also covered. Someone took some pictures so when we can, we will send them home to you.

We are sad and we know that you are sad with us, but we are trying to be brave and look to the future. We have a stronger faith in God and in his heaven for now a part of us is there. Don't worry about us, and in a few days on the next plane if possible we will have another letter on its way to you with more news and details.

Bye and may your prayers be with us.

All our love,

Helen \& Claylon

PS. The Bolenge nurse is staying with Helen and everyone is so sympathetic and kind to us. I do hope you won't worry for we are both doing extra well.

\begin{center}\rule{0.5\linewidth}{0.5pt}\end{center}

\hypertarget{july-17-1946}{%
\subsection{July 17, 1946}\label{july-17-1946}}

Dear Mom \& Dad,

Since I am feeling good tonight thought I would write or at least start a letter even though the plane does not go until Friday. I am sitting propped up in bed with a table in front of me so am really very well fixed. We have been thinking of you so much lately, especially today for you should have our news by now. It wasn't possible to send the cable until Tuesday because Monday was a holiday and if one can cable at all then, it is double price.

Although Claylon wrote, he said he told you only the barest facts and we know you are anxious for the entire story. As you can tell from the letter which we already had started (we are sending Claylon's part) I was very miserable on Sunday, in fact from Saturday night on until Sunday night at 5:00 when I went to the delivery room. It was all over by 6:00. The Dr.~had told Claylon while I was on the way to the Delivery R that something was wrong and that he was afraid the baby was dead. He had been afraid of this since Sun morning when he examined me. The baby never breathed at all so we didn't get to have our little boy after all. As sorry as we are and as badly as we feel, we realize this is best for the baby was not normal and had not been from the beginning. The head was about twice as large as it should have been and if he had lived he could never have been normal and could not have lived more than a few years. The first thing I asked the Dr.~was if we had done anything to cause this---our traveling for instance. He assured us of what we already knew - that it was nothing we could in any way help---that it was just one of those unfortunate things that sometimes happen. Also, I asked him if I was all right and if we could still have a normal healthy family. He said, yes, that our chances are still as good as anybody's. I am in good condition and we need not fear trying again.

The shock has been hard, right at first it seemed too much to bear, but as we are beginning to get used to it we are looking ahead rather than feeling bitter or that life has been too hard. Already we realize how much worse some folks have had to suffer. We never knew the joy of holding him in our arms so didn't have to give that up and we won't have to see him suffer or know that he couldn't grow up happy and healthy. We have felt again so much for Marjorie and Howard, whose sorrow is worse than ours in many ways.

Georgia Bateman stayed with me Mon. night and I asked her about having a family. She said there was no reason why we should be afraid or why we should wait too long---just until I am again strong and well. So we will all have to wait but someday we will have our Ronny yet. We hope you will just look forward with us and not grieve too much over this. We are accepting it as one of those sorrows which must come into every life. So far ours has all been happy and bright so we must expect some shadows.

We were not able to get word to the mission Sun PM but Mon morning Claylon saw the Lewis' and they took the word to Bolenge. Everyone has been very kind. They arranged a short grave-side service for Monday afternoon. The baby is buried in the Bolenge cemetery which is just beside the church. Claylon said someone had fixed it up so that the spot was beautiful. They had put roses and greenery over the cross which marked the grave besides have put many other flowers around it.

The administrator of the colony always arranges for the coffin so all of that was taken care of for us.

Claylon said all the missionaries attended the service and many of the natives from around the station. Our own boys were very sympathetic and shed tears with the rest. I have been so touched by the sympathy shown by the natives here in the hospital. Although they don't

even speak Lonkundo and we can't understand one another, by their every word and deed they have shown their sympathy and feeling for us.

This has been very hard on Claylon for he has had to accept the sympathy and meet more people than I besides seeing me suffer Sunday. However, he has been wonderful and I realize more and more what a really fine husband I have.

Now, I am writing this part of the letter especially for Hilda. I really wouldn't bother you with the gruesome details except that I want her to know how they do things in a Catholic hospital and see how that compares with the way her Drs do them. My paper is going to give out so will write note on here and you can send all the letter or just this part to her as you please.

Hilda, we though we might be going to give you a birthday present but now I'm glad we didn't---it was just 2 days later. I think you warned me I might have a hard delivery and I surely did. Everything was against me. The water broke on Wednesday morning before the baby was bom on Sun PM. I passed a lot of blood during that time and since I felt this wasn't natural, had a strange feeling that something was wrong. Finally Sat night I started having slight labor pains but by 3:00 am Sun. they were no longer slight. By noon I was in agony and by 5:00 pm I thought I could stand no more. All the relief or sympathy I had Sun. was from Claylon. The Sisters said Oh, the pain is good, that means the baby is pushing. She was practically insulting when Claylon asked for something to kill the pain. Neither would they give me anything to bring action faster. She said they were waiting for the rest of the water to break (which never did) And all the time I suffered the Dr.~knew or thought he did that the baby was dead. I supposed the worst was over when we started to the delivery room for never a kind word from the Sisters. You can realize how hard it was when the water had broken that far ahead, I had already suffered till I was practically hysterical, and the head was twice normal size. But I lived through that hour without any pain relief and had the baby without a tear or cut. He didn't have to take any stitches and I'm soon going to be as good as ever. All the help I had from the Sisters (2 of them) was to hold me, wipe the sweat and tell me to push harder when I thought I hadn't a thing left to push with. I never tried so hard to pass out and couldn't.

Dr.~Dormal is a fine Dr.---studied in America and I'm sure knows the best practices but he is first and foremost a Catholic. It was only after this was all over that we learned they will never give anything for childbirth regardless. They mustn't interfere with life's normal processes! They didn't even give me a sleeping pill Sun. night so Claylon \& I each slept about 2 or 3 hours. However, I seem to thrive on it for on this the 3rd day, I'm feeling fine. They make me stay in bed, yet I roll myself from one bed to another while they make them. The nursing care is good enough but sure different from what I'd guess yours would be. Well, I just wanted you to compare hospitals. Don't feel sorry for me. I lived through and since I don't think this was all necessary am not afraid to have another baby. Don't think I could face this ordeal again but from now on I'll trust our mission Drs \& am sure things will be easier. Wouldn't mind, though, if you would reassure me that this is true. Since it's over and I have told you am forgetting this part of it and am looking forward. Write,

Love,

Helen

\begin{center}\rule{0.5\linewidth}{0.5pt}\end{center}

\hypertarget{wednesday-july-24-1946}{%
\subsection{Wednesday, July 24, 1946}\label{wednesday-july-24-1946}}

Dear folks,

My but it was good to have your July 7 letter. We hadn't heard for about two weeks and were anxious to know what was going on up there in Indiana. We have been worried about your getting our last mail for we understand the Constellations (Pan-Am planes) are all grounded for a month. We are not sure if that means no air mail or whether they have other planes running. We hope so. We realize that if you got our cable and no letters you will be worrying but there is nothing we can do only keep writing. Just always remember that if you don't hear from us it is due to mail service and not because we don't write.

Guess we need to be careful what we say if you are broadcasting our letters over the countryside! Say, aren't you afraid you will take all the wind out of our sails for when we get home? We won't have anything left to talk about and not even any pictures to show! Seriously, though, I think it's grand you are boosting African Missions. Go right ahead and we'll feed you all the thunder we can.

Right now we have some more pictures to send but these are black and white. We have been so anxious to see some so sent two rolls with the folks to Leo to have developed. Can have regular snapshots made off these you know. We have seen them and are quite pleased. One roll Claylon took in the back country is not too good but we think it's because the film was old--- some Wesner had given him. Also we had left it in the camera too long and the black spots on it are mold spots. We had two sets of shots made so are keeping one and sending one to Claylon's folks. The negatives we are sending to you---as usual by air. Now, they are not ready to use in the projector for these are the negatives just like we would have from the Kodak but you take them to the photographer and have them developed into the positive. They can be left in the roll if you wish. Also if anyone wants pictures they could be made but we don't recommend them too highly.

We are still taking pictures. Have a film in now with 36 shots already taken. You should get some before too long. Yocum is due back to the States in a week or so with 2 rolls.

I am still at the hospital but am up most of the time. This is the 10th day and Dr.~says I can go home tomorrow if I promise not to do anything. That I gladly promised! Really, there is nothing to do when the cooking, washing, cleaning, etc. is all done for us. Everyone marvels at how well I have gotten along. Even the Dr., who says very little, told me I had been a good patient. Mom, I sure am appreciating my Spencer. It was such a help beforehand and now. I have had it washed and laced up good again and it sure feels good. I still haven't regained my girlish waistline but still have hopes whenever I can start taking some exercises and riding a bicycle.

Our last freight has reached Matadi so if it comes through customs as fast as the other we should have it in about two more months. Won't we be glad for some clothes, beds, etc.! Just hope not too much has been damaged. Also got our MW invoice at long last. They finally shipped the stuff we ordered last Dec.~20 on June 20 so we might have it by New Year's. You know I was sore before we left home. Well, you should see Claylon now! None of the things we really needed were included. No dishes, no mirrors, no baby bed, and lots of other smaller items were left out. Now we'll have to buy a mirror and set of everyday dishes here, and they are higher than anything. However, we already have another order made out for them.. It is fun picking it out of the catalog even if it never comes and we might get some by this time next summer.

Oh, yes, about the hose. Mom, whatever you can get on my cards, you keep. That is one thing I really don't need. Am sorry I didn't leave you one of my 3 prs. I wore the same pair all the time we were in NY and all the way across and they are still perfect. One pair I have never yet had on. In fact haven't worn them but about once since we arrived. No one wears them--- not even for church so I now have my all sealed up in a glass jar to keep the roaches out. Hope they will be good to wear home. I will be glad to get some socks. Just have one pair but have several in the trunk.

I didn't tell you about my green winter coat, did I? Remember we saw Ruth Musgrave in NY? She had a package that needed to get out here for another person quickly and since she was coming by boat was afraid it wouldn't arrive in time so I traded her my winter coat for the package (overweight problem) as I wouldn't any longer need the coat. Well, they had to put all of their suitcases in the hold of the ship and they got wet. Lots of their clothes were ruined--- among them my coat. She claimed insurance---valued the coat at \$50 and said I had worn it one year. She really didn't know. Anyway I had worn it 2 years had bought it for \$39.95 and she got \$35 for it. I figured that was a pretty good selling for something I wouldn't need and would be out of style in 3 years anyway, don't you think?

We like the way you use your stationery. Can read it fine and we sure like all we can get on it. Ours is too thin for that.

Dad, I suppose that you are all fixed to take to lecturing if you just get the chances with a new suit, new car and everything. Think you fellows had better get a camera and send us some pictures. Let us see what new things have taken place. Am glad you are not having to worry too much about the farming. Think your hen house idea is not bad. Folks have certainly lived in much worse.

Talk about farming---Claylon's garden takes the prize so everyone says. We have missed out on a lot of it but will get back in time for roasting ears and sliced tomatoes. Everyone ate radishes till they got too big and hot and I guess everyone on the station has had green beans. They say they are simply loaded. Lima's, too, will be ready in a day or so. We even had lots of nice lettuce and the old timers say you can't raise lettuce here. Guess Claylon just hit things right for we have had lots of rain and some fairly cool weather.

Mr.~Cobble from Monieka has been here and has invited us up to visit them for 2 weeks. We can't get a boat to Wema until Aug 24 but on Aug 6 we can get one that goes as far as Monieka, then go on from there the last of Aug.~Now we have nothing to keep us here except language study so guess we will accept his invitation and hope to study there. We can visit them and Hendersons. Also Martha Bateman will be going up on that boat. She is at the Leo meeting but lives at Monieka. We are anxious to get on to Wema in Aug to have some time with Boyers. They have decided to stay till we arrive but want to leave in September. Claylon and I can both get into the work now and we are dead anxious to get started.

Hi, folk, as usual I get almost a half-page to give my side of the story. We are still going strong and as Helen said will get to Bolenge tomorrow in time to get what's left of my garden. I have two rows of Kentucky Wonder beans about 10 feet long and last Sat I picked nearly a bushel.

They are fine. I know your garden in also very good. It would be well now to see all of the chickens, pigs, and calves. Guess by now you have cut the oats. Of course all the other things are flourishing.

I was interested in your picture story, am glad that you are using them and only wish you had more good ones to use. Also wish I could write you a little story to go with each of them.

You should have gotten a roll by now which we sent by airmail on June 12. I think I explained them in a letter written about that time. Also you should get the two rolls which Dr.~Yocum has within the next three weeks. We will try to make several rolls on our trip upriver. I won't promise to make any close-up pictures of the first leopards or elephants we see. Wards are sending 8 rolls of color film so will have plenty for a while. I have started a roll now which I will try to finish in and around Bolenge. I made 5 shots of the baby's grave. I took the camera out the day after the funeral. The flowers were still very pretty but it was very cloudy. I made 5 shots and hope at least one will be good. We still feel badly about the loss of our son and your grandson. But we have tried to accept it as a part of the will of him whom we are here to serve. The past will lose itself in the future which we always look to to heal our pains by bringing greater blessings.

I can't remember now exactly what I told in your letter about my jungle trip. You said though that you read it to your class. Iam curious as to what their reaction was. Also want to know the latest news about the church. Whatever happed to Dr.~Ginther? The paper is full so must stop. Love to all and we will look for another letter soon.

Helen \& Claylon

\begin{center}\rule{0.5\linewidth}{0.5pt}\end{center}

\hypertarget{august}{%
\section{August}\label{august}}

\hypertarget{august-4-1946}{%
\subsection{August 4, 1946}\label{august-4-1946}}

\emph{DCCM Coq}

Dear Mom \& Dad,

Sunday afternoon seems to be letter writing time both in America and Africa. The only difference is I'll bet you're not as cool as we have been on this August day. I wore my sweater to church this AM and this afternoon we rested and I covered up with a double blanket. I have been really cool several times since we came home from the hospital, whether the weather has been cooler or whether it makes that much difference to me. I am not sure but Claylon says the same. That it has always been this cool sometimes. Others say we are having unusually cool weather.

Well, at last we are getting ready to do what we have been living and working for for ages---go to Wema. We leave here on state boat Tuesday afternoon. It will take us from 7 to 9 days to reach Wema, depending on the weather and how well we made connections every place. We leave the boat at Boende (a state post and town similar to Coq) and a storekeeper there is to take his big truck to Wema. They are good friends of the Homers who have made all the arrangements. The truck trip will take about 2 hours so in a little over a week from now we will be home and oh how glad we will be. We have thoroughly enjoyed being here but to be in our own home and settled will be wonderful. Did we tell you our Leggett order has arrived here and our last trunks etc. from Winamac have come to Matadi? We hope they will be here by September so the Oregon can bring them up. Then we will have some beds and a change of clothes.

Mom, if you had a husband like mine, bet you wouldn't have had to live 25 years and more without a full-length mirror! Our Montgomery Ward order does not include a mirror so we had nothing bigger than a small windowpane size. They had some here at Coq that are 5 feet long and 20 inches wide so Claylon bought one. It was without a frame so he spent another 2 days making a nice frame for it so now I have a lovely mirror for our bedroom door.

\hypertarget{august-8-1946}{%
\subsection{August 8, 1946}\label{august-8-1946}}

You see, I got stopped right in the middle of talking about how well I can see myself! Quite a great deal has happened since then. Right now we are on a boat sailing along on the Ruki River headed toward Wema. However, in order to bring you up to date, I'll go back to Sunday. The thing that stopped me was that we had company---a group of Wema students who came to bid us good-bye and visit for a while. Goldie Ruth Wells came along as interpreter.

This is the same group that gave us the iyonza of fruit, etc., when we came. We can talk to them a little by ourselves now but not enough to have much of a conversation. After that visit, we went to Edna Poole's house where we all had a pot-luck supper before church. Every Sunday night all the missionaries get together for a white church service. Not always a supper but they come quite often.

Getting ready to leave on the boat trip has certainly been an experience. You would never have guessed that we left home with only airplane luggage! We had bought and accumulated enough food, furniture, and stuff of one kind and another that we had almost a truck load. Getting it all in solid boxes, except what we would need on the trip kept Claylon busy for several days. We had to bring food with us for the boat trip so we keep that in our cabin.

The boat is a rather small river boat. It has two decks. The natives all live on the bottom deck and the white passengers are up above them. There are 6 cabins with 2 persons to a cabin

but last night there were at least 14 passengers so some had their beds outside on the deck. The passenger list is made up of 3 Catholic sisters, the state administrator for this district, his wife and 3 children, and four other men, at least some of whom are coming out to begin work as government officials of some kind. Of course, they all speak French but fortunately 3 or r4 of the men speak some English. One young man speaks it very well. He is fresh out of the army and although he is a Belgian he was with the American first army in Germany so is quite familiar with Americans and their ways. He speaks very highly of our army. Says the soldiers were fine fellows and he liked our food, cigarettes, etc. for we had more and better than anyone else. He is to be some sort of official in our territory so we will probably be seeing him quite often as they often visit the mission stations.

He is acting as sort of head cook and tells the native boys what to do. We each give him some cans or fresh food or whatever we have and together they fix up the menu and we all eat together like one big family.

Bet none of you could guess what we have with us. The prettiest little white kitten you ever did see! We had a cabin full without him but we always find enough room to keep him comfortable. I never thought we would come to it so soon. Practically everyone at Bolenge has at least one cat. Especially they like these white Persians with such pretty blue eyes. We sort of laughed at them for all babying their kittens so much but here we are at it already. Claylon just can't bear to hear him cry so has him in bed with him, dirty feet and all unless I object because of the tracks he leaves. This morning he even had to get up at 5:00 to find the kitty!

We left Coq Tuesday afternoon about 4:00. We were much interested in the river as this is our first boat trip. The Congo is so large that it is hard to realize it is just a river. That evening we could see many large islands in front of us. We always thought they were mainland with other rivers coming into the Congo but when we came close could see they were islands. The banks were so far away when we were in the middle of the river that we couldn't see much of the jungle. We knew that sometime that night we would leave the Congo and travel on the Ruki River but although we kept watching, never knew when we changed from one to the other. This afternoon we should reach Monieka which is on the Busira River but I don't know whether we are on it yet or not. Later then we leave that for the Juappa on which we live, although this boat only goes up it as far as Boende. It is interesting to watch the jungle as we sail along. Part of the time we are very close to the bank, then we strain our eyes to see a hippo, crocodile or some other big animal, but all we have seen so far was one crocodile yesterday. Claylon spotted it clinging to a dead tree trunk. Haven't even seen a monkey. Claylon has taken some pictures which we hope will gave a better idea of what we are seeing.

The boat shakes enough that I am having difficulty writing. It really seems very smooth except when I'm doing this but there is just a little jerky motion. We stop every few hours at native villages along the way for wood. It certainly takes lots of it to keep the steam up. There is a good breeze---in fact, I can hardly hold my paper still and it is always quite cool. At night, a sweater feels good.

Something interesting may happen before I have a chance to mail this so will stop for now!

Sat. noon---Sometime this afternoon we are supposed to reach Boende so will finish this now and mail it there. We have had a very pleasant trip, but it seems strange to think that it took us less time to cross the ocean than to make the boat trip which is actually only a few hundred miles. The jungle is pretty and interesting so we enjoy the view. Have seen no animals.

Yesterday morning we were at Monieka for an hour. Mr.~Cobble came down and talked to the captain so we stayed longer than usual. The boat stops right in front of the mission so we went up to Henderson's. We were glad to have a chance to visit with them some. We saw everyone on the station---just Hendersons, Cobbles, and Mrs.~Hedges who is in bed at Hendersons' with a broken kneecap.

You realize that from now on it is going to be harder for us to send mail so much. However we understand we can still have good airmail service through Boende. Just how it works we will have to tell you later. Also how to address our letters. For now, do them as you have been.

Mr.~\& Mrs.~Rowe and Dr.~Yocum are still in Leo being held up because of the business of the Constellations all being grounded. Dr.~Yocum is having to miss the International Convention, which is too bad for him and the people, too. He should have made reports on Africa and the conference here.

We do hope you have had our letters before this but know you may not have had. We have had none from you for about 3 weeks and know it will be at least 2 weeks before we can expect any but we know they are on the way so don't worry.

Hope you can read this. Bye and lots of love from us both.

Helen \& Claylon

\begin{center}\rule{0.5\linewidth}{0.5pt}\end{center}

\hypertarget{august-27-1946}{%
\subsection{August 27, 1946}\label{august-27-1946}}

\emph{Wema, DCCM}

Dear folks,

I almost dread to start a letter to you for I know it is such a long, laborious task to tell you all I would like to. If you could just sit down with us here for a while it would be fun to tell you all about our work and we could show you the people so much easier than we can describe them. However, since that is impossible I'll do my best to give you a picture of Wema as we have seen it during our first two weeks here.

We arrived two weeks ago yesterday, August 11, just as church was being dismissed. We caused quite a lot of excitement, both among the missionaries and the natives, as we were two or three days earlier than they expected. We were as excited as anyone to see our new home and to meet the folks who are to be our associates through years to come. To go back just a little---our ride out from Boende was quite an experience. We came in an old truck that had a board for a front seat and no springs under the board over roads that you wouldn't call roads at all. (They are now building the road and in some places it was not bad.) The back of the truck was piled high with all of our stuff, among them my mirror and two good chairs. Every time we hit a bump that knocked us practically out of our seats our hearts dropped for we knew the mirror would never make it. To make matters worse, our driver was a native and we couldn't understand what he said. When we started we thought he asked us if everything was all right and we very agreeably said `yes'. However, later we decided what he asked was if we wanted to go fast for we simply flew through the villages and over the hills, and all we could do was to hang on tight for we didn't know how to tell him to slow down. We tried to think of something in our meager vocabulary that would get the idea across, but there was no way, and our dictionary was in the back of the truck. As we neared Wema he shouted to the villages ``the bondele (white man) has come'' and there would be shouting and waving of hands as we rode through. Well, we arrived about 10:30 after 3 hours ride no worse for the wear. Everything came through in good shape. The mirror wasn't even scratched, thanks to Claylon's packing!

It was a mighty pleasant surprise to drive into Wema. In fact, we didn't know we were there until he so announced after he had stopped before the Boyers house but we had just been commenting on what a beautiful place it was. We approached through a bamboo drive that is the beginning of the station which is perfectly beautiful and as we came out of that there was this lovely park-like place all covered with well-cut grass and many palm trees everywhere. There are several green, big ant hills that add to the beauty.

After greetings had been given, we were taken to our house. Boyers and Horners apologized greatly for not having it completed and all cleaned up for us, but we though it looked mighty good (especially after all the stories we had heard). They had been working hard every day trying to beat our arrival and did have a lot done. There is still much work to be done on it but we are settled now and feel for once that we are really at home.

We spent the first week getting moved in---painting furniture, putting in more screens, cleaning floors, etc. before we worried much about the station work. We had to get boys to work for us and it kept us busy keeping them busy and showing them what to do.

We did start off with some excitement though. On Monday night we had white ants.

They are the kind that work so fast that they can about eat up all the books or clothes in one night and they will eat anything of that sort that comes in their way---paper, cloth, leather, and wood.

It is tragic for them to get into the books, clothes closets or suitcases. They are tiny things and

come through very small cracks. Our floors are cement and it was poured in sections which left good-sized cracks. Also it was done 10 years ago and has not been taken care of so was not in too good shape. Well, we were at Boyers for supper and when we returned saw a pile of white ants in the middle of the living room floor where they had come up through a crack. We debated what to do and decided to use DDT. By the time we had it unpacked there was another big bunch in front of the door. Upon investigation we found they were all over the house in their little piles. We used DDT very freely and it killed them right now. We sprayed everything we thought they might eat and went to bed. The DDT is great stuff for they never came any farther. The next day the masons had to pour cement in all the cracks all over the house. We haven't had any more trouble with them so for.

Well, Tuesday night we started to bed early after a busy day but just after we turned out the light we heard a strange noise. We debated very deliberately whether we should investigate for it really didn't sound like much of anything but decided that maybe Claylon had better look around a bit. I turned our gas lantern back on and he opened the door just in time to see a black shadow dart out into the kitchen. He said his hair stood on end but he followed the shadow but was too late to see the fellow go out through the kitchen window. Just heard his footsteps and saw that he left a piece of soap and one of our crystal goblets on the window ledge. You see, our windows were not finished---didn't yet have hooks on them and no screens in the kitchen. We had a sentry or watchman at the house but he was asleep on the front porch. Anyhow Claylon says to tell you he chased the fellow so hard that he left what he had taken and ran for dear life. The sentry hunted him but imagine hunting a black man in the dark!

We had unpacked our china and crystal that day and apparently this was someone who had seen us do it. He had tried to take a dish that had 5 of my china cups in it, but apparently when he heard Claylon come he set them down on the floor and tried to get the first thing he could get his hands on. However he made away with nothing and, needless to say, the next day we had locks put on all doors and windows. It seems that so long as thing are locked all is OK but if not, they do walk away. They are not so bold as thieves at home for they won't break in. Anyway, we caused excitement and Homers and Boyers have laughed at how quickly we are getting broken into Congo life.

I guess I am having more experiences this morning at how things get as I am trying to use this typewriter. It belongs to the station and has been in the office so long that the rollers won't roll the paper, the ribbon won't work and part of the keys stick so it is really testing my Christian patience to use it. The only advantage is that it is possible to get more on a page than in long hand but it is sure a struggle.

The first week we were here we had a station meeting to plan the work for the next few weeks or months until Miss Ward gets here to help out. The Boyers are leaving on the same boat with this letter, about Sept 8 or 9 and Claylon and I will be in charge of all the educational and evangelistic work for the Homers know nothing at all about what goes on in the school here or about the back country work. Just so you can get some idea of what is ahead for us, I'll try to describe a day's work. First, here are some of my responsibilities: We get up at 5:15 and I must be at school at 5:45 to unlock the door. I will be in charge of the morning school which runs from 6 to 9 every morning. This is made up of village boys and those who live on the station who want to come to school. Now there are about 200 enrolled. More about the school later. Sometime during those first three hours, I go up home for breakfast, then at 9 I can go home for the morning. On Tuesdays and Thursdays I am supposed to meet with the choir at 1:15 and at 2:00 we have language lessons. On Monday and Thursday at 3:00 o'clock we are to meet with

the minister and his Bakimi class or those who are being taught how to become Christians. At 4:00 every day I have to be at the Women's school until 5:00. That ends the regular routine for the day except on Wed. and Fri. nights when there is either prayer meeting or Christian Endeavor at the church about 6:30 which we are expected to attend. Then on Friday afternoon after the boys all get paid I have a bank for them and sell supplies. Last Friday was my first day and it kept me busy all afternoon just banking their money while Mrs.~Boyer sold supplies. Don't know yet how I will manage alone. Then Saturday morning we have to be in charge of the native market which is a mess. The villagers bring in things to sell which they have gathered from the forest and the women on the station buy them. The native women are harder to handle than the men and if they don't think they are getting their share are as apt to start a fight as anything. They are also harder to understand than the men and so far I can't get one word of what they say. Well then besides all this for the ordinary day's work, it was up to me to take Mrs.~Boyer's job as mission secretary-treasurer, which means writing any station letters and doing all the bookkeeping, of which there is a lot both for the mission and the state. Also keeping the bank accounts for the missionaries and the station appropriations from the United Society. Well, during my spare time I have to keep things moving at the house. While I don't do the housework myself I have to see that it is done. All the rest of the time we are supposed to study the language. The book says every new missionary should spend 8 hours per day on the language alone for the first year! Oh, yes, then there is Sunday---the first lokoli rings for prayer meeting about 5:15. Sunday school is at 8:30, church at 90:30 so we get home about 11:00.

Then at 4:00 there is an afternoon service in one of the nearby villages which all white people attend and at 7:30 we have our own white service.

Don't get the idea that I am doing all the work for Claylon's day goes something like this: He has to be down for roll call with the men at 5:40 and get them all started on their days' work. He has a crew of carpenters, and masons, besides the boys who cut grass, palm nuts, etc. All have to be issued tools and told what to do. Then at 10:00 he is in charge of the second-degree school which lasts until 2:00. We have some boys who work on our yard and do odd jobs so after our lesson he has to see that these all get done. On Friday while I will be banking, he and Dr.~Homer will be paying all the men. This takes about 3 hours. This weekend he and Mr.~Boyer are going out to some villages about 3 hours from here by bicycle to baptize and take in the offering from the churches there. There are about 50 of these villages and we should visit every one of them before Christmas for Boyers say there are about 200 or 300 people waiting for the missionary to baptize them besides their wanting to turn in the years' offering. They have already baptized 500 during this past year. Then besides this regular mission work, Clay Ion has to finish our house which means putting in a cistern, a septic tank and plumbing fixtures, etc. and help Howard build their new house which is just being started. Well, I guess that is about all we have to do for a little while at least until we get on to the work a little better!! It may not be quite as impossible as it sounds for there are native leaders for some of the jobs, the schools, etc. However, we do think it looks pretty nearly impossible to us. To make matters even worse, Howard leaves about Sept.~20 for two weeks to go to Mondombe as Wema's representative to the MAC or Mission Advisory Committee meeting so we will be on Marjorie's hands so far as language and all is concerned.

Whenever Miss Ward comes, our tasks will be lightened but she does not have her permit yet or at least didn't just a few weeks ago so she might be here on the next boat or it may be the first of the year. You know haw that goes.

The other day after school I wrote down some of my impressions of school as I saw it after only one or two days. Since I though you might be interested in what the school is like, will just copy my impressions:

MY FIRST DAY AT SCHOOL

It was still dark when the alarm sounded at 5:30 but I knew it was time to get out of bed if I was to be at school by 6:00. Even with my hurrying, the teachers beat me there and were singing their opening hymn when I arrived. Following this they had a prayer for guidance in their day's work. The first half hour the teachers spend in study and on this day they were having a lesson in the Bakimi book with one of the teachers presenting the lesson and the others pretending to be his class. The Bakimi book teaches how to become followers of Christ and the teachers will use this with those who want to become Christians when they go out into the villages to teach.

At 6:30 the bell rang for school to begin and nearly 200 pupils lines up outside the schoolhouse and marched in very orderly fashion with their teachers to the assembly. School opened with a hymn and prayer, after which the classes marched to their places as they sang their `going to class' song.

What an inspiration it was to see all these boys who had come to school because they wanted to and not because they had to. They came from three different villages around Wema and had gotten up early enough to walk in from several miles around and be at school by 6:00.

Of course, they are all sizes and ages but some of the youngest must not be over six years old, perhaps not that. Many of them do not know their ages but they are small and so nearly naked that they must suffer from cold for the early mornings are quite cool. Sometimes I need a sweater but they have no sweaters to wear.

The schoolhouse is one large room with benches in the center where they assemble. The classrooms are on each side but they are not really separate rooms. They are partitioned off more like stalls with no doors between groups. Each class has a long table and benches for the pupils.

I was impressed by how well each class group paid attention to the teacher and did not seem to take notice of other groups. Naturally, there was a certain amount of noise in the room with about 10 or 12 teachers all teaching at once but it was a healthy kind of noise---that made by people working rather than the kind that is created when children are being mischievous.

(Incidentally, right now I seem to have a class. There are four boys standing right outside my window watching me type. I asked them what they wanted and they said they wanted to see me write so I am letting them stay. This is quite a trick to them.)

This school is known as the first-degree school and includes grades 1 and 2. Of course, that hardly compares with those grades at home for some of the work is much harder but for the most part my third grade at home could do the work that a lot of the teachers here do. They study reading, writing, and arithmetic. The plan is different from most school systems in America in that each pupil progresses at his own rate and whenever he has finished a book and is able to read it and write it, he is allowed to go on to the next without waiting for a whole class. The children are much slower in learning to read here so they may be in school for two or three years doing the same work. However, they do nor seem to become discouraged and keep right on coming.

They learn more slowly for several reasons, I think. First, their teachers are not well trained. In fact, most of them are just barely able to read themselves and they do not know how to teach others. (This is one reason for the need for more missionaries so there will be enough to

really be able to train the teachers. As it is, each missionary has too much work for any real teacher training classes.) Another reason why children do not learn is that most all of them are undernourished. They actually do not have enough food to eat and a lot of what they have is not the kind that builds healthy bodies. Also they have no opportunity to read much except when they are in school.

These things I either observed or was told by the missionary in charge on my first day at school. In two weeks I will be the missionary in charge. No doubt many impressions will change but now I feel very heavily the weight of responsibility in helping to train these children and young men so that they will become Christian leaders and workers in Congo. (End of story)

\hypertarget{thursday-august-29-1946}{%
\subsection{Thursday, August 29, 1946}\label{thursday-august-29-1946}}

While things seem to be moving along, I will see what I can do about putting some more in this letter. I can always think of lots to say when I am not at the typewriter and guess from the looks of this letter I think of enough when it is before me. Since letters will have to be more infrequent, they will just have to be longer, I guess.

This morning there is lots of activity around here. There are a group of about a dozen boys chopping grass out of the yard. We are getting ready to plant some regular clover grass which they use in yards here. Another group are working on building our wash house out in the back yard; another group is fixing a water tank for us. Then I have one boy in the kitchen, another doing a washing, and I think one is working in the garden. It sounds like we would be getting volumes of work done but it sounds bigger on paper than it shows up around here. In the first place, they don't any of them hurt themselves working, especially when they know we are not looking, and they when every lick of work has to be done by hand, it takes much longer than when we do it at home. More and more every day we are realizing why things move so slowly out here and while we get impatient, have to keep realizing that things are different.

You know, I told you that Mr.~Boyer and Clay Ion were going on a trip this weekend. As it has turned out, Mrs.~Boyer and I are going too. Don't know how I will get along on my first real bicycle ride. They say the place is 8 or 10 miles from here, no one knows just how far but is about 2 Vi hours over some quite hilly roads so that we will have to walk part of the way.

Anyway we will probably have some experiences to write about as a result of the trip if I am even able to write next week.

Here it is not quite 9:00 and I feel like the morning is about gone! Our alarm goes off at 5:15 but on, how I hate to get up just as badly as I always hated to at 7 at home. The other night we were talking with Boyers about getting started on this trip Saturday morning and Mrs.~B. thought we should try to go early so it would be cooler. She suggested leaving about 6:30. He said, ``You know you will never get around by then on Saturday morning. Why you know how hard it is for me to get you up then.'' She said ``Why, I don't think I lie in bed very late on Saturday.'' And his answer was ``I can't get you up before a quarter till 6 no matter how hard I try!'' What would you do with a lazy missionary like that?

Last Sunday afternoon we were given our names. Every missionary is named for some native Christian who has died. They always meet together and decide on the name and then have a little ceremony of presenting it. I am now Ncimo (pronounced Nchemo---the i is always pronounced as e and c as ch.) Claylon is Bofola.

Also you will notice I am sending you the carbon copy of this letter. It seems to be readable to write on both sides with the carbon but the black type is so heavy that you would never be able to make it out so decided to do it this way. We are going to keep copies of our

letters from now on anyway so that we will know what we have already written to you so we are keeping the original of this. It makes no difference how many sheets we have in our file but it does make a difference how many we put in your envelope.

I talked about how much was going on here but work is all going to cease now quite soon for the fellows go to school at 10:00. Guess I had better quit and check up on what is going on right now.

\begin{center}\rule{0.5\linewidth}{0.5pt}\end{center}

\hypertarget{september}{%
\section{September}\label{september}}

\hypertarget{september-4-1946}{%
\subsection{September 4, 1946}\label{september-4-1946}}

Sunday was a big day here---the first boat since we arrived came and we had lots of mail. Two letters from you, 1 from Doris, 1 from Grandma, from Holwagers, from the whole Vanderbilt crowd that was at the convention and oh, several others.

Hope Dale is really getting home soon and that big events will be happening.

About the refrigerator---are glad you have our name on a list for we aren't getting one from here after all. It seems state people come before missionaries. I do think a new one would be best and we can wait until they are available. Surprisingly enough we haven't missed one too much although some ice cream and jello would be mighty good.

We had a wonderful trip over this weekend. It must have been nearly 15 miles out to this village---Bofaka- and we rode our bicycles out Saturday am and back Sunday afternoon. I want to tell you all about the trip but have neither time nor space in this letter so it will just have to wait until next time.

Tell everyone ``hello.'' We appreciated Grandma's letter but just can't answer it right now. I think she will understand how truly swamped we are. Fortunately, we both feel fine. I am perfectly OK again. The trip didn't even make me very sore. Even yet I don't life anything at all heavy without a reprimand from my better half!

Love to all,

Helen \& Claylon

\begin{center}\rule{0.5\linewidth}{0.5pt}\end{center}

\hypertarget{september-22-1946}{%
\subsection{September 22, 1946}\label{september-22-1946}}

\emph{Wema, DCCM}

Dear folks,

Sunday night again! Someone is going to Boende tomorrow so the Homers and us called off our church to write letters. Is that too heathenish of missionaries to dismiss a church service of two families so we can write to our folks? We're all kind of excited today anyway for we expect the Oregon any time. In fact we thought it might come last night but since it didn't will surely get here tomorrow. Since we only have company from the down river stations once every two years when they go to Mondombe, it is quite an occasion and everyone has been trying hard to get things all cleaned up and in good order. The fact that it has rained hard every day this week hasn't been much of a help but most everything is done so the place looks about like a park anyway. The best news yet is that Merle Ward has arrived and will be coming up on the Oregon. That will certainly lighten our load considerably.

Today is Grandma's birthday and I never even sent her a greeting. Have thought of it several times but just was something else that didn't get done. So tell her ``happy birthday'' for me.

The Boyers have been gone just two weeks and they have been busy but interesting ones for us. It didn't take long to find that there is lots we don't know but slowly we are learning. Since there is no one else to do it, I'll brag on us a little! Homers said just today that they are constantly amazed at how well we get along with the language. They had expected that we would be down often to have them help us settle palavers (problems or troubles). So far we have had to call on them very little. I teach (?) my arithmetic class and supervise the morning school without too much trouble and Claylon does the noon school. For two Fridays he has paid all the fellows---teachers, workmen, etc., about 100 in all and I have banked their money and sold supplies (books, paper, pencils \& pens) without having to call on anyone to translate. That doesn't mean we don't still have a long, long way to go but we think it is a bit of progress.

I'm having a hard time doing this for watching our cat catch his tail. He has a big time doing it but has learned how to sneak up on it so he really gets it in his mouth. We are mighty proud of our cat. He is as white as snow with great big, blue eyes. He is a Persian, guess we told you that, and we think he is going to have a real fluffy tail. When we get some more color film will take his picture. His name is Fufu which means very white.

This afternoon we went out to the nearest village and met the chief. He is quite sympathetic to the Mission although not a Christian himself. He has 28 wives in this village and 10 somewhere else and has at least 23 children. We saw one wife with huge brass anklets on. They must have come halfway up to her knees and are quite thick. I lifted one once, not nearly so large, and it was very heavy. These denote great wealth. The more wives and goats a man has, the richer he is. He buys wives with goats and if he is really rich he puts these anklets on his wives. We might have a picture of a woman with anklets, I'm not sure. If not, we will take some.

Before I start on the story of our trip to Bofoka, I want to tell you a little about our house \& some funny things that happen. Hope I'm not repeating. Johnson, who planned and started this house some 10 years ago, must have been some man. How he stayed a missionary with all his screwy ideas I can't imagine, but he had them, especially along building lines. I told you about all the cracks in the cement floor---well, he built little round holes in the comer of every roo---of all things---to sweep the dirt and water through. Imagine in a living room and bedroom,

scrubbing with enough water that you need a water hole! Anyway, we didn't want rat and snake holes so had the masons fill them up. They muttered ``One white man tells us to do something and another tells us to change it.'' Then he has 3 doors to our bathroom, one of which leads to our bedroom, one to our office, and one outside. There is no window and only a little peephole in the outside door so you know what a dark hole it is. Well, we are putting in a window and taking out the outside door and again the masons and carpenters laughed at us ``white people'' but they don't know yet that there is a little room on the other side of the house that has a window instead of a door and that we are going to change that! That will probably be the last straw. Johnson never thought of the stairway you see, and we have searched and searched for a place for it. The only solution is to use this one little room to make an outside entrance so that we can carry things up the stairway. The upstairs is nice after you get there. Has a very pretty floor and big windows at each end.

Here is a model house plan. \textbar\_\textbar{} are windows and \textbar\textbar{} are doors as they

are now. The X's are the ones we plan to change. That's not to scale but our living room-dining room is mammoth---about 31'xl6'. The back porch we are now screening and plan to use for a breakfast porch. There is still money in the bldg, fund for it and a big part of it is supposed to go for inlaid linoleum for all the floors. I sort of gasped when they said that as they are expensive but as we have been told the most expensive thing about floor covering is the freight out here and inlaid will many times outlast cheaper linoleum so all we have to do is order it and wait for MW to send it or say no. We hope to have it before the end of our first term. Right now I want something else and am imposing on you to see what you can do for me. I want window blinds at our bedroom windows. Now we have cloth sort of like un-bleached muslin trimmed with some native print that we use as drapes and draw shut. They are not very pretty so I want pretty lace curtains and blinds. If we order from MW it will take so long that thought you might get them and send them parcel post and we could soon have them. Our bedroom has white walls and we painted the ceiling and door facings, etc., a very light blue---wall finish. With our pastel-trimmed bedspread I think the room can be made very pretty. You remember I bought some plain white curtain material in Logan but would rather have a light blue or blue trimmed tiebacks. Now you may not even be able to get blinds---certainly, they will have to be cut to fit for they have to be 56'' wide and 5' long. We need two like that. Then one for the bathroom 30'' wide and regular length. Our windows are very peculiarly shaped. Counting from the edges of all the sills, the windows are wider than they are long and are up rather high. Anyway, if you could get curtains they should be at least 6 ft long and would be better if 7,1 think. Would keep the window from looking so squatty. Now we intend to pay for anything you can get. The blinds would come out of the house funds and we will get the curtains. Will send you a check on Society if you can get anything. Don't worry about them. If you can't get curtains OK. You might send me some Rit dyes and I could color my own. Wouldn't mind having some dyes anyway.

Since I'm on subject of house will continue. I think the funniest thing yet is the way J(ohnson) was going to fix our water supply. He was going to spout the water from the house top to a tank in the bathroom. The tank is here how -one of those horse tanks like Bill Vernon always had. That was to be hooked somehow to the ceiling to catch the water. Then he was

going to fix some sort of overflow pipe to run the water outside when it rained too much. It made me think of Granddad and some of his modern conveniences. We have a cistern almost finished and have a pump which will pump water into a tank upstairs and we will have running water for that. Our septic tank is being started so before many weeks we'll have a modem bathroom. They put the stool in place yesterday and we already use the tub.

We have learned that it was from J(ohnson)'s peculiar ways that many false impressions have been circulated about Wema. It is not a backward station and the only reason it doesn't have more houses is that he didn't build and wouldn't let anyone else. That is all changed now that he has retired. By the way in the last World Call there is a little news item that Bolenge gave 1000 francs to UCMS from natives and as an added thought that Wema gave \$50. Well,

\$50 is 2000 francs and is the most any native church has ever given for work elsewhere. We are the only station whose native churches are all self-supporting and don't have to have funds from UCMS. We have a better schoolhouse than Bolenge, also a better church building (Virgil Havens built it) a better hospital than most and admittedly the best carpenter shop on the whole mission. So we don't quite know why folks gave the impression that we were going to the end of no-place. Anyway we don't think we are there yet. In spite of its little peculiarities our house is a mansion and the station is beautiful. The people are fine to work with and we are as happy as larks in the work. By the way, the tin houses are no ways near as bad as Sly and others shook their heads about. Are quite cute, in fact, and since we have a very few mosquitos, don't miss screens like you would.

Well, I sort of got off on a tangent but Wema folks don't appreciate everyone thinking they are to be pitied for being here. I must quit now for it is bedtime \& space is gone.

Next time I promise to tell you about our trip \& native customs. Also about 3 rolls of color film that Boyers took to Coq to mail. You probably won't get them before we write again in 2 weeks when the boat goes. Our color film is gone so will have to wait for trunks for more pictures.

Hope everyone is will and that Dale gets home soon.

Love

Helen \& Claylon

Say, this is not necessarily for publication. It might sound gossipy. Just for home folks only!! Dear folk,

You see Helen used more space than the law allows and I almost got left out entirely. Just decided to start another page. Postage is high but we won't know the difference in a hundred years.

Part of Helen's letter sounded very (????) and rightfully so. However the blues are now all gone for Helen did a wonderful job of mattress patching---they look nice and sleep wonderfully. As she said the sewing machine is my job. The machine is a dandy but the cabinet must be entirely rebuilt.

Today has been a busy one. We have worked all day getting things put away. The food is now all arranged on our storeroom shelves. Excess pots, pans, tubs, buckets etc. are along the wall. It looks like a store in reality. While we were working two of my carpenters were working in the house. They happened to see some of our pictures---the ones of you folk, Dale \& Doris, etc. The folk here have always bragged of how pretty Helen is. Well when they saw the hand-painted photo of Mom \& Dad they really howled. But whose picture were they howling over? Bet Dad

can guess. Yes it was his. They created so much noise till some other workmen came in to view the wonder. Their expression was ``Ise ekai ale jituka mongo.'' Literally it means ``her father is beauty itself'' ---it is their most emphatic expression. They also though Dale very beautiful. The funny thing was that the women were pretty a little while the men were the very symbol of beauty. Well fellows (assuming that Dale is there) guess that will hold the women in their places for a while.

Do hope Dale is there, the wedding all over and all of you getting along fine. Sure hope work clothes are available now for I really played havoc with what Dale had last winter. I looked at our pictures today and looked long at the ones which Helen made of Dad \& I picking corn last fall.

I can't go beyond this page this time. Do intend to write soon and hope to get my say in first or I will not have a say. Our love to everyone and thanks so much for the delicious candy which came in perfect condition. We now have sweet teeth and sore gums from candy and gum. But oh how we like it. Continue to keep us posted on all the latest news and we will try to do the same from this end. We expect to have a good story when we return from our trip to the back country. Bye now and all our love,

Helen \& Claylon

\begin{center}\rule{0.5\linewidth}{0.5pt}\end{center}

\hypertarget{october}{%
\section{October}\label{october}}

\hypertarget{october-4-1946}{%
\subsection{October 4, 1946}\label{october-4-1946}}

\emph{Wema, DCCM}

Dear folks,

It is soon going to be boat time again ant that means letter time. We always think that when we have a whole month, that we will get our letters written way ahead of time but we never do. Then sometimes we have a chance in between boats to mail some and that is really a rush. We have been working on an order to Montgomery Ward for days and if they would really send all the things we ordered, we would have some bill but we know that most of our fun comes in making out the order for they won't send but a small part of it and not that for months, probably.

Before I get started at something else like I usually do, I am going to tell you about our trip into the back country to Bofoka. We made it the first week in Sept, so you see it is already old stuff but we just never did get around to writing about it. We went the last weekend that Boyers were here. They wanted to make one last trip and to show us something of what they are like. We left here on Saturday morning about 7:00 after getting all the carriers started on their way with our belongings. They tied the bundles together with some of the strips off the trees that they always use for rope and ran long poles under the rope, then a boy gets hold of each end of the pole and puts it across his shoulder and off they go, faster than most of us could walk. We had to take our camp cots, camp chairs, a folding table, tub, water bucket and wash pan, dishes, cooking pots and some iron racks to put over an open fire for them to cook on, and enough food for the two days we were to be gone. Even though it was only a two-day trip, it took about as much stuff as if we were going to be gone two weeks.

We had a lovely ride out. It was a cool morning and the forest paths were beautiful. I had visions of little narrow paths all grown up with weeds and underbrush but these were as good to ride on as any road, if fact part of them were better for they were hard dirt rather than loose gravel. We did have some rather high hills but it was fun to ride down them. We walked up, of course. One or two were so steep that we even had to walk down. We stopped often along the way and visited with the people. They were all glad to see the new white people (or so they said). We met one group of women coming to the market with big packs on their backs. They looked me over and said they thought I was very beautiful for I was big and strong looking. That is their idea of beauty! Then they told Mrs.~Boyer that I should give them each a meya for telling me that. (A meya is a coin somewhat equivalent to our penny). I didn't think it was worth that much to me for them to think I was big and strong!

About 9:00 we stopped at one village long enough for Mrs.~Boyer to question the Bakimi. Those are the ones who have been studying with their teacher getting ready to be baptized and become Christians. Sometimes they haven't studied long enough and cannot be baptized for they must know certain things before they can join their church. They get more studying done before they are baptized than they do after so they are fairly strict about that part of it.

We got to Bofoka shortly after noon so the cook who had gone along fixed us up a nice lunch and after that all Claylon and I had to do was rest while the Boyers questioned more Bakimi. We couldn't be a bit of help with that. After they finished we rode on up into the village. Their villages are usually strung out along the road and may stretch for two or three miles. You see everyone lives in these villages and usually they are all related. We must have ridden two miles to the end of it when we came upon a man roasting his dog. He said it was sick in the head so he had to kill it and they were going to have a feast that night. We supposed it must have had rabies but even if it had been really sick they would have eaten it. It was being

done to a turn, head, skin and all. We thought, that is, Claylon and I, that that sight might affect our supper but it never fazed it. We ate as much porcupine as anyone! Yes we had porcupine and it was sure good. Better than rabbit----just as tender and nice as could be. They had gone out and hunted it that day so that we would have fresh meat. Also they had brought in bieya from the forest, which is something like asparagus at home, but we think is better. This and noodles were cooked with porcupine over the open fire and boy were they good. Also a fresh, sweet pineapple to top it all off wasn't bad!

Our house was very nice for a native house---in fact, it was the best they had. Someone had moved out so that we could have it. Claylon and I had one room and the Boyers another. A third we used to store things and a little porch-like room on the front we used for a dining room. It was made of mud but they are lots better than they sound. The mud dries out and sticks to poles which they have put up first and are not too different from a sort of crude brick. The doors are very small and I guess they had a joke on me. They gave Boyers the biggest room so she asked her cook why they had done that. He said that the door on it was smaller and since they hadn't yet ``measured'' me they weren't sure that I could get through it. We just laughed good about that but I don't think the cook ever did see the joke. Anyway I made it a point to go through the door quite often to prove that I could!

We had a good church service that night---at least I guess it was good---everyone seemed to enjoy it and the church was full.

Sunday morning was another big morning. The first thing was prayer meeting at 5:15. After that we had breakfast and then went down to the stream for the baptismal service. Claylon and Mr.~Boyer did that. There were 35 baptized and it was perfectly beautiful along the bank of a very pretty, clear stream with the sun just up good and all the people along the bank singing. It was one of the most inspiring things I have seen yet and to think that all of these people are dependent upon such a few of us for guidance in their Christian lives. It has meant a whole new way of living for them. They usually even move out of their heathen village and build their homes around the church where the other Christians live. After the service we saw people running as hard as they could go and we wondered where they were going. Imagine our surprise to find that they were running to get seats in church. It wasn't time for the service but they knew that if they didn't hurry they would be left out. Many had to sit on the outside and look in over the walls. Every seat was crowded to overflowing and the aisles were full.

After church there were three couples that received their Christian marriage. That is quite a thing to understand. I am not sure we understand yet just all involved in their marriages. However, the husband has to buy his wife from her father if father isn't a Christian and the father can demand as much as he wants to. Usually the husband is paying for her for several years. Until the father is satisfied, he can always demand that the girl come back home and there is nothing that she or her husband can do about it. Even in Christian marriages this has to be done for very seldom is the father a Christian. So before the young couple can be married in church, the father has to agree that the payment has all been settled and that the girl really belongs to the man. Usually this is all settled before the service but try hard as they will, Boyers say that often there is a slip. One of these happened this time. It seems that the man's family is rather well off and the girl's father know it and wants all that he can get. This is a custom that the church can do very little about except to teach that we do not believe in this and hope that the second and third generation Christians will change their ways. At that there were three couples married that morning. Of course, they had been living together for some time, some of them have children

but now they are both Christians and so are going to try to live as nearly up with their beliefs as they can.

It is surprising to us that they do as well as they do rather than that they sometimes fall short of the highest ideals when we realize how little training they have and how many obstacles they have in all their traditions and customs.

Well, after church we ate a hearty noon meal and started on our way home. This was really a hot trip for we came through the very hottest part of the day. We were from 1:30 till 4:00 getting back and we traveled all the time except a few minutes now and then when we had to catch our breath. We saw many people out in the forest catching beto (caterpillars) for their food. They are great big caterpillars and not wooly like the ones at home. They do have lots of prongs or feet, I suppose, all over them. They cook them different ways. One way is to boil them a long time in a mixture they make from palm nuts and another is to roast them over the fire after they have been wrapped in some of their native leaves. One of the missionaries says that they eat them whenever they can get them. They boil them a little, take the skins off and fry them real brown. We have gotten to the place where we can look at them and watch the natives cook them without squirming but I haven't yet brought myself to the place where I could taste them.

Well, our trip is over. We arrived home that evening tired but happy to think that we are out here where we can have a small part in all of this most interesting and worthwhile work.

Now that Miss Ward has come, we expect to be making some real itineration's soon. I think we are going to have a school vacation during November and spend the month visiting as many of the villages as we can get to, then start our new school year the first of December. It will be a great experience to make this trip I think. By the way, we never saw any wild animals nor any traces of them. We heard one monkey but couldn't see him. It seems that animals are never seen anymore. In fact the natives say that they can't find enough for their food.

It has certainly been cool here. We sleep under blankets all the time and the other night I had to get out our wool blanket I was so cool. We are supposed to be due for hot weather before long for Nov, Dec, and Jan are supposed to be the hottest. We haven't had any weather since we have been in Africa as hot and unpleasant as we have every summer at home. In fact, right now it is raining a nice little shower and there is a cool breeze. It will be a perfect night to sleep.

We certainly had a time when the Oregon came in the other day with all the folks from down river going to the conference at Mondombe. They all had supper here and there were 21 of us. It was just like a big banquet. Marjorie Horner's cook and mine divided things up and then about all of our boys helped with the serving and dishwashing until it was just like eating at a big hotel! Ha, ha! We splurged with all of our nice crystal and china and we did have three very pretty tables. Our room is plenty large enough and by borrowing some things from Marjorie we were able to seat everyone at once. Now they will all be back though again either tomorrow night to spend Sunday with us or else sometime on Monday. They will just about finish up our rabbits we are thinking, for that is all the meat we have to feed them. However, we will always have more rabbits coming on, you know. Homers raise the rabbits and we took over the Boyers chickens. There are 48 of them and we usually get about 12 eggs. That is not so good but it does keep the station in nice, big, fresh eggs---Rhode Island Reds.

I had to take time out during this to help Claylon hold our cat while he bathed him. He is getting so big that he fights right back and scratches like everything. He would stay pretty and white if we didn't get so much red clay carried in on our floor and he rubs it all over him.

Anyway he didn't like the bath idea and I think Claylon has decided he his a little too big for that.

We are still getting work done on our house slowly but surely. The masons are working on the septic tank. They had been told just how to do it and were working while Claylon was down putting some of the other fellows to work. When he came back up he found that they had started to brick it up at about a 45-degree angle from the house. First he thought he would just leave it that way since it would be underground and no one would see it but then he climbed down in the hole to look it over more closely and as he jumped down off the ladder he hurt one of his toes. That didn't help his feelings any so he just told the masons they could tear the bricks out and do it over, that they knew they were supposed to put it in straight, that he didn't like it and that Howard wouldn't like it either. So very patiently the masons tore it all out. It does look better this time.

The Oregon brought among lots of other things, 100 80-pound bags of cement. Miss Ward brought all of her furniture and there were other supplies for the station. The beach is about 'A mile from the station down a steep hill and every bag and every bundle had to be carried up on somebody's back or else pushed up in a wheelbarrow. Then on Sunday, the state boat brought us more cement---this time in 200-pound bags. The state beach is even farther away---at least \% of a mile and again most of the way uphill. This the boys had to carry in wheelbarrows. Claylon hated for them to have to do it at all and especially on Sunday but the boat dumped it off and it even then looked like rain and you know how much good cement would be after a good wetting. The big men were not even to be found that day so the small boys had to do it. Claylon had to help them load the bags for they couldn't manage them. So you see how much we have been wishing for a truck these last few days. There are good roads to both beaches and it would have been much simpler if we would have had some horsepower behind the loads rather than all manpower.

I think I have talked about enough for one time. Guess it is time to let Claylon take over.

It seems that Helen has to do all the writing lately---but I literally don't have time. From 5:00 AM till 8 or 9 PM is quite a full day and I am always ready for bed. Up till now I have been unable to rest at noon because of the school. Miss Ward is now relieving me of this burden.

Even so Hike it very much and once I get the language I expect to teach regularly.

Helen told most of the news so here's about some pictures. The ones which we took from NY to Lisbon came last week. They look very good and do we wish we had a viewer---We sent you a list of them long ago. Perhaps you still have it. We are sending the pictures on this boat. Also three rolls of black and white which you can have printed---they are already developed. We sent three rolls of color film about 1 month ago. The first was made of Bolenge (I think) and of places we passed on the way upriver. I can't find the list for them now. The 2nd is as follows---1) Native soldiers in Boende, 2) Boys working at my house (PS the ones who carried the 200 lb. bags of cement), 3) grass boys of the mission, 4 \& 5) group in Bofoka around small antelope, 6) Mrs.~Boyer and Helen resting after reaching Bofoka, 7) Cooks washing dishes, 8) the village church and Bakimi class, 9) native house with baby in front, 10 \& 11) Teacher and his baby, 12) group of village boys 13) a group of girls, 14-18) shots made at creek, very early and rainy--- very likely not good. The third roll, not written. Shots of Horners and Helen \& I, of Boyers ' house, our house, of ant hill, of school \& boys houses. I didn 't write these down so just can 7 ' remember them all.

You asked about pictures which you have received---the animal in question was an antelope. He had been dead for some time---almost made me sick. They were made on the village trip which I told you about back in June.

The blacks and whites are hard and many to describe---one was made in the village on the first trip the same as the antelope. Merely village scenes. It is the roll which we started at home as you will see. Also has several Bolenge shots which you will recognize. The second is made entirely in and around the hospital in Coq. The third was started at the airport in Coq when Dr.~Yocum was leaving. Also the Rowes---the large crowd of natives came out (walked the 6 miles) just to bid them goodbye. Also some shots of planes landing and taking off. Then I returned to the hospital and made other shots of Helen sitting up in bed on the porch. She also made mine as you can see. This takes care of the back and whites.

I do remember that some of the color pictures on the first roll mentioned were of the baby's grave. It was so pretty and I surely hope they will be good.

Must close now---It is late and I must get up early tomorrow. WE enjoy your letters so much -especially the long newsy one. Hope Dale is home by the time you get this. We had two letters from Doris last week.

Say hello to all the folks. We think of them often and see them in our dreams. Life here is wonderful if we are strong enough to take it. I am sure we are.

Bye for now,

Helen \& Claylon

(Someone said one-page letters are more effective than longer ones. If you agree you are sure having to struggle with this.)

\begin{center}\rule{0.5\linewidth}{0.5pt}\end{center}

\hypertarget{october-20-1946}{%
\subsection{October 20, 1946}\label{october-20-1946}}

\emph{Wema}

Dear folks,

Claylon says you'll be sitting comfortable before your new oil burner as you read this. How about it?

We do here on a rainy Sunday what folks at home would like to do. We just don't have Sunday School and church and all stay home and be real lazy. It is too cold and wet for people without clothes to be out and folks with clothes prefer to stay in! We have decided that we want a fireplace in our house for these rainy days are cold and damp. It feels and looks and sounds just like a fall day at home. We can hardly wait for our trunks to come for many reasons but for one thing I will be glad for some dresses with sleeves. Folks keep talking about the hot season but it still hasn't arrived, neither has the dry one which was supposed to have started a few weeks ago. It has simply poured since 5:00 am and is now 1:00 pm and is just now slacking off.

Someday soon we are going to have a pretty lawn. They have dug up and carried off all the old coarse grass and weeds and have leveled it up ready for planting. We plant clover in all the yards. That sounds funny to you, I know, but this is an entirely different kind of clover from yours. It looks something like little 4-leaf clover but is very fine and never grows tall. It creeps along the ground and once it has covered a yard chokes out all the weeds. It never has to be cut for it probably never grows higher than an inch off the ground. Sounds good doesn't it? We just cut up a patch of it from one spot and set it out in rows and eventually it runs together and there is a beautiful green lawn. Don't need mowers, you see. The weeds have to be pulled until it covers everything then a going over once every week or two does the trick.

Have I told you we are going on a 2-week itineration soon? School will close Nov.~1, then we will spend a week getting ready to go, then itinerate for 2 weeks, spend another week recuperating and getting ready to open school about Dec.~1. Miss Ward is the one who will be doing most of the work but we'll be along. We sure hope our trunks come on the boat next week for if they don't it will be another month at least and the bottoms are out of Claylon's shoes and mine are neither very good nor comfortable anymore. We didn't realize it would be so long and I didn't bring any with arch supports. Boy, will I be glad for some of my old oxfords. Have 1 pr of oxfords but they are those very flat heeled ones and have no supports.

What I had started to say was that when we get back I want to plant some flowers. Have zinnias, marigolds and maybe a few other seeds. I'm sorry I didn't bring more for I'm sure most any flowers will grow. At the hospital they had glad's and dahlias and all kinds of seed flowers; pansies, touch-me-nots, petunias and others. There are some beautiful big red cannas here. If I had been smart would have put bulbs and seeds in my trunk more than that 1 iris that was probably frozen.

Mon. AM

Dad, I wish you could have seen the boys building our chicken house. It sure looks like no chicken house you every saw. The process is like this: first they stick poles in the ground about a foot apart all around for the sides of the bldg., as high as it is to be built. They split big bamboos and put them on for the roof like this I /jT\l ---two bamboos with the edges turned up and one turned down over the two inside edgesW They are hollow, you know, and are split

in two lengthwise. Today they were mudding the sides. Oh, yes first they tie strips of bamboo across the sides so the frame looks like this ~ Then the boys bring dirt and water and

wade in mud up to their knees mixing it ' while others throw it against the side of the

house. When this dries it makes a very solid wall and really doesn't look bad. Sort of reminds me of stucco. Claylon is making 2 rooms in his house se we hope to really be in the chicken business soon. It was too dark this AM but if possible he is going to take pictures tomorrow of the boys working.

Our septic tank is finished except for the top and the cistern just needs one half of the top yet. A part of the bathroom fixtures are installed now so in another week or two will be so modem we won't know how to act.

We are getting anxious for news from you. Haven't had any for ages. Just hope everyone is OK Probably we'll get a big bunch of letters on the boat next week. Sure hope so We have wondered about Dale. Hope he is home or on his way.

Mom, it seems as if there is always something we are wanting but it is so hard to get some things and MW is so infernally slow. One thing is certain---no one ever need worry about what to send us. We can use anything and everything. If we don't use it for what it was intended we use it elsewhere. Right soon we will be needing cupboard door catches, handles, and hinges. Also always need screen door springs and braces. We are going to build cupboards in the kitchen and have no fasteners and very few hinges. If you can get them (maybe you can't) would sure love some. I don't know how many but several for we want high wall cupboards on two walls and low ones under the tables on at least 2 walls. Any kind of door catches and handle pulls would be fine. I saw them in the new catalog but we hate to wait two years. Did order a kitchen sink so we may have it before our furlough. Hate to keep suggesting things but still know if you can get them it will be much faster when you ship by parcel post.

While I am asking for things, will go a step farther---this time for the station. If anyone at Winamac ever asks you what we need---want you to have something to tell them! Ha! (Don't do this yourselves.) Miss Ward \& I want to start some new kind of women's work and at the same time have a kindergarten for the tiny children. We need all sorts of supplies---crayons, scissors, colored paper, pictures, even balls to play with. In fact most anything children there would like. No paste for we make that. Pictures and picture books (coloring books) but not many. We can trace patterns \& so many are not suitable. For instance they don't know cars, horses, cows, trains, etc. I'm hoping I packed some stuffed animal patterns but don't remember. Never did get Lila's. Clay Ion's boys need balls---they play with oranges now. Soft balls, rubber balls, and soccer--- (he really wants but is expensive). Just though if the children at Winamac are meeting at all now they might be interested in a little project to help others. If they're not, OK. Even if they are, they shouldn't over do it for postage gets to be too high. We can use pictures from some of the lesson sheets---particularly the real Bible pictures but already have a good many. I find I didn't do wrong by putting in school supplies and probably should have put in more. Must study Lonkundo now!

Tues. AM

I can see now that this is apt to be another overweight letter but we got one from you this AM so have to visit a little more. Yours was written Sept 15, arrived at Coq Oct 5 and we got it Oct.~22. Still little more than a month. We were at the breakfast table when a carrier from Boende brought it in. It might be better if you write Wema in the address, too. Sometimes the letters are flown to Boende without stopping at Coq so we get them a little sooner. Address just like now except add Wema.

So Dale is home and probably married. We sure wish them all the luck and success in the world. Just sorry we couldn't be in on the big times but hope they will write us all about it. I'm not conceited enough to think they'll have time to miss us but know a big time will be had by all. What are they going to do about living conditions? All of you live together for a while?

It seems folks have to make out in unexpected ways these days. I'm sure you'll work it out so everyone will be happy.

It's kind of fun to be here away from the worries of price ceilings, black markets, no goods, etc. If we don't get what we want we think up something else. We wanted (were supposed to have) inlaid linoleum on our floors but MW has none now so we are having a grass mat made for our bedroom floor and will enamel it. If that works will do it elsewhere. Can paint it just as modernistic as we please. The mat will probably cost \$1. \& enough paint for a 16 x 13 room.

I needed a pair of shorts and skirt for itineration but didn't have enough cloth for all so made shorts and blouse from some old material of Miss Ward's, made skirt and bolero of unbleached muslin and trimmed it with the green. Claylon said I looked like Sonjie Heinie so what more could I want?!!

There are only two things wrong with our workmen here: they are slow and they do poor work!! Another fault in this house was that the windows were put on the outside and the screens on the inside so I always had to climb up on a chair to unfasten the top button on the screen, open them and close the windows. In case of quick rain, everything was wet before we could get around. Claylon has 3 carpenters changing them. He thought they could do 4 windows yesterday and hang 2 windows upstairs. Instead they do not have the 4 windows changed yet and will probably be another 2 days doing the entire job. I hardly think you need to worry about sending us cupboard things. We can probably pick them up on our furlough!

With things so hard to get there we probably shouldn't tell you this but I will anyway! Claylon bought a huge bunch of bananas yesterday---weighed 31 kilos (62 lbs.) Only one was ripe so he measured it. It is 6 inches around and 9 inches long. This morning he weighed two small ones---weight V2 lb. each. Cross my heart this is true!! I'm on a diet, you know, so I never eat more than 1 pai-pai (size of cantaloupe) 2 oranges, Vi grapefruit, bowl of oatmeal with a banana, piece of toast and coffee for breakfast. If we have guests we top this off with bacon \& eggs. Our cook makes a salad for the two of us on a meat platter. Uses 2 bananas, 2 tomatoes, 3 or 4 slices of pineapple, and pai-pai balls.

I was afraid we were going to run out of coffee so ordered some from Boende but didn't say how much. They sent one-half bushel, not ground. We all use a coffee mill. Was afraid they wouldn't send any so ordered 5 lbs. from Coq and put in our Leggett order for 1 case of good American brand-12 lbs.! How about a cup of coffee! Claylon prefers tea but I forgot to order that from Boende. Have 2 lbs. coming from Coq (Can get it at native store, though)

Must stop now. Just want to say again we wish the best for Dale \& Doris. If Doris has as good a husband as I have and Dale has as good a wife as I am, they are bound to be happy!! (Conceit?!)

Say, what did you mean, Mom, the things Claylon took pictures of were pretty? Don't you know that just anyplace we would want to shoot it would take a pretty picture. Just wait till you see Wema! I think the woman with the little girl is Mrs.~Don Edwards with Mary Jo Lewis on her lap. Time to study Lonkundo!

Claylon just won't let me quit---he is behind this letter even if I am writing it. Guess we have told you that on weekday mornings w get up at 5:20---just as our house boys come to work but on Sat and Sun we play lazy---stay in bed till 6. By the time we get up, breakfast is on the table, the floors have all been swept, fresh flowers are in all the vases, and a boy is waiting for us to get out so he can make our beds. Then many mornings and nearly every Sun morning there is a gardenia by my plate for me to wear in my hair. Oh, what a life! The cat just got in the drawer and was about to eat up your letter (Better not report this last paragraph outside the family. It

would make good publicity for missionaries I'm afraid!) We really do work hard even though not at housework.

\hypertarget{november}{%
\section{November}\label{november}}

\hypertarget{november-1-1946}{%
\subsection{November 1, 1946}\label{november-1-1946}}

Behold! I am using my own typewriter again after all these long months. Probably I won't even be able to write. I just have to tell you all about yesterday for it was one of those days that only happen once in a lifetime, I hope! Our things came. We have been waiting for this boat since last Fri and it finally arrived on the next Thurs. We were sure it had sunk or done something drastic. The report came up yesterday morning that it was here with 21 boxes for us, and enough for the other two families to make 72 big boxes. Naturally, they had to be carried up on the backs of natives from this far-off state beach so we just dismissed school and put everyone to work. Clay Ion went down to see the things off the boat and the first things he saw were our beds. Our lovely Sealy Tuftless mattresses are both ruined. He was simply sick when he saw them. They had been burned and were then water soaked. In the first place they had not been packed properly by Keating for they were not even wrapped in paper, let alone it being waterproof. They were in thin wooden boxes with the springs in the middle and a mattress on each side. The box had been broken open and then on this river boat they had been stored on the deck and no doubt, a spark fell down in the crack in the wood and caught the mattress on fire. They are both burned in the corners but have great big holes as well as being quite wet so we are going to ask Keating for replacements. The springs seem to be fairly good although are some rusty already. We hope we may be able to patch and dry out the best of the two mattresses to use until we can get others but we certainly intend to claim them as a total loss.

Well, that was bad but when I saw my sewing machine I just couldn't take it anymore and broke down and cried. It looked like a total wreck. It too, was not packed. The machine had not been wrapped in any paper, was packed in a little thin, plywood box, and was not even fastened to the box so one of the casters had punched a hole in one end. It was soaked so that water simply ran from all the packing, a little dab of excelsior in the box. All the glued pieces are apart, the varnish is ruined and at first I though the entire machine was ruined but I do think it will sew. At least the treadle runs the machine so it should work if it is not too rusty. The whole head was loose and had dropped down into the insides of the thing but Claylon got it up and the lid part seems to be all right. Actually I think, and hope, that he will be able to fix it by building a new case. May be able to fix it so that I can use the drop leaf part but am not sure.

Guess you had just as well know all the worst first, for actually it is not all bad though we were two very sick people for awhile yesterday. You remember the two chairs we bought at Seyboldt's and then hired a man to crate for 6.50 per chair? Well, there is not enough wood in the two crates to build a chair. We get disgusted at these people for having such a little sense until we see what supposedly civilized, and educated people do when they are cautioned about the importance of a job and paid well. There was no paper around them either and the crating boards were at least two feet apart on each chair. Just boards at the corners and one down each side. One chair had a hole all the way though the bottom but our surprise was that they got here at all. Of course, they are dirty and somewhat faded and scratched but will not look too bad when cleaned up. Since they just had excelsior bottoms we will stuff some more in the hole and I will have to recover the one chair. We know what folks mean when they warn us about bringing our good things. Well, one box didn't come. The little metal file box that we had packed all of our cosmetics, shaving cream, tooth past and various and sundry little articles. It

may still be on this boat and will get put off when it comes back down river, but we will wait and see.

That's the worst. Now for the rest. Dad, the things that you and Claylon packed were by far the best off. Claylon had another heart failure when he saw them take the barrel off the boat for the water just poured from it and he immediately thought of our jello! We were so sure that things in it would be such a mess that we waited till last to do it. All the packing was soaked it is true, but even most of the jello was still all right. It had been put inside of pans until it was OK. The only casualties were the top of my glass double boiler, which had just a little chipped off around the top and which Claylon thinks he many be able to smooth down, the handle off one small vase, and a comer off that little mirror-covered cosmetic box that Otto's once gave me for Christmas. My big Pyrex roaster, all those thin vases, and everything else was in perfect condition. Much better than the professional packers did on our china. The trunks were all in good condition, all the mirrors, pictures, etc. perfect. The crate of tubs which had the lamps was OK except for the Aladdin which was broken off at the base. All of those lamp globes in that one trunk were in perfect condition. The crate with the boiler looks OK but we haven't unpacked it yet. Naturally, the wind charger crate is solid and so it goes. Everything that you did at home was solid. Anyway we have learned our lesson for the future.

Oh, before I forget, Mom, your parcel post pkg. came yesterday, too. Just three months which is not bad. It looked just like you had packed it. I could just see you working and worrying to get it right. Not even a scratch on the pkg. I am chewing gum so hard now that my jaws ache.

We had a party last night here. The entire station and a state man, 7 of us in all. We had Whitman's Samplers and gum and did we have fun. The Belgian appreciated it, too, for he had been with the American First Army long enough to take up our ways. I like all the patterns and just wish I could sit down to my machine and sew some new dresses.

Well, we are here now. Today things don't look so bad. We had to expect some casualties in coming this far, I guess. However, we do intend to send Keating a airmail and tell him what we think of his foreign packing and put in a big insurance claim. He is supposed to have entire coverage and we expect him to replace the beds and pay well for the other damage.

Today is Friday and school is closing for a month. Everyone is excitedly getting started home and I have to go to the office now to bank their money or probably let them get out what they have. We will have next week to get straightened up before we start on our trip.

Forgot to mention your tea. Claylon is going to appreciate some good American tea for it is better than what we can get. We don't realize the difference until we smell the better. The chocolate is as good as new and sealed as it is, will stay for a long time. Thanks so much for the box. It was just next to a visit home to have the things which I knew you had worked over so carefully---a Christmas gift in advance.

\begin{center}\rule{0.5\linewidth}{0.5pt}\end{center}

\hypertarget{november-5-1946}{%
\subsection{November 5, 1946}\label{november-5-1946}}

This has to be the last of this but will add a word before I seal. Everything looks bright again. Shouldn't even send this about our things but it is written. I patched one mattress and it looks real good and sleeps fine. Dale's bedspread tops it off fine. The sewing machine had been oiled and it runs like a new one. I made a mosquito net on it last night. It had no drawers, no front, no back but the fundamentals are here so that Claylon can build new ones and fix it up like new. He had been looking for excuses to use his tools anyway. He let one of the carpenters use the saw here in the house yesterday and he saw USA on it and said: USA must be a big store for a lot of good things come from there! Well, we still think USA is a pretty good place, too.

I'm marking out the part about sending us hinges, etc. for the postage is so high and we will probably have them from MW before we have things ready anyway.

I had a lovely letter from Vera and she says Billy is doing fine again. He did limp but has been going to a Children's Home for treatment until he is nearly well.

Again I am sending you the carbon of this for I hope you can read both sides which you could not do on the original. Give our love to all the family and tell them we could always use more letters. We did have a nice one from Lola which we appreciated very much.

Love

Helen \& Claylon

Back of page so I'll just write another page---Helen used more space than the law allowed and I almost got left out entirely. Just decided to start another page. Postage is high but we won't know the difference in a hundred years.

Part of Helen's letter sounded (page torn) and rightfully so. However the blues are now all gone for Helen did a wonderful job of mattress patching. They look nice and sleep wonderfully. As she said the sewing machine is my job. The machine is a dandy but the cabinet must be entirely rebuilt.

Today has been a busy one. We have worked all day getting things put away. The food is now all arranged on our storeroom shelves, excess pots, pans, tubs, buckets, etc. are along the wall. It looks like a store in reality. While we were working two of my carpenters were working in the house. They happened to see some of our pictures. The ones of you folk, Dale \& Doris, etc. The folk here have always bragged of how pretty Helen is. Well when they saw the hand-painted photo of Mom \& Dad, they really howled. But whose picture were they howling over? Bet Dad can guess. Yes it was his. They created so much noise till some other workmen came in to view the wonder. Their expression was ``Ise ekae ale jituka mongo.'' Literally it means ``her father is beauty itself'' it is their most emphatic expression. They also thought Dale very beautiful. The funny thing was that the women were pretty a little while the men were the very symbol of beauty. Well fellows (assuming that Dale is there) guess that will hold the women in their places for a while.

Do hope Dale is there, the wedding all over and all of you getting along fine. Sure hope work clothes are available now for I really played havoc with what Dale had last winter. I looked at our pictures today and looked long at the ones which Helen made of Dad and I picking corn last fall. I recognized just about every patch on the coveralls.

I can't go beyond this page this time. Do intend to write soon and hope to get my say in first or I will not have a say. Our love to everyone and thanks so much for the delicious candy which came in perfect condition. We now have sweet teeth and sore jaws from candy and gum. But oh how we like it. Continue to keep us posted on all the latest news and we will try to do the same from this end. We expect to have a good story when we return from our trip to the back country. Bye now and all our love,

Helen \& Claylon

\hypertarget{section-1}{%
\chapter{1949}\label{section-1}}

\hypertarget{january}{%
\section{January}\label{january}}

\hypertarget{january-26-1949}{%
\subsection{January 26, 1949}\label{january-26-1949}}

Dear friends,

The Christmas season has come and gone. Was it as joyous for you as it was for us? Our most wonderful gift was the arrival on Christmas morning of baby Linda Lou! She is a present that grows sweeter as the days go by. She weighed 8 lbs. 6 oz. was twenty inches long and is a perfect beauty. We had all been worrying if her arrival was going to interfere with other Christmas plans but if she times all her arrivals as well as she did this one, she nor her friends will have to worry. The Christmas pageant by the village school had been on Friday afternoon, then the Christmas story in pictures was shown in an outside service on Christmas eve after which we missionaries had our gift exchange and a little party. Than Linda arrived at 10:00 A.M. and that night all enjoyed the annual Christmas dinner.

The Congolese were almost as tickled as we were over it all, especially the fact that she came on noel. They all wanted to see her. Some of our people had never seen a white baby. One of the women remarked--why she leeks just like one of our babies. Their babies are white for the first few days. It is the custom to name each white person for some Christian native who has recently died. Just after one of our young couples had made all preparations to go to the Congo Christian Institute at Bolenge for their further training, the wife died in child birth. Linda Lou is now Botoka Malia in her honor.

\hypertarget{may-1}{%
\section{May}\label{may-1}}

\hypertarget{may-28-1949}{%
\subsection{May 28, 1949}\label{may-28-1949}}

Linda is growing like a weed. She has two teeth, sits alone and crawls as big as life. She absolutely refuses to stay on her blanket, I have a boy watch her to keep her on but he was gone for a minute and Claylon said he thought she would be O.K. as she was lying quietly but by the time he had put on a tie she
had gotten clear off the blanket and was scooting across the fleer. she cries so seldom that Mrs.~Boyer has been declaring that she couldn't but this afternoon Linda showed her. We went down to visit the new state people for tea and everyone was saying their ``hellos'' and making over Linda and she really let loose and kept on until I was almost ashamed of her after having bragged about her conduct. She isn't used to crowds and even the place looked strange as we have scarcely had a chance to have her away from home.

\hypertarget{september-1}{%
\section{September}\label{september-1}}

\hypertarget{sept.-13-1949}{%
\subsection{Sept.~13, 1949}\label{sept.-13-1949}}

(From the Penn Sheraton Hotel in Philadelphia)

Hi from good old U.S.A.! We docked about noon here at Philadelphia but by the time we finished customs and got to the hotel it was nearly 5 P.M. so Ells (Lewis) and Claylon left Lil and me with all the kids and dashed off to New York for the evening. They were to see Emory Ross, pick up any mail, get our clergy permits, etc. and spend the evening with the Rosses. So Lil and I had the fun of taking all the young ones right from the heart of Africa to a restaurant. Mary Jo, at 6 1/2 should be sort of self reliant but she's all eyes and ears and can't pay much attention. Casey at 2 1/2 is impossible. Knows no fear and pays no attention to anyone. Then Mark at 13 mo. and Linda at 9 are our babies. Our first obstacle was the revolving door that Casey had to stop and examine when half way through. Red lights meant nothing to them but the red and white stripes on the barber pole across the street was a great fascination. We finally managed to get supper ordered and eaten but not without some struggles but half-way through Mary Jo says, ``My, but isn't this fun and aren't we glad our husbands aren't here?'' Lil has harnesses for both little boys so she harnessed them up and we `drove' out of the restaurant. We have been wondering ever since we arrived at the hotel if people stare at us because we look as queer as we feel, if it's the children, or if we're just not used to people.

Linda Lou is a little dear. She has acted all day as if she knew something exciting was on foot and she didn't want to miss out. She hasn't been fussy but just jumps up and down and won't let us out of her sight for fear she will be left behind. She is a wonderful traveler, seems to love every minute of it and every person she sees.

We had an uneventful crossing. Just 15 days from Matadi and very wonderful weather. I'm no sailor so was sick a lot but that gave Claylon lots of experience at his housewifely duties and he came through fine. It was just the opposite with the Lewises. Ells was sick so Lil had it all to do. Still can't realize we've arrived. Can you imagine - 3 months ago yesterday we left Wema. Nothing speedy about that, is there?

(During the rest of 1949 and 1950 we were in the U.S. so there are no long letters about any of us. Since Ronnie was born in a hospital at Rochester, Indiana on Feb.~27, 1950 there is not much recorded about his earliest days or antics. It is only when we start traveling again that he comes into his own.)

\hypertarget{section-2}{%
\chapter{1950}\label{section-2}}

\hypertarget{december}{%
\section{December}\label{december}}

\hypertarget{december-20-1950}{%
\subsection{December 20, 1950}\label{december-20-1950}}

(From a publicity letter written just before I left the U.S.)

Linda Lou, Ronnie Dee and I are sailing January 5, 1951 on the Queen Mary and hope to be met at Cherbourg, France by Claylon who left the U.S. in August far a year's study in Brussels, Belgium. I am anticipating my trip more now that I an sure my sister, Hilda Mitchell, is going along. She has a leave of absence from the Methodist Hospital in Indianapolis and will have the experience of the trip and of seeing Europe for about seven weeks.

\hypertarget{section-3}{%
\chapter{1951}\label{section-3}}

\hypertarget{february}{%
\section{February}\label{february}}

\hypertarget{february-3-1951}{%
\subsection{February 3, 1951}\label{february-3-1951}}

(From a letter of Hilda's to the folks)

Ronnie is walking with holding to someone with one hand but won't walk alone yet--the lazy bum! But he is cute as a bug's ear. Played in the play pen for about an hour this morning before I put him to bed. He makes all kinds of noises, says Ma Ma, Da Da and grins all over the place-is fat as a little butterball and eats everything he can lay hands on. Linda came strolling out last night at about 9:45 after I got here (Hilda had just returned from a trip to Rome) - we were talking too loud as usual. She surely hadn't forgotten me in a week for she hugged me good and hard and crawled all over me. When she finally fell asleep, she had her Book of Prayers open, dropped down on her chest just as if she had fallen asleep reading her prayers.

\hypertarget{february-9-1951}{%
\subsection{February 9, 1951}\label{february-9-1951}}

(Again fron Hilda)

Helen and Claylon have gone te town to sign Helen and the children in at the American Consulate. I am baby sitting. Oh! what kids your grandchildren are. Linda has learned to crawl in and out of the play pen by herself and does it often. She pulls a chair up to the side and climbs over. She is very independent and has to be handled with kid gloves part ef the time to get her to do what she must. She is still as sweet and loving as ever most of the time. Ronnie is growing like a weed, and in a very short time he is going to be defending himself adequately when his sister starts pushing him around. Incidentally, she does that quite often. He has taken a step or two a few times but still is not walking by himself. Their ``Auntie Bea'' brought them each a helium filled balloon last night and you should have seen the excitement. Ronnie's eyes got as big as silver dollars, and he was so excited he shook all over.

\hypertarget{february-21-1951}{%
\subsection{February 21, 1951}\label{february-21-1951}}

We have been having baby troubles! Briefly it's that Linda doesn't want to go to bed and no matter whether we put her in at 7:30 or 10:30 she wants to wake Ronnie and play. She can climb into his bed and I know of no way to keep her out except to solidly screen it. We have moved everything movable away from it but still she gets in. We've pleaded, entertained her for long sessions, spanked, threatened, and have done everything we know to do except lay her out but still we have problems. It is better than when Hilda was here. The day that Hilda left I lay down with Linda and she was asleep in 10 minutes so thought that was the solution but last night and tonight that didn't work, though it does for afternoon naps.

Ronnie still doesn't walk alone much. Hilda may tell you about his first steps. Claylon bought some popcorn and fed Ronnie 2 or 3 grains. Then he put some on a chair which just happened to be about 2 steps away and Ronnie simply flew over to them and then repeated the performance twice more but he hasn't done it since. Of course he hasn't had any popcorn since!

Later: Today (the 24th) Ronnie took several steps and does he think he's somebody! It sticks out all over him. He has been fussy again lately but today one of those bottom teeth is finally through and he was as chipper as a squirrel. The other gum is swollen about 3 times normal size so surely only a few more days for it. If he thinks he is not getting his share of attention he throws up his hand and says, ``Hi, there.'' and then looks like the cat that swallowed the canary. Quite often I wish you had him at least for a while. He absolutely refuses to be fed. Everything is to be put before him and he helps himself - sometimes even with a spoon. He'll eat eggs for breakfast and leave the cereal because I won't give it to him to feed to himself. Two little independent ones we have and you should see them when they cross one another. Ronnie doesn't take much now but I can already see we are going to have to watch the discipline. If we scold Linda for mistreating him, then he cries at any little thing so he'll be petted. If we have Linda love him, then she hits him so he'll cry and she can put her arms around him and say, ``Don't cry, Ronnie. Linda loves you.''

\hypertarget{march}{%
\section{March}\label{march}}

\hypertarget{march-28-1951}{%
\subsection{March 28, 1951}\label{march-28-1951}}

(This from Claylon)

Wish you could see the kids. Ronnie is entirely a walker now. He has a fight at the table because he wants to hold his own spoon and he usually wins. He can do it well enough for his age but you should see him after having chocolate pudding, which incidentally is the favorite dessert for both the kids. He loves books as well as Linda ever did and has a few of his own though he is prone to be very hard on them.

Linda is doing so many things. She talks like a parrot. She repeats any kind of word or phrase which she hears in addition to making her own. Her favorite now is telling everyone to be careful. ``Be careful, mommy. Be careful Ronnie.'' And she always wants to be kissed when she hurts herself. Her mommy is always kissing her head or fingers. The best one was when she slid down and sat down hard hurting her backside and then came out saying, ``Mommy, kiss it.''

\hypertarget{april}{%
\section{April}\label{april}}

\hypertarget{april-22-1951}{%
\subsection{April 22, 1951}\label{april-22-1951}}

Linda pulls some fast ones now and then. Yesterday she wiggled so I was quite sure she needed to go to the bathroom but once we go there I was having quite a struggle to get her to sit. Was in fact holder he with main force when she says very emphatically, ``Now Mommy, don't make me mad!'' And then when I still held on, ``Mommy, I mean it, you're going to make me mad!'' Wonder where she ever heard that. She talks constantly simply gasping for breath sometimes until I think I will surely go crazy.

Big old Ronnie has taken to pushing chairs around, climbing up onto them and then going wherever he likes. A few days ago I found him sitting on the desk surrounded with fountain pens, pencils, etc. Then just this evening Bea found him in the kitchen sink. We from shoes on up. He is so pesky quick that you think you know where he is and he's up and into something more interesting.

\hypertarget{may-2}{%
\section{May}\label{may-2}}

\hypertarget{may-13-1951}{%
\subsection{May 13, 1951}\label{may-13-1951}}

Ronnie gets cuter all the time. He climbs into and onto everything. They both get on the table in our room; up into the window or anywhere else available. When he gets spanked for going places he is not allowed he pouts and puckers up just like Linda used to even if he isn't hurt a bit. He has quite a vocabulary, we think, for 15 months for he says bow-wow whenever he sees a dog either in a picture or on the street. He says book, points to one and wants to be read to. He also says spoon, high chair, and wanna-go-out so that they are quite easily understood. He has 13 teeth with the three others showing a bit white as if it wouldn't be long but I can't see that he has any more hair than when we left and what he has isn't long enough to hardly stand up by itself. I am beginning to think he is going to just be a baldy. He is good at pulling Linda's hair and today I saw her trying to get hold of his without success. He is a little old dickens about some things. He still wants a bottle at night but we never know when. It may be midnight or 6am. He wakes up, grunts a little, then talks a little and we hope he'll talk himself to sleep, then soon he begins to get angry and so gets louder and louder. Ignoring him does no good unless we could put up with it all night. I think I could but Claylon is afraid the neighbors might not like it so after so long a time he gives in. Ronnie doesn't bother to stand up or get out from under his covers, he just lies on his tummy and yells. This A.M. he had his head turned to the wall so didn't even bother to look at me. He could hear me coming so just stuck his arm out behind his back, took to the bottle, grunted and went to work.

\hypertarget{may-22-1951}{%
\subsection{May 22, 1951}\label{may-22-1951}}

The package with the children's books came the other day and have by now been read through and through. Ronnie is as bad as Linda ever was about a ``book, a book'' and he hangs on to my tail until he gets it and has it read to him. He is different in that instead of sitting quietly and looking at it himself he walks around with it opened in front of him and read aloud. His reading aloud and talking is equal to what his crying used to be in volume! He jabbers lots more than Linda ever did but he is beginning to get quite a vocabulary. He climbs like a monkey, gets up on and then off Linda's tricycle all alone. Gets all seated and then wants to be pushed. They are some pair. Linda says ``Where's my Ronnie? Ronnie boy, where are you?''

\hypertarget{july-1}{%
\section{July}\label{july-1}}

\hypertarget{july-8-1951}{%
\subsection{July 8, 1951}\label{july-8-1951}}

Linda loves jig-saw puzzles better than most anything. We asked what she wanted from London and it was a puzzle and a book so that's what she got. After help in putting a 15 piece puzzle togetehr once she can figure it out herself and will do it over and over for an hour at a time. They are big wooden pieces, very lovely and durable.

They didn't miss us in the least, it seems. (We had gone to London from Monday till Friday.) Bea said Linda asked for her daddy once and when she first saw me Saturday morning she languidly says, ``Oh, I see Mommy's back.'' I said, ``How are you, Linda?'' and she very formally replied, ``i'm just fine, thank you.'' and that was that.

There are both three day measles and mumps in the neighborhood. This afternoon Linda has been extra cross and has tried that her ear aches so don't know if its cold or mumps but soon shall I suppose.

\hypertarget{july-2}{%
\section{July}\label{july-2}}

\hypertarget{july-15-1951}{%
\subsection{July 15, 1951}\label{july-15-1951}}

Linda really has the mumps but they don't bother her in the least. One side is real fat, the other just enough that I can tell it mostly by feeling. She eats anything, sweet or sour and is not the least bit sick. However, for three days before she swelled up she had quite a bit of a fever. It has all been most peculiar for Ronnie had one, too. For almost one whole day his temp was 104.2 without any other symptom of illness. By evening it was nearly normal, the next day he was O.K. and has been fit as a fiddle ever since. I have expected him to be swollen up, break out with measles or something, but they have not even been cross as they sometimes are after such a flare-up. Guess there are just a lot of mysteries in the rearing of children.

We are not going to the country this week what with the measles and one thing and another but hope to go next week. It is so cold here today folks are out in overcoats again.

\hypertarget{august-1}{%
\section{August}\label{august-1}}

\hypertarget{august-9-1951}{%
\subsection{August 9, 1951}\label{august-9-1951}}

I wonder if you can imagine how glad we are to be able to be writing dates in August? It means September is getting closer but not a minute too soon. This is the sort of a time one is glad not to have to live through too often. The Colonial Course exams are on. Claylon leaves in a few minutes and will have 5 this afternoon. He has already had 3 and will have 3 more Monday and then it's over -- be it good or bad. So far he feels he has passed but the hardest come this afternoon so tonight we'll see how he feels. If writing State Boards or taking Bar Exams are any worse I don't see how either the contestants or their families survive. If we don't get away from here soon Claylon is going to have ulcers unless he already does for he complains constantly. If the profs are all really fair there is no real danger of any of our folks not passing but one never knows.

They have to study and be able to give details on such things as Belgian History and Congo History back to the Stone Age (or whatever was first); education history and the present set up in Congo (and it's really complicated); hygiene; laws and government of Congo, etc., etc. Belgian state officials going out for the first time have to take the same course and exams and even they consider it tough. We have been living in a sort of vacuum fo so long I don't remember what I wrote last. Anyway I think I told you about going to the Lewises every day for 10 days while Lil lived here as her youngsters were exposed to three day measles. She was 2 months pregnant and hadn't had them so Dr.~Watson said she must take no chances.

It was during that time that Linda had mumps. I had just been home 2 or 3 days when Ronnie's fat little old jaws got still fatter. He is finished now, too, but it sort of kept us inside during the few days that were pretty enough for us to be out. Now it's cold and rainy again. We had about 2 weeks of fairly good summer but acts like fall is with us now. I wouldn't live here by choice if they gave us the place for that and other reasons.

The lady I had for a baby sitter went to Holland for 3 weeks at least 5 weeks ago and I haven't heard of her since so about all I do is act as baby sitter. They are both naturally bored with everything they have inside to play with and so long as they had mumps there was no place to take them without exposing others and now it rains. Oh, me. How do millions of mothers do it in the city? I just know one thing -- they don't all live with someone else where nothing is private, not even your own thoughts!

\hypertarget{september-2}{%
\section{September}\label{september-2}}

\hypertarget{september-14-1951}{%
\subsection{September 14, 1951}\label{september-14-1951}}

(Aboard the ``Elizabethville'' on the way to Matadi)

Today is wonderful -- smooth, warm and sunny and everybody is happy after having been seasick most of the time up till now. Sunday we are to dock at Tenerife (Canary Islands) for nearly half day and there are excursions planned for us. This is tropical so from then on we'll get rid of all our warm clothes.

I think our accommodations are better than on the Queen Mary probably because there is only one class so we are all treated alike. We have a big cabin - 3 beds and a folding baby bed for Ronnie with our own private bath equipped with shower.

Our meals have been good with lots of everything and my children eat as if they had been starved for years. They could eat cookies and other goodies continuously and still put away full sized meals. Today I had a special birthday cake with the compliments of the Commanding Officer and his crew. There is a coffee bar which serves ice cream sodas and sundies. There is a hair dresser with whom I have an appointment for tomorrow as part of my birthday from my hubby.

The nursery is on the open deck as well as a room inside so it is very nice for the children except that as usual ours don't like to stay alone. Since lots of the parents stay there and deck chairs are available it is easy to watch them and just being in sight is all they ask.

\hypertarget{october-1}{%
\section{October}\label{october-1}}

\hypertarget{october-20-1951-wema}{%
\subsection{October 20, 1951 (Wema)}\label{october-20-1951-wema}}

I started school with Bill and Norma (Horner) this week so from 8:30-12 we do that and Marjorie has been keeping my infants. Everything is going smoothly and I'm sure all Bill needs is lots of patience, praise and confidence that he can do O.K. I was really surprised at how well he does and he is so much better physically than before. Norma is a little whiz bang at her work but spoiled enough that she sometimes likes to get off too easy. Anyway they are well brought up children and very pleasant to work with.

We are going to build a yard fence and then I hope to keep ours corralled at least part of the morning so Marjorie will be free. I have a woman, Malia, who follows them around and keeps them out of trouble. It seems like it will work out nicely. Norma and Linda haven't hit it off too well yet as Norma thinks Linda is too little. Bill plays better with them but I notice the girls are doing better each day. It is wonderful for all of us to have some freedom. Horners think the kids are wonderful and the smartest things they every saw!

Marjorie says Ronnie is just like you, Dad. His build, his walk and mostly his expression. He has a way of wrinkling his forehead and puckering his mouth when he says
no" or ``I do'' very determinedly that she thinks is like you would do it. They think Linda is more like Claylon. She took her first penicillin needle without wimpering and they were both agast. Both Bill and Norma watched and they were too and said, ``She's better than we are for we cry.'' Then the next day she took the next one just as calmly and I think Marjorie and Howard were both about in tears. Said they had never before seen a child take shots without crying and especially when she knew what was coming so think that she has some friends for life.

\hypertarget{november-1}{%
\section{November}\label{november-1}}

\hypertarget{november-3-1951}{%
\subsection{November 3, 1951}\label{november-3-1951}}

We are sort of getting settled into a routine now so that it is hard to imagine that we were gone so long and that have so recently returned. At the moment the Horners and we are alone on the station as Heimers left a few days after we arrived to get their things down from Mondombe. They should be in on the boat in two or three days.

Everybody thinks Ronnie is about the last word in cuteness, it seems. Of course, he is at the age when everything he does looks that way but he knows it too! He rolls those big brown eyes around and knows he'll get a laugh. They are both so much easier to manage here and seem so much happier all the time. Instead of them always being under foot and all of us wondering what on earth to do they are up and out as quickly as possible. I have a woman follow them around and she keeps them in sight of the house and helps to keep them interested in their play. However, she doesn't have to entertain them all the time like we used to in the house. She just sort of acts as sentry. Ronnie loves to follow the goats and I think he would follow them all the way into the forest without any fear whatsoever if she didn't stop him. They play together now much better than they used to.

They love bananas better than anything. I asked Eale yesterday how many they had eaten and by the middle of the afternoon it was 4 big ones apiece. Then they each had some for supper. Marjorie is afraid we'll go broke this term, and she never saw children eat like these do.

I was glad to hear from Doris all about Randy and his development. Especially how he gets too busy to get to the bathroom on time. I'm going to tell her I am more and more in agreement with those who say boys are more difficult than girls. I never had any trouble with Linda, but Ronnie -- we haven't even started to make any progress and it even makes him firey mad to have his pants changed. Just would rather wear them as is than be bothered.

\hypertarget{december-1}{%
\section{December}\label{december-1}}

\hypertarget{december-31-1951}{%
\subsection{December 31, 1951}\label{december-31-1951}}

Guess this is the last time I'll use that date. Just think it will be seven years tomorrow since I agreed to thinks sort of life! Hardly seems ike it could be that long and yet a tremendous amount of living has been done during those years.

Right now I am very much aware of it. Just out the window is our play yard where about seven or eight children are playing. For the most part the little ones do very well together but now and then I hear a yell indicating that at least two of them are wanting the same thing at the same time. The Cardwell children (the family has spent Christmas with us) have marvelled at our play yard and say it is just like a park. Mary Ann (4) told her daddy she wants a ``skidding'' board when she gets home. The swing, teeter-totter, slide and sand box furnish entertainment for them all.

We had a good Christmas. Linda was tickled with her dolly and doll buggy but some pets and pans and sand toys from Santa tickled her just as much. Even before she had hardly looked into the buggy Ronnie blew a whistle from Santa and she says, ``Don't blow that thing, Ronnie. My baby is asleep and I don't want her awake yet!'' At the breakfast table she was paying no attention to any of us and saying in a sort of singsong voice, ``I'm just as happy as I can be!'' So we decided Christmas and Happy Birthday had been a success.

\hypertarget{section-4}{%
\chapter{1952}\label{section-4}}

\hypertarget{february-1}{%
\section{February}\label{february-1}}

\hypertarget{february-11-1952}{%
\subsection{February 11, 1952}\label{february-11-1952}}

Guess I'd just as well start off telling you some more cute things your grandchildren do. They are the ``book'' craziest children I ever saw I think. Don't know what that is a sign of, if anything, but it almost gets to be a nuisance sometimes. If I so much as sit down on a chair, no matter what the time of day, and one of them sees me in they come for a story. We have read nursery rhymes until I could say them in my sleep and by now even Ronnie can recite fairly long portions of several of them. Linda does them with no coaching, scores of them. But the thing that really made his daddy swell up with pride was the other day when Ronnie exhibited his counting ability. Claylon had been helping Linda count to 5, then immediately Ronnie spoke up 6, 7. We thought it a coincidence and tried him out several times but always he had the right answer. Then the next morning Claylon said 1, 2 and Ronnie follows 3, Claylon 4, Ronnie 5, and so on up to 7 so naturally his daddy is convinced he is just smart. Some of the time lately I have been letting them come upstairs to school if they will be fairly quiet. Ronnie doesn't stay long but Linda would keep at her ``studies'' all morning if I would let her. One day I wanted her to go down and play and she says no, I just have to study my spelling.

Within a few months I plan to order the Calvert kindergarten course so that if she is really interested next year she can do some work. Think it may be easier to manage her that way than to insist she stay down and play outside.

Guess yon had as well know all our troubles at once. We are expecting another little Weeks about September 1! That isn't trouble except that since I haven't quite reached the third month yet I am sick as a dog most of the time so am no help either in the work or the morale of things around here. It is
encouraging to know that I am no ways nearly as bad off as when either of the first two were on the way. About like with Ronnie, I guess. I have been able to keep at school each morning but when our hot afternoons hit then I'm done for the day. Another 2 to 4 weeks should find me feeling fine, I hope, then maybe everything will look brighter. We're happy about the baby, of course, for Ronnie will be 2 1/2 by then and then maybe we won't be traveling with a tiny baby again.

\hypertarget{april-1}{%
\section{April}\label{april-1}}

\hypertarget{april-12-1952}{%
\subsection{April 12, 1952}\label{april-12-1952}}

It is so hot and dry that one can hardly get through the afternoons and by night when the children are in bed are so exhausted from putting up with the heat and it's usually still hot, that all one can do is go to bed. The heat is bothering blacks and whites alike and causing more sickness than the doctors have most ever seen. There has been an epidemic of dysentery that really lays them out. It is tropical disease but up until last year they say we had never been bothered around here. Last year while the Horners were gone and the nurses ran out of medicine several patients died. There have been no deaths this year but at the last count there had been about 75 cases treated at the hospital. There were 25 from the boys' house in one day. It's from the low water, for the folks drink fron the streams and when the streams are badly contaminated and the water isn't high enough to keep moving on, they get sick. Claylon has been boiling the water for the school boys lately and that has helped.

Then there have been a lot of cases of polio. There have been three of the nurses' children at the hospital had it just recently and several others from near here---even an adult or two. We are keeping all the children away from crowds as much as we can.

I told you earlier about the measles. Marjorie gave all the children shots and though ours both had the measles, neither was very sick so guess they really helped. They weren't even very cross but were broken out like genuine measles.

We acquired a new member of the household this A.M. A Belgian couple are leaving and wanted a good home for their dog. When they had visited with us we noticed that Ronnie was making up to the dog especially well. (He has always been a little afraid of dogs.) I had always been opposed to such young children having dogs but what could I say when he was loving it and when I was asked for my opinion in front of all the visitors and family. So-- it is a lovely one, about three years old, out of its baby ways and well house broken, though is definitely a house dog. Since it understands only French, both Ronnie and Linda are beginning to talk to it in French. Ronnie suggested this afternoon that it take its nap in bed with him but we quickly vetoed that. Since it slept under one of their beds, don't know how long it will be till we find them all curled up together.

Linda has been picking up a lot of her elders' expressions. She rattles off ``for goodness sakes'' and ``my gracious'' just like her mother. Then she has Marjorie's ``it's simply wonderful'' down pat. She says, ``My gracious, can't you see that I can't do that right now. I am simply too busy.'' But Ronnie is the one with a mind of his own. we have been trying, though not so very hard, to impress correct bathroom procedures on him. He goes willingly and alone IF it is all his idea but just let someone else suggest it! One day while Marjorie was here he grabbed himself so I rushed him to the bathroom and Very Much against his will he did everything right. Then he threw himself and had a regular tantrum. ``I didn't want to go on the potty!'' He must have yelled for 5 min. Marjorie repeated that to Howard who, of course, thought it about the funniest thing he ever heard of (since it wasn't his child, I expect.) They used to think Billy was abnormal with lots of tantrums but it looks perfectly OK to them in Ronnie. Another one he has, Dad, is sometimes at the table when he doesn't want to say ``Thank you''. He's not like you in not being thankful because he doesn't want the food, he just wants it right now and not after time for a blessing. Sometimes halfway through the meal he is ready for time out.

\hypertarget{may-3}{%
\section{May}\label{may-3}}

\hypertarget{may-13-1952}{%
\subsection{May 13, 1952}\label{may-13-1952}}

My spirits as well as my physical condition seem to have revived with the coming of cooler weather. Hope its a lasting condition. My school will be out in less than a month and I am hoping to get a lot of housework done as well as other things. I have an assignment to write another Lonkundo text, I need to take lots of pictures, l want to work in the yard, paint some furniture, write a form letter, and would like to take three or four days off to go with Claylon into the back country for a look into schools there and a bit of a vacation. Then we should start school by the middle of July since we'll have to take a few days off during the fall.

The children are as cute and noisy as ever. Linda knows we are going to have a babv one of these days and is convinced that it is to be a brother. We were discussing names today and I was saying that if it is a girl Hilda wants her called Susan and Linda pipes up ``And if it's a boy it will be called Tommy.'' She says it's all right for Ronnie to have a baby sister but she wants a baby brother.

I am being besieged with questions now. Linda picks up a book -- ``Now, Mommie, who makes that book? Who makes this picture?'' Tonight at supper, ``Where did we get this pineapple?'' ``Well, it grew in the dirt and sand.'' ``What is sand?'' ``What makes flowers grow?'' I say, ``God'' knowing I'm letting myself in for something. ``Where is God? How does He make the flowers grow?'' Then I wish she had a grandmother or two here to help me out. The books I have read don't seem to give the answers, they just say all questions should have sensible answers. Oh, me.

Ronnie has taken up Randy's old trick of calling his daddy by his name. In the morning it's ``Where's Claylon? Claylon, breakfast is ready. CLaylon, read me a story.'' As Claylon says, if everyone could say his name as plainly as Ronnie surely we wouldn't get so many letters addressed incorrectly. I heard Linda tell him one day, ``That's Daddy.'' But Ronnie get her with ``But his name is Claylon.'' Linda's favorite expression this week is, ``Oh Mother, I'm sooooooooooooo--excited!''

\hypertarget{may-27-1952}{%
\subsection{May 27, 1952}\label{may-27-1952}}

I thought I'd get this done before the noisiest half of the family got up fron his nap but no such luck. Linda has been quietly sitting on the floor drawing but now she tells me, ``My little Ronnie boy is getting up!'' Sometimes she is very motherly and condescending to him and takes great delight in pleasing hin but the spells are usually short lived. She says now that he is such a happy little boy!

They are messily having iced drinks in ``breakable'' glasses. A few days ago they grew up and any more are highly insulted to be served in their tin plates and plastic glasses. As Linda says, I'm growing bigger
every day, and now I'm big enough for a ``breakable'' plate. Of course, whatever she does is good enough for little brother.

\hypertarget{june-1}{%
\section{June}\label{june-1}}

\hypertarget{june-3-1952}{%
\subsection{June 3, 1952}\label{june-3-1952}}

Ronnie and Hal, Jr.~play outside all morning long until they are both so tired they can hardly eat lunch. Ronnie is going to miss Hal, Jr.~for they leave July 3 and they are real good playmates. Hal, Jr.~is just 4 months younger than Ronnie though a lot smaller. He holds his own so sometimes we hear some real squalls but not too often. Linda is more prone to play by herself.

She is quite interested with the new arrival. When asked what she wants, she says, ``A baby brother.'' Well, what if it's a sister? ``Oh, Ronnie may have a sister but I want a brother!'' Of course, their daddy thinks they are real cute and especially when Linda begins acting subtle to get her way. The other day I was rushing around from one job to another when she says, ``Mommy, aren't you awfully tired?'' I admitted I was getting that way. Her comeback was, ``I just thought so. Now you come right over here and sit down
on this nice soft couch and read me a story!'' So what could I do?

\hypertarget{june-30-1952}{%
\subsection{June 30, 1952}\label{june-30-1952}}

My `vacation' is nearly over---2 more weeks to go, and I am going to have to keep moving or I won't get everything done that I had planned. I haven't done too badly though for took out some days whan I just was plain lazy and didn't do a thing. I am so anemic that I take iron pills all the time and still am on the low borderline. Then my blood pressure is so low that Marjorie says its a wonder I cam move at all. Most of the time, though, I feel real good and when I do have sometimes overdone the work, such as housecleaning, then have to take it easy for awhile.

Horners had a doctors' convention here over last week end and that meant entertaining for us, too. Since the Keane Watsons stayed here we were delighted to have them. Then Bakers from Mondombe were here, too, for several meals. The others were Belgians. Anyway I had wanted to clean house so thought this was a good goal to set to get it done before they came. We just about made it but it took some doing. It was amazing the amount of dirt we got out. I guess it was sort of like spring housecleaning at home but one would think that with hired help all the time we could at least keep the house clean but they don't even see cobwebs and dust unless they have their noses stuck in them. After spending several days in the kitchen, Eale told me one day, ``I just don't like this.'' I asked him why not, if he didn't like to work in a clean kitchen or if he just didn't like to clean it. But he just laughed so I still don't know. Of course, compared to the way they live we must seem too particular for words. He is clean about his work and keeps the kitchen in good enough shape generally but his cupboards get as messy as they do for any of us and he just can't be bothered with cleaning them until we make an issue of it.

I am enclosing a picture that Linda cut out of her color book. I thought it was pretty good for a 3 1/2 year old without any help whatsoever. She gets terribly provoked when she doesn't do a good job of following the lines. She still sucks her thumb but I think that is about her worst habit. I wish she would quit but we have tried all sorts of talk and persuasion and nothing helps though she does often promise that when the new baby comes she won't do it any more so it won't learn how!

This is the last week of Claylon's school until the first of August. He is so glad to have it out and so am I. He can't keep up this pace just on and on. Now with Sat. classes there is no break whatsoever for him. He is up at 5:30 every morning. Sunday maybe it's 6, and stays right on the job until 6:30 at night. Then goes to church 3 nights a week, usually works int the office afterwards or has native palavers until lately he hasn't been going to bed until 10:30 or 11:00. I believe the pressure is worse than it has ever been and we can see no hope for its getting better. This next term of school he'll have that as usual and any evangelistic work and palavers that come up. Heimers leave this Thursday and Davises, who will be new to the work can't possibly get here until in January.

We are planning to make one short trip into the back country together and then he is going to make a long one by himself where he will be riding his bike a lot and won't have all the teachers and pupils, too, on his neck every day.

\hypertarget{july-3}{%
\section{July}\label{july-3}}

\hypertarget{july-23-1952-bofanya}{%
\subsection{July 23, 1952 (Bofanya)}\label{july-23-1952-bofanya}}

We are out on my vacation so I'll begin by starting a letter to you. We came out ot this village this morning and plan to stay for two nights. The children are home with Marjorie so it's a real vacation for me. I guess it's the first time I've been away from them since we went to London for even as much as a day or part of a day.

They both seemed so thrilled with the possibility of sleeping at Aunt Marjorie's that they didn't mind our leaving at all. I though there might be trouble at the last but no way. They both kissed us, waved good-bye and then ran to play as if New, that's over.

I started the children's school again last week and Linda began `school' too now so she is really very `busy'. Actually her kindergarten is no different from what we were doing except in her eyes. She draws, colors, cuts out, pastes, uses modeling clay and as soon as I have the courage to introduce it will have poster paints. This leaves Ronnie along of a morning except that Linda usually gets tired by 10 A.M. and willingly joins him. I was afraid he'd cause trouble but on the contrary he has been having a wonderful time with two little native boys who come up every morning to play with him. They are just his age -- one is the preacher's boy and the other's a teacher's.

When I knew he was going to be alone so much I asked if there were any little boys who would like to come up to play. On Sunday about two weeks ago a mother brought her little boy up to see if he could come. The family is one of our best back country leaders - he is a teacher-evangelist. They have lost several babies because the mother can't feed them enough. She now has a baby eight months old that Horners have been giving milk all the time. The little boy, just exactly Ronnie's age, had been mission fed, too, and was a little husky. I agreed he could come. That afternoon Marjorie and I were out on the road when we met the daddy come in from his village to get the family as Marjorie had agreed the baby was old enough to go home whenever they wanted to leave so that took care of the play. Anyway just two weeks later the father came carrying the little boy in to the hospital with something terribly wrong. Horners thought it was a n intestinal obstruction that happens in perfectly healthy children but causes death unless operated immediately. Howard operated but instead of finding the type of thing he expected, found an obstruction but it was from som very bad infection. They said as if something poison had reached there, though it was local enough they couldn't figure out how it could have gotten there without affecting the stomach. We were all quite concerned about the little fellow and quite as they expected, he died last evening. Their home village is here in Bofanya though he had been teaching in another village nearby. The father thinkshe knows the cause of the death.

He found the heathens in the village where he was teaching smoking ``hemp'' marajuana and did what he is supposed to -- reported them to the state man who put them in prison. He says often the families get revenge by grabbing a child and giving it some of their heathen medicine -- poison, of course. Marjoirie and Howard agree that this is very likely what happened.

I wanted to come out here to visit the school and just sort of get the feel of real village life again so that's what I plan to do tomorrow. Claylon may have some baptisms but hopes mostly to get in some hunting and real recreation.

July 29 -- Tuesday A.M. While Eale finishes preparing breakfast I'll add a note tho. I hear trouble in the Jr.~Dept. Ronnie usually gets up early 5 to 5:30 happy as a lark but this A.M. he is cross as a bear.

We got back Fri. night and it was good to get home though guess everything ran smoothly in our absence. Marjorie says a piece of chocolate candy will settle any difficulty in our family -- from Ronnie's wanting Mommy at bedtime to a noisy quarrel. Howard says he thinks we should ask for a living supplement to feed our two. They all marvel at how much they eat. Bill and Norma always have to be coaxed but not these!

Linda asked me the other day to draw her a rooster. I was too busy at the time besides the fact I can't draw anyway, even a good looking rooster so I put her off by telling her to go to her daddy for that was more in his department. A few minutes later she came up to me and says, ``Mommy, just what is your department?'' There she had me.

A few days ago at the table I heard Ronnie saying something over and over almost to himself and learned he was saying ``Norma goat, Norma goat.'' When I plied for an explanation Linda told me, ``Well, you know there are Billing Goats so Ronnie thinks there are Norma Goats, too.''

One can't even use the old spelling system with them any more for Linda, at least, is sure to ask now just what are you spelling? One day at mealtime I said -- for daddy's ears alone -- I guess we won't have any c-a-k-e for dinner today when Linda pipes up, ``I guess I want some c-a-k-cake for dinner! They got it.''

\hypertarget{august-2}{%
\section{August}\label{august-2}}

\hypertarget{august-20-1952}{%
\subsection{August 20, 1952}\label{august-20-1952}}

Ronnie's chief occupation now is to gather caterpillars. Last night some fell out of his pocket at the supper table on to his plate before we could get them out of the way. He has no inhibitions about them and told me one day after his nap that he had a bucket full of worms to take to bed with him and that his daddy would not let him! He always thinks they need a bath when he does. Right now he is ``Daddy's Boy''. For several days he simply refused to let me do a thing for him which didn't make me mad at all. I said I sort of hope the phase lasted but it is already passing.

I am desperately busy. There was a while when I felt like I simply did not have time to have a baby but am getting things taken care of so that maybe it will be OK to spend at least a few days in bed.

\hypertarget{september-3}{%
\section{September}\label{september-3}}

\hypertarget{september-9-1952}{%
\subsection{September 9, 1952}\label{september-9-1952}}

Your latest grandson put in his appearance last Saturday A.M., September 6th at 8:00 A.M. He is Thomas Claylon popularly known as Tommy. He didn't quite keep up with his brother and sister in that he only weighed in at 8lbs 2 oz. Even so, think he'll make the grade! He, his mommy, and all the family are doing nicely. This is the first peaceful moment that there has been to write. Since mail goes tomorrow we decided that was as quick a way as any to let you know what he arrived on schedule.

Ronnie and Linda are now quite pleased and want to see, feel, feed and play with him all the time. Sometime ago Linda had changed her mind and decided she wanted a Susan, so far a little while she was disappointed as she seemed to have her heart set on Susan. But yesterday she accepted the change by saying, ``Our next baby will be Susan, won't it, Mommy?''

Tommy may or may not look like the other babies. I can't even remember except he has one very decided difference -- he has quite a crop of dark brown hair! He at this stage -- 3 1/2 days is quite contented and is making his schedule to please us. (The dark hair didn't last.)

The mail came yesterday with 3 bundles of magazines from you and the gift box of dresses! Wasn't that wonderful timing -- just when I could read a bit and I'll have something new when I get going. Also before my birthday. The package came in just 2 months and had no duty so that makes it extra special. It could well have had four or five dollars on it and parcel post usually takes 4 months now so it worked out fine.

This has been a real three-ring circus with very little change in sight though everything has worked out better than expected. I wrote we didn't expect help. That day Howard was in Boende and who should be there by plane that same day but Stobie. Remember Miss Stober was with me for Linda's birth and there's nobody better to have around. We rejoiced exceedingly. The next evening V.G. and A.C. Cuppy arrived from Monieka with her sick in bed with nausea. You may remember she is the one who had to go home beofre with the same trouble. Dr.~Watson who is the doctor at Monieka thinks its all in your head so he wasn't doing anything for her. At 2 1/2 months when she arrived here she had lost 20lbs and was in bed. So she stayed here with little Dwight -- 18 months -- while A.C. went back to Monieka and packed up. They are due for furlough anyway so are leaving Thursday from Boende right to New York by plane. They, with Stobie, have all been at our house, upstairs, all the time. Then the Belgian woman had her baby O.K. on Tuesday before Tommy came on Saturday and as she hadn't much help Stobie nursed her. Though the delivery was at Horners I came home Saturday P.M. so you can imagine the confusion here Sunday and Monday -- 9 people -- 4 children from 3 1/2 to 1 day, 2 bed patients, one up and one downstairs, with 2 men and one nurse in attendance. 'Twas a gay time but as I said I'm doing fine. Still enjoying the bed for the most part.

Cuppy's leave Thursday but that same day the Tillers from Lotumbe are to arrive with their two school age children. Marilyn and Herbie are to live with us and go to school with Bill and Norma for 3 or 4 months anyway. If I haven't told you about that I'll have to do it later. They are older so won't be any trouble!! Herbie is 10 or 11 and Marilyn is about two years older.

\hypertarget{october-2}{%
\section{October}\label{october-2}}

\hypertarget{october-13-1952}{%
\subsection{October 13, 1952}\label{october-13-1952}}

I have no illusions about getting anything done on this as it is just 15 minutes till time to go to school and I have dinner to plan beside I heard Tommy wailing meaning something must be done for him. I suppose I have been this busy before but I sure can't remember when. 5 children plus a school teaching job. Whew!

In re-reading your letters, Mom, I see you hoped we would have either a peaceable little boy or girl. We have a little boy, all right, but I wouldn't say he was any more peaceful than the other one was and, if you remember, his daddy thought he was well nigh impossible at times. I guess he is ordinarily good but he doesn't sleep any 20 to 22 hours a day as the books advocate for the first few weeks. I guess the most noticeable thing about it is that he seems most apt to cry during our supposedly rest hour after dinner, or about 10 at night.

knew he vas Yoing to be alone so much I asked if there were any
lfttle boys who would 11ke to come up to play,
a mother brought her little boy up to see if he could come. Tbe fewily 1s one of our best back country leaders- he fsa teacher-evangelist They have
lost several babies because the can't feed thea enougb, She nov has a baby eight nont hs old that Horners have been giving milk all the time,
The little boy, just exactly Ronnie' had been mission fed, toos and was
they wanted to leave so that took care of the play, Anyway just two weeks later the father caue carrying the little boy in to the hospital with sowe thing terribly Horners thought it was an intestinal obstruction that bappens in perfectly bealthy children but mauses death unless operated inmediately,
a little busky, I agreed he could come
Thit afternoon Marjorle and I were the road when we net the daddy coming in from his v1llage to get the
out on
family as Marjorie had agreed the ba by was old enough to go home whenever
Howard operated but instead of finding
obstruction but it was from some very bed infection,
On Sunday about tvo weeks ago
the type thing be expected, found an
They said as if ometbing thougb it was local enough they coulda't figure out
poison bed reached there,
how it could have gotten there without affecting
quite cor.cerned about the l1ttle fellov and quite as they
Ve were all expected. he died
Their home village is here in Bofanya though he bad beep teach-
last evening,
1ng in another village near by. death

  \bibliography{book.bib,packages.bib}

\end{document}
